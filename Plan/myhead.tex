%------------------Includings -----------------------
\usepackage{amsopn} % for \DeclareMathOperator
\usepackage{amstext,amsxtra,amssymb,amsmath}
\usepackage{subfig} % subfloat
\usepackage{stmaryrd} % for \llbracket and \rrbracket
\usepackage{paralist} % inline list
\usepackage{caption}
%\usepackage{fdsymbol} %diamond
\usepackage{empheq} % for \xmapsto
\usepackage{tikz} % for \circled
\usepackage{fancyvrb}
\usepackage{pstricks}
\usepackage{pst-2dplot}
\usepackage{pgfplots}
\usepackage{cite}
%\usetikzlibrary{pgfplots.groupplots}
%\usepackage{pbox}

\usepackage[oldcommands,boxruled]{algorithm2e}
\usepackage{algpseudocode}
\usepackage{algpascal}
\usepackage{algc}

\usepackage{multirow,tabularx}
\renewcommand{\arraystretch}{1}


\usepackage{mathptmx}       % selects Times Roman as basic font
\usepackage{helvet}         % selects Helvetica as sans-serif font
\usepackage{courier}        % selects Courier as typewriter font

\newcommand\mycommfont[1]{\footnotesize\ttfamily\textcolor{blue}{#1}}
\SetCommentSty{mycommfont}

%-----------------------------------------------------

%-----------------ALGORITHM2e--------------------------
\usepackage{algorithm2e}
\newsavebox{\mycode}

%----------------- Colors --------------------------
\definecolor{naranja}{RGB}{241,70,34}
\definecolor{verde}{RGB}{104,198,135}
\definecolor{dred}{RGB}{120,7,7}
\definecolor{darkgreen}{rgb}{0.0, 0.42, 0.24}
%--------------------------------------------------

%----------------- Commands --------------------------
\newcommand{\tet}[1]{\textcolor{naranja}{#1}}
\newcommand{\new}[1]{\textcolor{dred}{#1}}
\newcommand{\comment}[1]{\textcolor{verde}{#1}}
\newcommand{\hsep}[1]{\hspace{7pt} #1}
\newcommand{\conf}{$s$}
\newcommand{\af}{{\sc POSL}}

\newcommand{\opch}{{\it open channel}}
\newcommand{\opchs}{\opch{\it s}}
\newcommand{\dopch}{{\it data open channel}}
\newcommand{\oopch}{{\it object open channel}}
\newcommand{\om}{{\it operation module}}
\newcommand{\oms}{\om{\it s}}
\newcommand{\m}{{\it module}}
\newcommand{\ms}{\m{\it s}}
\newcommand{\cstr}{{\it computation strategy}}
\newcommand{\cstrs}{{\it computation strategies}}
\newcommand{\commch}{{\it communication channel}}
\newcommand{\commchs}{\commch{\it s}}
\newcommand{\cm}{{\it compound module}}
\newcommand{\cms}{\cm{\it s}}

\newcommand{\lbk}{\left\llbracket}
\newcommand{\rbk}{\right\rrbracket}
\newcommand{\produce}{&\rightarrow}
\newcommand{\OR}{\hspace{5pt}|\hspace{5pt}}

\DeclareMathOperator*{\argmax}{arg\,max}

\newcommand*\circled[1]{\tikz[baseline=(char.base)]{
		\node[shape=circle,draw,inner sep=2pt] (char) {#1};}}

\newcommand{\defop}[2]{\left\{\begin{tabular}{#1} #2 \end{tabular} \right.}	

\newcommand{\stage}[3]
{
\vspace{5mm}\noindent\underline{{\bf #1 stage:} #2} \vspace{5mm}

\hfill\begin{minipage}{\dimexpr\textwidth-5mm}
#3
\end{minipage}
\vspace{5mm}
}

%\newtheorem{definition}{Definition}[section]

\newcommand{\defname}[2]{ \begin{definition}\textbf{(#1)}
	#2
\end{definition}}

\newcommand{\TM}{$^{\scalebox{0.6}{\mbox{TM}}}$}
\newcommand{\R}{$^{\scalebox{0.7}{\textregistered}}$}
\newcommand{\curiosiphyfull}{Intel\R{} Xeon\TM{} E5-2680 v2 (10$\times$4 cores, 2.80GHz)}
\newcommand{\curiosiphy}{Intel\R{} Xeon\TM{}}

\usepackage{listings}
\usepackage{fancyvrb}
\fvset{fontfamily=courier,fontsize=\scriptsize,numbers=left,framerule=.3mm,numbersep=1mm,commandchars=\\\{\}}

\usepackage{array}
\newcolumntype{L}[1]{>{\raggedright\let\newline\\\arraybackslash\hspace{0pt}}m{#1}}
\newcolumntype{C}[1]{>{\centering\let\newline\\\arraybackslash\hspace{0pt}}m{#1}}
\newcolumntype{R}[1]{>{\raggedleft\let\newline\\\arraybackslash\hspace{0pt}}m{#1}}

\setcounter{totalnumber}{4}
\renewcommand{\topfraction}{0.85}
\renewcommand{\bottomfraction}{0.85}
\renewcommand{\textfraction}{0.2}
\renewcommand{\floatpagefraction}{0.8}
\renewcommand{\textfraction}{0.2}
\setlength{\floatsep}{8pt plus 2pt minus 2pt}
\setlength{\textfloatsep}{8pt plus 2pt minus 2pt}
\setlength{\intextsep}{8pt plus 2pt minus 2pt}

\usepackage[multiple]{footmisc}