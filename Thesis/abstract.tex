\cleardoublepage
\newgeometry{left=3.2cm,right=2.3cm,top=1cm,bottom=0.7cm} % Change margins locally
\begin{vcentrepage}
\noindent\rule[2pt]{\textwidth}{0.2pt}\\
%\begin{center}
%{\Large\textbf{Spectroscopic Tools for Quantitative Studies of DNA Structure and Dynamics}}
%\end{center}

{\large\textbf{Resumé:}\\}
The goal of this thesis is to develop new quantitative fluorescence-based tools for studying the three-dimensional structure and dynamics of DNA and RNA. The work may be divided into three interconnected sub-topics: 1) Development, 2) characterization, and 3) use of synthetic fluorescent DNA modifications. In addition, a continuous goal was to develop general methodologies for the quantitative evaluation of fluorescence-based experiments, including the development of user-interfaced software.

A central theme throughout the thesis is Förster resonance energy transfer (FRET), an energy transfer phenomenon which is widely used as a "molecular ruler" for monitoring distances and interactions at the nanoscale level. However, measuring quantitative nanoscale distances using FRET is highly challenging. This research field is reviewed in Paper I. In the pursuit for an improved quantitative FRET toolbox we developed "base-base FRET": a FRET pair system consisting of two DNA base analogues. This technique facilitates a very high control of both the position and orientation of the FRET probes relative to the nucleic acid which allows more information to be obtained from the data. Paper II reports the characterization of base analogue FRET acceptor, tC$_\mathrm{nitro}$. The information gained from this study, such as the direction of the lowest energy electronic transition dipole moment, was vital in order to use base-base FRET quantitatively.

Paper III concerns the development of a new generic method called FRETmatrix for analysing FRET experiments in nucleic acids quantitatively. Paper III demonstrates how base-base FRET in combination with FRETmatrix can provide quantitative information about the three-dimensional structure and dynamics of nucleic acids. In relation to Paper III, Paper IV reports a reversible five-state DNA switch with readable fluorescence output. This paper additionally demonstrates how FRETmatrix can be used to model any type of FRET system in nucleic acids.

The development of new fluorescent DNA base analogues with improved brightness and photostability is a challenging field. Paper V reports new insight into the quenching processes of the tC base analogues as well as into the their potential energy surfaces important for their adaptability into various constrained (biological) environments. Paper VI reports a new fluorescent adenine analogue, qA, and describes its photophysical and nucleobase mimicking properties.



\noindent\rule[2pt]{\textwidth}{0.8pt}
\end{vcentrepage}

%\newcommand{\fpd}{fulleropyrolidin\xspace}
%\newcommand{\Fpd}{Fulleropyrolidin\xspace}
\begin{vcentrepage}
\noindent\rule[2pt]{\textwidth}{0.2pt}\\

{\large\textbf{Resumé:}\\}
Formålet med denne afhandling er at udvikle nye kvantitative fluorescens-baserede værktøjer til at studere den tre-dimensionelle struktur og dynamik af DNA og RNA. Afhandlingen kan groft inddeles i tre forbundne temaer: 1) Udvikling, 2) karakterisering, og 3) anvendelse af syntetiske fluorescerende DNA modifikationer. Derudover er det et løbende formål at udvikle mere generelle metoder til at evaluere fluorescens-baserede eksperimenter kvantitativt, hvilket inkluderer udvikling af brugerflade-styret software.

Et centralt tema gennem afhandlingen er Förster's resonans-energioverførsel (FRET), et energioverførselsfænomen der kan anvendes som en "molekylær lineal" til at overvåge afstande og interaktioner i nanoskala-størrelsesordenen. Det er imidlertid ret udfordrende at anvende FRET til at måle kvantitative afstande. Artikel I giver en gennemgang af det nyeste inden for dette forskningsfelt. I jagten efter en forbedret kvantitativ FRET-værktøjsboks udviklede vi "base-base FRET": et FRET-par system bestående af to DNA base analoger. Denne teknik faciliterer en øget kontrol over både positionen og orienteringen af FRET-proberne relativt til DNA molekylet, hvilket betyder at der kan opnås mere information fra eksperimenterne. Artikel II rapporterer karakteriseringen af base-analogen, tC$_\mathrm{nitro}$, med henblik på dens anvendelse som FRET probe. Informationen opnået under dette studie, såsom retningen af overgangsmomentet, var vital for at anvende base-base FRET kvantitativt.

Artikel III omhandler udviklingen af en ny generisk metode kaldet FRETmatrix til at analysere FRET eksperimenter i DNA kvantitativt. Artikel III demonstrerer hvordan base-base FRET i kombination med FRETmatrix kan give kvantitativ information omkring den tre-dimensionelle struktur og dynamik af DNA. I relation til artikel III rapporterer artikel IV en reversibel kontakt med fem stationære tilstande og et aflæseligt fluorescens-output. Denne artikel demonstrerer yderligere hvordan FRETmatrix kan anvendes til at modellere et hvilken som helst FRET system i DNA.

Udviklingen af nye fluorescerende DNA base analoger med forbedret lysstyrke og stabilitet er et udfordrende forskningsfelt. Artikel V rapporterer ny indsigt i quenching-processerne af tC base analogerne samt deres potential energi-overflader. Sidstnævnte spiller en rolle for hvordan disse prober tilpasser sig forskellige biologiske miljøer. Artikel VI rapporterer en ny fluorescerende adenin analog, qA, og beskriver dennes fotofysiske egenskaber samt dens evne til at efterligne adenin i DNA.



\noindent\rule[2pt]{\textwidth}{0.8pt}
\end{vcentrepage}


\restoregeometry
