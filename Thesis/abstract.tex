\cleardoublepage
\newgeometry{left=3.2cm,right=2.3cm,top=1cm,bottom=0.7cm} % Change margins locally
\begin{vcentrepage}
\noindent\rule[2pt]{\textwidth}{0.2pt}\\
%\begin{center}
%{\Large\textbf{Spectroscopic Tools for Quantitative Studies of DNA Structure and Dynamics}}
%\end{center}

{\large\textbf{Resumé:}\\}
The goal of this thesis is to develop new quantitative fluorescence-based tools for studying the three-dimensional structure and dynamics of DNA and RNA. The work may be divided into three interconnected sub-topics: 1) Development, 2) characterization, and 3) use of synthetic fluorescent DNA modifications. In addition, a continuous goal was to develop general methodologies for the quantitative evaluation of fluorescence-based experiments, including the development of user-interfaced software.

A central theme throughout the thesis is Förster resonance energy transfer (FRET), an energy transfer phenomenon which is widely used as a "molecular ruler" for monitoring distances and interactions at the nanoscale level. However, measuring quantitative nanoscale distances using FRET is highly challenging. This research field is reviewed in Paper I. In the pursuit for an improved quantitative FRET toolbox we developed "base-base FRET": a FRET pair system consisting of two DNA base analogues. This technique facilitates a very high control of both the position and orientation of the FRET probes relative to the nucleic acid which allows more information to be obtained from the data. Paper II reports the characterization of base analogue FRET acceptor, tC$_\mathrm{nitro}$. The information gained from this study, such as the direction of the lowest energy electronic transition dipole moment, was vital in order to use base-base FRET quantitatively.

Paper III concerns the development of a new generic method called FRETmatrix for analysing FRET experiments in nucleic acids quantitatively. Paper III demonstrates how base-base FRET in combination with FRETmatrix can provide quantitative information about the three-dimensional structure and dynamics of nucleic acids. In relation to Paper III, Paper IV reports a reversible five-state DNA switch with readable fluorescence output. This paper additionally demonstrates how FRETmatrix can be used to model any type of FRET system in nucleic acids.

The development of new fluorescent DNA base analogues with improved brightness and photostability is a challenging field. Paper V reports new insight into the quenching processes of the tC base analogues as well as into the their potential energy surfaces important for their adaptability into various constrained (biological) environments. Paper VI reports a new fluorescent adenine analogue, qA, and describes its photophysical and nucleobase mimicking properties.



\noindent\rule[2pt]{\textwidth}{0.8pt}
\end{vcentrepage}

\input{abstract_danish}

\restoregeometry
