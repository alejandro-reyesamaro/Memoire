\cleardoublepage
\begin{vcentrepage}
\noindent\rule[2pt]{\textwidth}{0.8pt}\\
\begin{center}
{\Large\textbf{\titre}}
\end{center}
{\large\textbf{Short abstract:}\\}
For a couple of years, all processors in modern machines are multi-core. Massively parallel architectures, so far reserved for super-computers, become now available to a broad public through hardware like the Xeon Phi or GPU cards. This architecture strategy has been commonly adopted by processor manufacturers, allowing them to stick with Moore's law. However, this new architecture implies new ways to design and implement algorithms to exploit its full potential. This is in particular true for constraint-based solvers dealing with combinatorial optimization problems. Here we propose a Parallel-Oriented Solver Language (\posl, pronounced "puzzle"), a new framework to build interconnected meta-heuristic based solvers working in parallel. The novelty of this approach lies in looking at solver as a set of components with specific goals, written in a parallel-oriented language based on operators. A major feature in \posl{} is the possibility to share not only information, but also behaviors, allowing solver modifications during runtime. Our framework has been designed to easily build constraint-based solvers and reduce the developing effort in the context of parallel architecture. \posl's main advantage is to allow solver designers to quickly test different heuristics and parallel communication strategies to solve combinatorial optimization problems, usually time-consuming and very complex technically, requiring a lot of engineering.

{\large\textbf{Keywords:}}
Constraint programming, meta-heuristics, language.

\noindent\rule[2pt]{\textwidth}{0.8pt}
\end{vcentrepage}
