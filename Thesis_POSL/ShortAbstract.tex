\cleardoublepage
\begin{vcentrepage}
\noindent\rule[2pt]{\textwidth}{0.8pt}\\
\begin{center}
{\Large\textbf{\titre}}
\end{center}
{\large\textbf{Short abstract:}\\}
The multi-core technology and massive parallel architectures are nowadays more accessible for a broad public through hardware like the Xeon Phi or GPU cards. This architecture strategy has been commonly adopted by processor manufacturers to stick with Moore's law. However, this new architecture implies new ways of designing and implementing algorithms to exploit their full potential. This is in particular true for constraint-based solvers dealing with combinatorial optimization problems.

Furthermore, the developing time needed to code parallel solvers is often underestimated. In fact, conceiving efficient algorithms to solve certain problems takes a considerable amount of time. In this thesis we present \posl{}, a Parallel-Oriented Solver Language for building solvers based on meta-heuristic, in order to solve Constraint Satisfaction Problems (CSP) in parallel. The main goal of this thesis is to obtain a system with which solvers can be easily built, reducing therefore their development effort, by proposing a mechanism of code reusing between solvers. It provides a mechanism to implement solver-independent communication strategies. We also present a detailed analysis of the results obtained when solving some CSPs. The goal is not to outperform the state of the art in terms of efficiency, but showing that it is possible to rapidly prototyping with \posl{} in order to experiment different communication strategies.

{\large\textbf{Keywords:}}
Constraint satisfaction, meta-heuristics, parallel, inter-process communication, language.

\noindent\rule[2pt]{\textwidth}{0.8pt}
\end{vcentrepage}
