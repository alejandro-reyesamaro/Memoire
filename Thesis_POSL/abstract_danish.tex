%\newcommand{\fpd}{fulleropyrolidin\xspace}
%\newcommand{\Fpd}{Fulleropyrolidin\xspace}
\begin{vcentrepage}
\noindent\rule[2pt]{\textwidth}{0.2pt}\\

{\large\textbf{Resumé:}\\}
Formålet med denne afhandling er at udvikle nye kvantitative fluorescens-baserede værktøjer til at studere den tre-dimensionelle struktur og dynamik af DNA og RNA. Afhandlingen kan groft inddeles i tre forbundne temaer: 1) Udvikling, 2) karakterisering, og 3) anvendelse af syntetiske fluorescerende DNA modifikationer. Derudover er det et løbende formål at udvikle mere generelle metoder til at evaluere fluorescens-baserede eksperimenter kvantitativt, hvilket inkluderer udvikling af brugerflade-styret software.

Et centralt tema gennem afhandlingen er Förster's resonans-energioverførsel (FRET), et energioverførselsfænomen der kan anvendes som en "molekylær lineal" til at overvåge afstande og interaktioner i nanoskala-størrelsesordenen. Det er imidlertid ret udfordrende at anvende FRET til at måle kvantitative afstande. Artikel I giver en gennemgang af det nyeste inden for dette forskningsfelt. I jagten efter en forbedret kvantitativ FRET-værktøjsboks udviklede vi "base-base FRET": et FRET-par system bestående af to DNA base analoger. Denne teknik faciliterer en øget kontrol over både positionen og orienteringen af FRET-proberne relativt til DNA molekylet, hvilket betyder at der kan opnås mere information fra eksperimenterne. Artikel II rapporterer karakteriseringen af base-analogen, tC$_\mathrm{nitro}$, med henblik på dens anvendelse som FRET probe. Informationen opnået under dette studie, såsom retningen af overgangsmomentet, var vital for at anvende base-base FRET kvantitativt.

Artikel III omhandler udviklingen af en ny generisk metode kaldet FRETmatrix til at analysere FRET eksperimenter i DNA kvantitativt. Artikel III demonstrerer hvordan base-base FRET i kombination med FRETmatrix kan give kvantitativ information omkring den tre-dimensionelle struktur og dynamik af DNA. I relation til artikel III rapporterer artikel IV en reversibel kontakt med fem stationære tilstande og et aflæseligt fluorescens-output. Denne artikel demonstrerer yderligere hvordan FRETmatrix kan anvendes til at modellere et hvilken som helst FRET system i DNA.

Udviklingen af nye fluorescerende DNA base analoger med forbedret lysstyrke og stabilitet er et udfordrende forskningsfelt. Artikel V rapporterer ny indsigt i quenching-processerne af tC base analogerne samt deres potential energi-overflader. Sidstnævnte spiller en rolle for hvordan disse prober tilpasser sig forskellige biologiske miljøer. Artikel VI rapporterer en ny fluorescerende adenin analog, qA, og beskriver dennes fotofysiske egenskaber samt dens evne til at efterligne adenin i DNA.



\noindent\rule[2pt]{\textwidth}{0.8pt}
\end{vcentrepage}
