\chapter{Tuning methods for local search algorithms}
\label{chap:prior_paramils}
%\textit{In this chapter are presented Prior works leading to \af. In Section~\ref{sec:relaxation} we explain how to tackle \csps{} by modeling it through a continuous optimization problem, as a first attempt aiming for the right direction in order to find the proper approach. In Section~\ref{sec:split} we present a brief work where we applied the the {\it problem subdivision} approach to solve the {\it k-medoids problem} in parallel. Finally we present in Section~\ref{sec:paramils} a study applying the {\sc ParamILS} tool in order to find the optimum parameter configuration to {\it Adaptive Search} solver.}
\textit{In this appendix a prior work leading to \posl{} is presented: a study applying the {\sc ParamILS} tool in order to find the optimum parameter configuration to {\it Adaptive Search} solver.}

\vspace{2ex}\vfill
\minitoc
\newpage

%\textcolor{green}{
%My idea here is to present the work with ParamILS. We worked with the sets of training and test instances, and we think that the key is there, because we obtained a little change in the parameters configuration. The other point is that maybe Costas Array is not a good example to work, because the difference between the runs are too small. We hope that working with other problems, including more training instances, we can obtain better results.}

% short paragraph: about autotuning of parameters; work in progress.

\section{Introduction}

In this appendix we present some results applying {\sc ParamILS} (version 2.3)\footnote{Open source program (project) in {\it Ruby}, available at \href{http://cs.ubc.ca/labs/beta/Projects/ParamILS}{\texttt{http://cs.ubc.ca/labs/beta/Projects/ParamILS}}}, a stochastic local search approach for automated algorithm parameter tuning, first introduced by Hutter, Hoos and St\:utzle in 2007. The source is available on internet and includes some examples that you can run and see how the tool works. In addition, it brings a complete User Guide with a compact explanation about how to use it with a specific solver \cite{Hutter2008,Hutter2009}. In this study we used it to tune {\it Adaptive Search} solver\footnote{An implementation from Daniel D\'{i}az available at \href{https://sourceforge.net/projects/adaptivesearch/}{https://sourceforge.net/projects/adaptivesearch/}}. %It is an open source program (project) in {\it Ruby}, and the public source code include some examples and a detailed and complete \textit{User Guide} with a compact explanation about how to use it with a specific solver . 

The first step was to build a {\it wrapper} in C++ language, in order to tune more than one problem with the same code. The goal of doing this is using the tool to tune the solver, \ie finding the best parameter setting for a specific problem, but also the best parameter setting to solve any kind of benchmark (a general parameter configuration).

\nocite{Rickard}

Table~\ref{table:param} presents the parameter list to be tuned:

\begin{table}[ht] 
\centering 
\begin{tabular}{c c l}
\hline\hline
Parameter & Type & Description \\ [0.5ex]
\hline
-P & PERCENT & probability to select a local min (instead of staying on a plateau) \\
-f & NUMBER & freeze variables of local min for NUMBER swaps \\ 
-F & NUMBER & freeze variables swapped for NUMBER swaps \\ 
-l & LIMIT & reset some variables when LIMIT variable are frozen \\ 
-p & PERCENT & reset PERCENT of variables \\ [1ex]
\hline
\end{tabular} 
\caption{Adaptive Search parameters}
\label{table:param}
\end{table} 

In this section, the implementation and the experimentation process are explained in details.

\section{Using ParamILS}

To use the tool {\sc ParamILS}, we have installed Ruby 1.8.7 in our computer. We used a laptop \mylaptopName (\mylaptopProc, \mylaptopMemo) with {\sc Ubuntu~14.4}. To run the tool, we needed to use the following command line:

\begin{BGVerbatim}
>> ruby param_ils_2_3_run.rb -numRun 0 -scenariofile /.../<scenario_file> -validN 100
\end{BGVerbatim}

where \texttt{$<$scenario\_file$>$} is the name of the file where we need to introduce all the information that {\sc ParamILS} needs to tune the solver (the \textit{tuning scenario file}). Its content is explained in details in the next section.

\section{Tuning scenario files}

The {\it tuning scenario file} is a text file with all needed information to tune the solver using {\sc ParamILS}. It includes where to find the solver binary file, the parameters domains, etc. For this experiment the {\it tuning scenario file} looks like the following:

\begin{shadedbox}
	\texttt{algo = ./QtWrapper\_wrapper\\
		execdir = /.../src \\
		deterministic = 0 \\
		run\_obj = runtime \\
		overall\_obj = mean \\
		cutoff\_time = 50.0 \\
		cutoff\_length = max \\
		tunerTimeout = 3600 \\
		paramfile = instances/all\_intervals-params.txt \\
		outdir = instances/all\_intervals-paramils-out \\
		instance\_file = instances/.../all\_intervals-lower-instances.txt \\
		test\_instance\_file = instances/.../all\_intervals-upper-instances.txt \\
	}
\end{shadedbox}

Details of each line in this file are the following:

\begin{itemize}
	\item \textbf{\texttt{algo}} $\rightarrow$ An algorithm executable or a call to a wrapper script around an algorithm that aims the input/output format of \textit{ParamILS} (the wrapper).
	\item \textbf{\texttt{execdir}} $\rightarrow$ Directory to execute \textbf{\texttt{algo}} from: "cd $<$\texttt{execdir}$>$; $<$\texttt{algo}$>$" 
	\item \textbf{\texttt{run\_obj}} $\rightarrow$ A scalar quantifying how "good" a single algorithm execution is, such as its required runtime.
	\item \textbf{\texttt{overall\_obj}} $\rightarrow$ While \textbf{\texttt{run\_obj}} defines the objective function for a single algorithm run, \textbf{\texttt{overall\_obj}} defines how those single objectives are combined to reach a single scalar value to compare two parameter configurations. Implemented examples include {\bf mean}, {\bf median}, {\bf q90} (the 90\% quantile), {\bf adj\_mean} (a version of the mean accounting for unsuccessful runs: total runtime divided by number of successful runs), {\bf mean1000} (another version of the mean accounting for unsuccessful runs: (total runtime of successful runs + 1000 x runtime of unsuccessful runs) divided by number of runs -- this effectively maximizes the number of successful runs, breaking ties by the runtime of successful runs; it is the criterion used in most of Frank experiments), and {\bf geomean} (geometric mean, primarily used in combination with \textbf{\texttt{run\_obj}} = \texttt{speedup}. The empirical statistic of the cost distribution (across multiple instances and seeds) to be minimized, such as the mean (of the single run objectives). \footnote{We use {\bf mean} but maybe we can experiment with other values}
	\item \textbf{\texttt{cutoff\_time}} $\rightarrow$ The time after which a single algorithm execution will be terminated unsuccessfully. This is an important parameter: if chosen too high, lots of time will be wasted with unsuccessful runs. If chosen too low the optimization is biased to perform well on easy instances only.
	\item \textbf{\texttt{tunerTimeout}} $\rightarrow$ The timeout of the tuner. Validation of the final best found parameter configuration starts after the timeout.
	\item \textbf{\texttt{paramfile}} $\rightarrow$ Specifies the file with the parameters of the algorithms. 
	\item \textbf{\texttt{outdir}} $\rightarrow$ Specifies the directory where \textit{ParamILS} creates a text file with the results.
	\item \textbf{\texttt{instance\_file}} $\rightarrow$ Specifies the file with a list of training instances. 
	\item \textbf{\texttt{test\_instance\_file}} $\rightarrow$ Specifies the file with a list of test instances.
\end{itemize}

Another important file that we have to compose properly is the {\it algorithm parameter file}, just following the instruction from \cite{Hutter2008} --\textit{[...] each line lists one parameter, in curly parentheses the possible values considered, and in square parentheses the default value [...]}. The algorithm parameter file looks as follows:\\

\begin{shadedbox}
	\texttt{P \{20, 25, 30, 35, 40, 45, 50, 55, 60\} [50]\\
		f \{0, 1, 2, 3\} [1]\\
		F \{0, 1, 2, 3\} [0]\\
		l \{0, 1, 2, 3\} [1]\\
		p \{1, 2, 3, 5, 10, 20\} [5]
	}
\end{shadedbox}

In the current Adaptive Search implementation, the solver binary file and the problem instance are the same thing. It means that we only have to use the following command to solve the All--intervals problem of size $K$, for example: 

\begin{BGVerbatim}
>> ./all-intervals K
\end{BGVerbatim}

So, to use {\sc ParamILS} we modified a little the code: now our solver takes the size parameter from a text file. That way, the instance file is a text file only containing a number.

The solver we want to tune using {\sc ParamILS} ({\it Adaptive Search} in this case), must aims specific input/output rules. For that reason instead of modifying the current code of {\it Adaptive Search} implementation, we preferred to build a wrapper.

\section{Building the wrapper}

The algorithm executable must follow the input/output criteria presented below: 

\textbf{\large Launch command:} 

\begin{BGVerbatim}
>> <algo_exectuable> <instance_name> <instance-specific_information> ...
<cutoff_time> <cutoff_length> <seed> <params>
\end{BGVerbatim}

\begin{itemize}
	\item \texttt{$<$algo\_exectuable$>$} Solver 
	\item \texttt{$<$instance\_name$>$} In our case, a text file containing only the problem size
	\item \texttt{$<$instance-specific\_information$>$} We don't use it 
	\item \texttt{$<$cutoff\_time$>$} Cut off time for each run of the solver (see above)
	\item \texttt{$<$cutoff\_length$>$} We don't use it
	\item \texttt{$<$seed$>$} Random seed
	\item \texttt{$<$params$>$} Parameters and its values
\end{itemize}

\underline{Exmaple:}

\begin{BGVerbatim}
>> ./QtWrapper_320.txt "" 50.0 214483647 524453158 -p 5 -l 1 -f 1 -P 50 -F 0
\end{BGVerbatim}

\textbf{\large Output:} 

\begin{BGVerbatim}
>> <solved>, <runtime>, <runlength>, <best_sol>, <seed>
\end{BGVerbatim}

\begin{itemize}
	\item {\bf $<$solved$>$} \texttt{SAT} if the algorithm terminates successfully. \texttt{TIMESOUT} if the algorithm times out.
	\item {\bf $<$runtime$>$} Runtime
	\item {\bf $<$runlength$>$} -1 (as Frank Hutter recommended)
	\item {\bf $<$best\_sol$>$} -1 (as Frank Hutter recommended)
	\item {\bf $<$cutoff\_length$>$} We don't use it
	\item {\bf $<$seed$>$} Used random seed
\end{itemize}

\underline{Exmaple:}

\begin{BGVerbatim}
>> SAT, 2.03435, -1, -1, 524453158
\end{BGVerbatim}

To build the wrapper we have followed a simple algorithm: launch two concurrent process. In the parent process we translate the input of the wrapper to the input of {\it Adaptive Search} solver. The solver is executed, and the runtime is measured. After that, we post the output properly. In the child process a {\it sleep} command is executed for \texttt{$<$runtime$>$} seconds and after that, if the parent process has not finished yet, it is killed, posting a time-out message. See Algorithm~\ref{wrapper} for more details.

%\incmargin{1.4em}
\linesnumbered
\begin{algorithm}[H]
	\caption{Costas Wrapper}
	\label{wrapper}
	\dontprintsemicolon
	\SetLine
	\SetKwData{paramConfig}{$\theta$}
	\SetKwData{seed}{s}
	\SetKwData{Inst}{$Pth_{\pi}$}
	\SetKwData{cotime}{$k$}
	\SetKwData{pilsOut}{$PiLS_{out}$}
	\SetKwData{tstart}{$t_0$}
	\SetKwData{tend}{$t_e$}
	\SetKwData{timet}{$t$}
	\SetKwData{strCal}{strCall}
	\SetKwFunction{fork}{fork}
	\SetKwFunction{TIC}{clock\_TIC}
	\SetKwFunction{TOC}{clock\_TOC}
	\SetKwFunction{buildStr}{build\_str}
	\SetKwFunction{call}{systemCall}
	\SetKwFunction{kill}{killProcess}
	\SetKwFunction{output}{paramilsOutput}
	\SetKwFunction{sleep}{sleep}
	\SetKwInOut{Input}{input}
	\SetKwInOut{Output}{output}
	
	\Input{\Inst : problem instance path, \cotime : cut off time, \seed : random seed, \paramConfig : parameters configuration}
	\Output{\pilsOut : Output in a {\sc ParamILS} way}
	\BlankLine
	
	\fork{} \tcc{Divide the execution in two processes}
	\eIf{$<$in child process$>$}{
		\tstart $\leftarrow$ \TIC{}\;
		\strCal $\leftarrow$ \buildStr{\texttt{" ./AS\_Wrapper \%1 -s \%2 \%3"}, \Inst, \seed, \paramConfig}\;
		\call{\strCal}\;
		\tend $\leftarrow$ \TOC{}\;
		\kill{$<$parent process$>$} \label{paso7}\;
		\timet $\leftarrow$ \tend - \tstart\;
		{\bf return} \output{SAT, \timet, \seed}\;
	}{
	\sleep{\cotime}\;
	\kill{$<$child process$>$}\;
	{\bf return} \output{TIMESOUT, \cotime, \seed}\;
}
\end{algorithm}

%\subsection{Using the wrapper}
%
%In this section we explain how to use our wrapper to be able to tune easily instances of {\it All-Interval Series} and \carr{} problems. The {\it All-Interval Series Problem}\footnote{CSPlib:007 (\href{http://www.csplib.org/Problems/prob007/}{\texttt{http://www.csplib.org/Problems/prob007/}})} is the problem of finding a vector $s=\left(s_1,\dots,s_n\right)$, given $n \in \mathbb{N}$, such that $s$ is a permutation of the vector $(0, 1, \dots, n-1)$ and the interval vector $v = \left(\left|S_2-s_1\right|, \left|S_3-s_2\right|, \dots, \left|S_n-S_{n-1}\right|\right)$ (called an all-interval series of size $n$) is a permutation of the vector $(1, 2, \dots, n-1)$. The \carrp{} consists in finding a Costas array, which is an $n\times n$ grid containing $n$ marks such that there is exactly one mark per row and per column and the $n(n-1)/2$ vectors joining each couple of marks are all different (see below for more details about this problems).
%
%\subsubsection{Factory call}
%
%The first step is to implement the class {\sc ICallFactory}. Here, the string-binary-name for the command call is statically obtained. We present, as example, the class {\sc All\_IntervalCallFactory}:
%
%\begin{Verbatim}[fontsize=\normalsize]
%\textcolor{verde}{\bf// all_interval_call_factory.h}
%\textcolor{blue}{\bf class} All_IntervalCallFactory: \textcolor{blue}{\bf public} ICallFactory
%\{
%   \textcolor{blue}{\bf public}:
%      std::string BuildCall();
%      std::string BuildDefaultCall();
%\};
%\end{Verbatim}
%
%\begin{Verbatim}[fontsize=\normalsize]
%\textcolor{verde}{\bf// all_interval_call_factory.cpp}
%\textcolor{dred}{\bf #define} ALGO_EXECUTABLE "./all-interval"
%\textcolor{dred}{\bf #define} DEFAULT_CALL "./all-interval _100.txt"
%
%std::string All_IntervalCallFactory::BuildCall()
%\{
%   \textcolor{blue}{\bf return} ALGO_EXECUTABLE;
%\}
%std::string All_IntervalCallFactory::BuildDefaultCall()
%\{
%   \textcolor{blue}{\bf return} DEFAULT_CALL;
%\}
%\end{Verbatim}
%
%All we have to do is to define our new macro {\bf ALGO\_EXECUTABLE} ({\bf DEFAULT\_CALL} is not being used)
%
%\subsubsection{Main method}
%
%Let's suppose now that we want to run an algorithm called {\it mySolver} that receives a file as parameter, called {\it my\_instance\_size.txt} (this is mandatory). We have to create (as we've explained before) the class {\sc My\_SolverCallFactory} and defining the macro as follows:
%
%\begin{Verbatim}[fontsize=\normalsize]
%\textcolor{dred}{\bf #define} ALGO_EXECUTABLE "./mySolver"
%\end{Verbatim}
%
%Now, the main method would be exactly like this:
%
%\begin{Verbatim}[fontsize=\normalsize]
%\textcolor{blue}{\bf int} main(\textcolor{blue}{\bf int} argc, \textcolor{blue}{\bf char}* argv[])
%\{
%   shared_ptr<ICallFactory> problem = 
%      make_shared<My_SolverCallFactory>();
%   shared_ptr<TuningData> td = 
%      (make_shared<TuningData>(argc, argv, problem));
%
%   shared_ptr<ADWrapper> w (make_shared<ADWrapper>());
%   string output = w->tune(td);
%
%   cout << output << endl;
%   \textcolor{blue}{\bf return} 0;
%\}
%\end{Verbatim}

\section{Results}

In this section we present results of applying {\sc ParamILS} to the resolution of {\it All-Interval Series} and \carr{} problems through {\it Adaptive Search}. In both cases, we need to chose a set of {\it training instances}, to train the tuner, and a set of {\it test instances}, used to obtain the parameter setting. 

\subsection{ Tuning {\it Adaptive Search} for  {\it All-Intervals Series Problem}}

In this section we explain how to use our wrapper to be able to tune easily instances of {\it All-Interval Series Problem} (AISP). The All-Interval Series Problem\footnote{CSPlib:007 (\href{http://www.csplib.org/Problems/prob007/}{\texttt{http://www.csplib.org/Problems/prob007/}})} is the problem of finding a vector $s=\left(s_1,\dots,s_n\right)$, given $n \in \mathbb{N}$, such that $s$ is a permutation of the vector $(0, 1, \dots, n-1)$ and the interval vector $v = \left(\left|S_2-s_1\right|, \left|S_3-s_2\right|, \dots, \left|S_n-S_{n-1}\right|\right)$ (called an all-interval series of size $n$) is a permutation of the vector $(1, 2, \dots, n-1)$. 

\underline{Study cases:}
\begin{enumerate}
	\item The {\it training instances set} is composed by instances of AISP of order $N$ with $$N \in \left\{100, 110, 120, 130, 140, 150, 160, 170, 180\right\}$$
	\item The {\it test instances set} is composed by instances of AISP of order $N$ with $$N \in \left\{190, 200, 210, 220, 230, 240, 250, 260, 265\right\}$$
	\item The time-out for each run is 50.0 seconds
	\item The test quality is based on 100 runs
\end{enumerate}

In a {\bf First Experiment} we use the following {\it parameters domains}:
\begin{itemize}[itemsep=0.2mm]
	\item {\bf P}\texttt{ \{41, 46, 51, 56, 60, 66, 71, 76, 80\}}
	\item {\bf F, f, l}\texttt{ \{0, 1, 2, 3\}}
	\item {\bf p}\texttt{ \{5, 10, 15, 20, 25, 30, 35\}}
\end{itemize}

Table~\ref{table:allint5yh} shows results using a time-out of 5.5 hours (20,000 seconds), and Table~\ref{table:allint1h} shows results using a time-out of 1 hour. In the second case more runs were performed, but in both cases the training qualities are not so different. However, we can conclude that results using 5 hours of time-out are more reliables.

\begin{table}[h] 	
\centering 
\renewcommand{\arraystretch}{1.2}
\resizebox{\columnwidth}{!}{%
\tablePILSresults{
	0 & 66 & 1 & 1 & 25 & 0 & 80 & 2 & 1 & 35 & 0.79666 & 1780 & 8.274 \\
	2 & 56 & 2 & 2 & 20 & 1 & 80 & 1 & 1 & 10 & 0.795 & 1637 & 5.508 \\
	0 & 41 & 0 & 0 & 5 & 1 & 80 & 3 & 0 & 15 & 0.789 & 1547 & 5.8478 \\
	3 & 80 & 3 & 3 & 35 & 1 & 80 & 2 & 0 & 10 & 0.880686 & 1258 & 6.15398\\
}
}
\caption{AISP: \texttt{tunerTimeout} = 20,000 seconds}\label{table:allint5yh}
\end{table}

\begin{table}[h] 	
\centering 
\renewcommand{\arraystretch}{1.2}
\resizebox{\columnwidth}{!}{%
\tablePILSresults{
	0 & 66 & 1 & 1 & 25 & 0 & 80 & 0 & 1 & 25 & 0.815 & 384 & 5.8191 \\
	0 & 66 & 1 & 1 & 25 & 1 & 80 & 1 & 1 & 35 & 0.737 & 452 & 6.267 \\
	0 & 66 & 1 & 1 & 25 & 1 & 56 & 0 & 1 & 35 & 1.03 & 371 & 9.056 \\
	0 & 66 & 1 & 1 & 25 & 0 & 76 & 0 & 1 & 20 & 0.814 & 385 & 4.915 \\
	0 & 66 & 1 & 1 & 25 & 0 & 80 & 3 & 1 & 20 & 0.76 & 469 & 5.417 \\ 
	\hline
	2 & 56 & 2 & 2 & 20 & 0 & 41 & 0 & 1 & 10 & 0.919 & 239 & 18.364 \\
	2 & 56 & 2 & 2 & 20 & 0 & 56 & 1 & 1 & 20 & 0.819 & 407 & 5.409 \\
	2 & 56 & 2 & 2 & 20 & 1 & 80 & 1 & 1 & 35 & 0.772 & 457 & 5.43 \\
	2 & 56 & 2 & 2 & 20 & 1 & 80 & 0 & 1 & 10 & 0.858 & 504 & 5.566 \\
	2 & 56 & 2 & 2 & 20 & 0 & 80 & 1 & 1 & 10 & 0.7845 & 562 & 18.944 \\
	\hline
	0 & 41 & 0 & 0 & 5 & 0 & 41 & 1 & 0 & 10 & 0.9749 & 367 & 5.97813 \\
	0 & 41 & 0 & 0 & 5 & 0 & 41 & 1 & 0 & 10 & 0.885 & 450 & 5.706 \\
	0 & 41 & 0 & 0 & 5 & 0 & 41 & 1 & 0 & 10 & 0.906 & 335 & 18.707 \\
	0 & 41 & 0 & 0 & 5 & 0 & 41 & 1 & 0 & 10 & 0.995 & 335 & 19.558 \\
	0 & 41 & 0 & 0 & 5 & 0 & 41 & 0 & 0 & 5 & 0.855 & 404 & 5.686 \\
	\hline
	3 & 80 & 3 & 3 & 35 & 0 & 66 & 3 & 1 & 25 & 0.9118 & 230 & 26.585 \\
	3 & 80 & 3 & 3 & 35 & 0 & 80 & 1 & 1 & 10 & 0.732 & 310 & 7.875 \\
	3 & 80 & 3 & 3 & 35 & 0 & 80 & 0 & 1 & 20 & 0.816 & 303 & 7.2896 \\
	3 & 80 & 3 & 3 & 35 & 1 & 80 & 3 & 1 & 35 & 0.821 & 327 & 6.812 \\
	3 & 80 & 3 & 3 & 35 & 0 & 80 & 0 & 1 & 30 & 0.9203 & 443 & 5.401 \\
} 
}
\caption{AISP: \texttt{tunerTimeout} = 3,600 seconds}\label{table:allint1h}
\end{table}

In a {\bf Second Experiment} we decide to enlarge a bit more the parameters domains and use a time-out of 5 hours. The {\bf Parameters domains} are the following:
\begin{itemize}[itemsep=0.2mm]
	\item {\bf P}\texttt{ \{10, 20, 30, 40, 50, 60, 70, 80, 90\}}
	\item {\bf F, f, l}\texttt{ \{0, 1, 2, 3, 4, 5, 6, 7, 8\}}
	\item {\bf p}\texttt{ \{10, 20, 30, 40, 50, 60, 70\}}
\end{itemize}

\begin{table}[h] 	
\centering 
\renewcommand{\arraystretch}{1.2}
\resizebox{\columnwidth}{!}{%
\tablePILSresults{
	0 & 10 & 0 & 0 & 10 & 0 & 40 & 7 & 0 & 50 & 0.883188 & 936 & 6.3191 \\
	0 & 10 & 0 & 0 & 10 & 0 & 80 & 2 & 1 & 40 & 0.774659 & 1584 & 5.45674 \\ 
	0 & 10 & 0 & 0 & 10 & 0 & 40 & 2 & 0 & 10 & 0.96885 & 1104 & 6.82643 \\ 
	\hline
	4 & 60 & 4 & 4 & 40 & 0 & 60 & 8 & 1 & 40 & 0.90358 & 1566 & 5.48127 \\
	4 & 50 & 4 & 4 & 40 & 0 & 80 & 5 & 1 & 20 & 0.78536 & 1662 & 11.5649 \\
	3 & 50 & 4 & 2 & 30 & 0 & 90 & 6 & 1 & 70 & 0.79440 & 1395 & 5.08108 \\
	\hline
	0 & 90 & 0 & 0 & 10 & 1 & 90 & 6 & 1 & 10 & 0.859569 & 1379 & 5.4286 \\ 
	0 & 90 & 0 & 0 & 10 & 1 & 90 & 6 & 1 & 30 & 0.80738 & 1117 & 5.47126 \\
	8 & 90 & 8 & 8 & 60 & 0 & 80 & 5 & 1 & 10 & 0.834934 & 1384 & 5.5377 \\
	\hline
	5 & 30 & 2 & 3 & 60 & 0 & 90 & 1 & 0 & 20 & 0.862013 & 1707 & 5.21837 \\
	3 & 20 & 2 & 4 & 60 & 0 & 80 & 6 & 1 & 10 & 0.805604 & 1630 & 5.4467 \\ 
	6 & 70 & 1 & 3 & 50 & 0 & 80 & 5 & 1 & 10 & 0.792600 & 1344 & 5.46558 \\  
	6 & 40 & 1 & 3 & 30 & 1 & 80 & 7 & 0 & 20 & 0.822703 & 1977 & 5.41185 \\
} 
}
\caption{AISP: \texttt{tunerTimeout} = 18,000 seconds}\label{table:allint5h}
\end{table}

Table~\ref{table:allint5h} shows better results in terms of test quality with respect to Table~\ref{table:allint5yh}. For that reason, in the \textbf{FINAL Experiment}, only results obtained in those tables were taken into account (also because they were obtained by using longer times-out). As it can be observed in those tables, Adaptive Search seems to show a good behavior if the parameters {\bf F}, {\bf P} and {\bf l} are in the following sets: \texttt{\bf F} $\in$ \texttt{\{ 0, 1\}}, \texttt{\bf P} $\in$ \texttt{\{ 80, 90\}} and \texttt{\bf l} $\in$ \texttt{\{ 0, 1\}}.

In that sense, a specific parameter setting was extracted from previews results, and 60 runs of {\it Adaptive Search} were performed solving AISP ($N = 600$) benchmark:
\begin{itemize}
	\item[-] 30 using the default parameter setting (\texttt{-F 0 -P 66 -f 1 -l 1 -p 25})
	\item[-] 30 with an optimal parameter setting extracted from the Tables~\ref{table:allint5yh}, \ref{table:allint5h} (\texttt{-F 0 -P 80 -f 6 -l 1 -p 10})
\end{itemize}

Table~\ref{table:testaibad} shows results by using the default parameter settings, and Table~\ref{table:testaigood} shows the results by using the parameter configuration found by {\sc ParamILS}, and it is clear that the default configuration shows better results than {\it ParamILS}'s one, in terms both of runtime mean and standard deviation. Using the default parameter settings, {\it Adaptive Search} can obtains best results int terms of {\it mean} and {\it slowest run}. However, using the {\sc ParamILS} found parameter settings, it reached a {\it fastest} run two times faster than the one using the default parameter settings. 

\begin{table}[h]
\centering
\renewcommand{\arraystretch}{1.2}
\begin{tabular}{|ccccc|}
	\hline
	37.210 & 411.300 & 112.510 & 171.000 & 73.770 \\ 
	327.880 & 214.910 & 124.910 & 482.740 & 530.440 \\  
	\hline 
	212.660 & 99.370 & 287.400 & 533.540 & \textcolor{naranja}{\bf 18.410} \\ 
	197.290 & 1016.950 & 110.230 & 566.480 & \textcolor{intenso}{\bf 1362.010} \\  
	\hline 
	94.860 & 819.700 & 434.460 & 620.600 & 95.920 \\ 
	80.580 & 333.370 & 121.590 & 489.700 & 248.370 \\  
	\hline 
	\multicolumn{5}{|c|}{\bf mean: 341.005333}\\
	\multicolumn{5}{|c|}{\bf spread: 310.444635}\\
	\hline
\end{tabular}
\caption{AISP: Default configuration runtimes (secs)}\label{table:testaibad}
\end{table}
	
\begin{table}[h]
\centering
\renewcommand{\arraystretch}{1.2}
\begin{tabular}{|ccccc|}
	\hline
	154.460 & 264.530 & 169.840 & 26.990 & 108.790 \\ 
	550.210 & 104.900 & 31.100 & \textcolor{naranja}{\bf 9.870} & 1242.900 \\  
	\hline 
	678.760 & 475.570 & 201.200 & 622.410 & 297.960 \\ 
	526.930 & 375.620 & 293.380 & 598.850 & 350.270 \\  
	\hline 
	540.290 & 252.940 & 673.630 & 423.030 & 589.210 \\ 
	32.080 & 254.640 & \textcolor{intenso}{\bf 2034.020} & 571.100 & 207.090 \\  
	\hline 
	\multicolumn{5}{|c|}{\bf mean: 422.085667}\\
	\multicolumn{5}{|c|}{\bf spread: 404.618226}\\
	\hline
\end{tabular}
\caption{AISP: {\sc ParamILS} configuration runtimes (secs)}\label{table:testaigood}
\end{table} 

%--------------------------------------- COSTAS

\subsection{ Tuning {\it Adaptive Search} for \carrp}

The \carrp{} (\CARRP) was already introduced in Chapter~\ref{chap:expe}. It consists in finding a Costas array, which is an $n\times n$ grid containing $n$ marks such that there is exactly one mark per row and per column and the $n(n-1)/2$ vectors joining each couple of marks are all different.

\underline{Study cases:}
\begin{enumerate}
	\item The {\it training instances set} is composed of instances of \CARRP{} of order $N$ with $9 \leq N \leq 15$
	\item The {\it test instances set} is composed of instances of \CARRP{} of order $N$ with $14 \leq N \leq 19$
	\item The cutoff for each run was 60.0 seconds
	\item The test quality is based on 100 runs
\end{enumerate}

The {\bf First Experiments} with this benchmark was using the following parameter domains:
\begin{itemize}[itemsep=0.2mm]
	\item {\bf P}\texttt{ \{10, 20, 30, 40, 50, 60, 70, 80, 90\}}
	\item {\bf F, f, l}\texttt{ \{0, 1, 2, 3, 4, 5, 6, 7, 8\}}
	\item {\bf p}\texttt{ \{5, 10, 20, 30, 40, 50, 60, 70\}}
\end{itemize}

Table~\ref{table:ca1} shows results selecting directly a time-out of 5 hours (18,000 seconds). In this case the training quality of the solutions is better, but do not observe any improvement in the test quality. We can see also how {\it Adaptive Search} seems to be not sensitive to parameters {\bf F} and {\bf p}, \ie they don't change during the tuning process. On the other hand, the tuner seems to find some optimum values for the other parameters: \texttt{\bf P} $\in$ \texttt{\{ 80, 90\}}, \texttt{\bf f} $\in$ \texttt{\{ 4, 5\}} and \texttt{\bf l} $=$ \texttt{2}.

In that case also, an specific parameter setting was extracted from the results showed in Table~\ref{table:ca1}, and 60 runs of Adaptive Search were performed solving \CARRP{} ($N = 20$) benchmark: 
\begin{itemize}
	\item[-] 30 using the default parameter setting (\texttt{-F 0 -P 50 -f 1 -l 0 -p 5})
	\item[-] 30 with an optimal parameter setting extracted from the Table~\ref{table:ca1} (\texttt{-F 3 -P 90 -f 5 -l 2 -p 30}) 
\end{itemize}

\begin{table}[h]
\centering 
\renewcommand{\arraystretch}{1.2}
\resizebox{\columnwidth}{!}{%
\tablePILSresults{
	0 & 10 & 0 & 0 & 5 & 2 & 90 & 2 & 2 & 5 & 0.0493699 & 957 & 5.8461 \\
	0 & 10 & 0 & 0 & 5 & 0 & 90 & 5 & 2 & 5 & 0.0509388 & 1783 & 6.52742 \\ 
	0 & 10 & 0 & 0 & 5 & 0 & 90 & 5 & 2 & 5 & 0.049901 & 1759 & 5.21828 \\ 
	\hline
	3 & 40 & 4 & 4 & 30 & 3 & 90 & 5 & 2 & 30 & 0.053974 & 856 & 6.3539 \\ 
	4 & 50 & 3 & 5 & 20 & 4 & 90 & 5 & 2 & 20 & 0.0500355 & 2000 & 5.4047 \\ 
	4 & 60 & 5 & 3 & 50 & 4 & 60 & 5 & 3 & 50 & 0.0520575 & 2000 & 6.09106 \\
	\hline
	8 & 90 & 8 & 8 & 70 & 8 & 80 & 4 & 2 & 70 & 0.052685 & 550 & 3.85682 \\
	8 & 90 & 8 & 8 & 70 & 8 & 80 & 4 & 2 & 70 & 0.054104 & 536 & 4.17855 \\ 
	8 & 90 & 8 & 8 & 70 & 8 & 80 & 4 & 2 & 70 & 0.0497819 & 1284 & 3.90945 \\ 
	\hline 
	3 & 10 & 1 & 6 & 60 & 3 & 90 & 5 & 2 & 60 & 0.054934 & 2000 & 6.81675 \\ 
	5 & 70 & 6 & 1 & 10 & 5 & 90 & 4 & 2 & 10 & 0.0499895 & 2000 & 4.07365 \\ 
	1 & 30 & 5 & 7 & 5 & 1 & 90 & 4 & 2 & 5 & 0.0525747 & 1237 & 2.70091 \\ 
	7 & 80 & 2 & 0 & 70 & 7 & 90 & 5 & 2 & 70 & 0.0502264 & 212 & 5.2637 \\ 
} 
}
\caption{\CARRP{}: \texttt{tunerTimeout} = 18,000 seconds}\label{table:ca1}
\end{table}

Table~\ref{table:testcabad} shows results by using the default parameter setting, and Table~\ref{table:testcagood} shows results by using the parameter setting found by {\it ParamILS}. One more time, "in the mean", the default setting outperforms {\sc ParamILS}'s.

\begin{table}[h]
\centering
\renewcommand{\arraystretch}{1.2}
\begin{tabular}{|ccccc|}
	\hline
	452.980 & 91.420 & 31.510 & \textcolor{intenso}{\bf 827.860} & 96.670 \\ 
	635.030 & 295.830 & 272.360 & 151.040 & 170.660 \\  
	\hline 
	183.550 & 161.340 & 91.240 & 426.470 & 62.020 \\ 
	138.090 & 236.030 & \textcolor{naranja}{\bf 2.850} & 187.240 & 21.510 \\  
	\hline 
	165.370 & 90.440 & 195.580 & 15.390 & 229.720 \\ 
	170.840 & 174.210 & 30.520 & 6.570 & 115.880 \\  
	\hline
	\multicolumn{5}{|c|}{\bf mean: 191.007}\\
	\multicolumn{5}{|c|}{\bf spread: 185.362}\\
	\hline
\end{tabular}
\caption{\CARRP: Default configuration runtimes (secs)}\label{table:testcabad}
\end{table}

\begin{table}[h]
\centering
\renewcommand{\arraystretch}{1.2}
\begin{tabular}{|ccccc|}
	\hline
	546.260 & 263.230 & 17.200 & 29.220 & 495.940 \\ 
	237.340 & 187.760 & \textcolor{naranja}{\bf 7.810} & 43.120 & 94.370 \\  
	\hline 
	59.930 & 128.690 & 247.810 & 265.010 & 231.260 \\ 
	209.640 & 465.340 & 21.840 & 8.740 & \textcolor{intenso}{\bf 1264.610} \\  
	\hline 
	57.700 & 122.890 & 450.610 & 229.580 & 174.540 \\ 
	414.080 & 402.250 & 91.150 & 677.190 & 58.640 \\  
	\hline 
	\multicolumn{5}{|c|}{\bf mean: 250.125}\\
	\multicolumn{5}{|c|}{\bf spread 263.539}\\
	\hline
\end{tabular}
\caption{\CARRP: ParamILS configuration runtimes (secs)}\label{table:testcagood}
\end{table}    
%
%\subsection{Tuning comparison}
%
%\subsubsection{Experiment 1: Around Default parameters}
%
%{\bf Parameters domains}:
%
%\begin{itemize}[itemsep=0.2mm]
%	\item {\bf P}\texttt{ \{43, 45, 47, 50, 53, 55, 57\}}
%	\item {\bf F, f, l}\texttt{ \{0, 1, 2\}}
%	\item {\bf p}\texttt{ \{5, 7, 10\}}
%\end{itemize}
%
%The results are presented in Table~\ref{table:allint5hdef}.
%
%%\FloatBarrier
%\begin{table}[H] 
%\caption{Results with \texttt{tunerTimeout} = 18000 seconds}
%\centering 
%\renewcommand{\arraystretch}{1.2}
%\tablePILSresults{
%2 & 43 & 0 & 0 & 7 & 0 & 45 & 1 & 0 & 5 & 0.0438025 & 952 & 3.13061 \\ 
%1 & 55 & 2 & 2 & 10 & 1 & 53 & 2 & 0 & 5 & 0.0434366 & 1120 & 6.8108005 \\ 
%1 & 55 & 2 & 2 & 10 & 1 & 53 & 2 & 0 & 5 & 0.0435660 & 2000 & 4.6961601 \\ 
%} 
%\label{table:allint5hdef}
%\end{table}
%
%\subsubsection{ Experiment 2: Around ParamILS parameters}
%
%{\bf Parameters domains}:
%
%\begin{itemize}[itemsep=0.2mm]
%	\item {\bf P}\texttt{ \{75, 77, 80, 83, 85, 87, 90, 93, 95\}}
%	\item {\bf f}\texttt{ \{4, 5, 6\}}
%	\item {\bf F}\texttt{ \{2, 3, 4\}}
%	\item {\bf l}\texttt{ \{1, 2, 3\}}
%	\item {\bf p}\texttt{ \{20, 25, 30, 35, 40\}}
%\end{itemize}
%
%The results are presented in Table~\ref{table:allint5hparamils}.
%
%%\FloatBarrier
%\begin{table}[H] 
%\caption{Results with \texttt{tunerTimeout} = 18000 seconds}
%\centering 
%\renewcommand{\arraystretch}{1.2}
%\tablePILSresults{ 
%2 & 85 & 6 & 1 & 35 & 2 & 85 & 6 & 1 & 35 & 0.0447855 & 2000 & 5.1182902 \\ 
%4 & 75 & 4 & 3 & 25 & 4 & 75 & 4 & 3 & 25 & 0.0458100 & 2000 & 3.4968102 \\ 
%3 & 95 & 5 & 2 & 40 & 3 & 95 & 5 & 2 & 40 & 0.0470930 & 2000 & 4.6591102 \\ 
%} 
%\label{table:allint5hparamils}
%\end{table}

\section{Conclusion}

The conclusion of this study is that the tunning process by hand in this case was more effective than using {\sc ParamILS}. Results show that default parameters used in the current Adaptive Search implementation are more effective and consistent than those found by {\sc ParamILS} for both benchmarks ({\it All-Interval Series} and \carr{} problems).