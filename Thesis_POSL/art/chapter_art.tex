\chapter{State of the art}
\label{chap:art}
\textit{This chapter presents an overview to the state of the art of \COPs{} and different approaches to tackle them. In Section~\ref{sec:combi} the definition of a \COP{} and its links with \CSPs{} (\csp) are introduced, where we concentrate the main efforts, and we give some examples. The basic techniques used to solve these problems are introduced: {\it \cp} (Section~\ref{sec:cp}) and {\it meta--heuristic methods} (Section~\ref{sec:meta}). We also present some advanced techniques like {\it hyper--heuristic methods} in Section~\ref{sec:hyper}, {\it hybridization} in Section~\ref{sec:hybrid}, {\it parallel computing} in Section~\ref{sec:parallel}, and {\it solvers cooperation} in Section~\ref{sec:cooperation}. Finally, before ending the chapter with a brief summary, we present {\it parameter setting techniques} in Section~\ref{sec:tunning}.}
\vfill
\minitoc
\newpage

%This chapter presents an overview to the state of the art of \COPs{} and different approaches to tackle them. In Section~\ref{sec:combi} the definition of a \CSP{} (\csp), emphasizing in the concept of \CSPs, where we concentrate our main efforts. Constraint propagation techniques are deterministic methods to attack these kind of problems (presented in Section~\ref{sec:progagation}), but in some cases they are incapable to solve them (they are mostly used to reduce the problem's search space or to prove it unsatisfiable). For that reason, the model presented in this thesis is based on \textit{meta-heuristic} methods (Section~\ref{sec:meta}). The \textit{Hybridization} approach combines different techniques in the same solution strategy, so the progresses in this field are exposed in Section~\ref{sec:hybrid}.

%The evolution of computer architecture is leading us toward massively multi-core computers for tomorrow, composed of thousands of computing units. A parallel model to solve \csps{} is the core of this work, and its advances, as well as those obtained in the field of \textit{cooperation between solvers}, are presented in Sections~\ref{sec:parallel} and \ref{sec:cooperation} respectively. Finally, this chapter presents in Section~\ref{sec:tunning} an overview of the progresses in the field of \textit{parameter settings}.  

\section{Combinatorial Optimization}\label{sec:combi}
\textit{Combinatorial Optimization} problems come from industrial and real world. We can find them in {\it Resource Allocations} \cite{Akplogan2011}, \textit{Task Scheduling} \cite{Sibbesen2008}, \textit{Master-keying} \cite{Espelage2000}, \textit{Traveling Salesman} and \textit{Knapsack} problems, among others, which are well-known examples of \cops{} \cite{Smith2005}. They are a particular case of \textit{optimization} problems, and their main goal is to find an optimum value (minimal or maximal, depending on the problem) for a discrete function $f$, called \textit{objective function}, involving a set variables $X = \left\{x_1, \dots, x_n\right\}$ defined over a set $D = \left\{D_1, \dots, D_n\right\}$ of discrete domains. These problems generally contain restrictions on the variables called \textit{constraints}, defining the set of forbidden combinations of values for variables in $X$, tacking into account the problem characteristics.

A \textit{configuration} $s\in D_1\times D_2\times\dots\times D_n$ is a combination of values for the variables in $X$. The fact of assigning values $v_i \in D_i$ to all variables in $x_i \in X$ is called \textit{evaluation}. When this evaluation is only performed to a given set of variables in $X$, we called \textit{partial evaluation}. In combinatorial optimization, a \textit{feasible} configuration is a configuration fulfilling all constraints. Finally, a \textit{solution} $s^*$ to the problem is a configuration such that $f(s^*)$ is optimal.

In many practical cases, the main goal is not to find the optimal solution, but finding one feasible configuration. This is the case of \CSPs. Formally, we present the definition of a \csp{}. 

\begin{definition}{\bf (Constraint Satisfaction Problem)}
\label{def:csp}
A \CSP{} (\csp, denoted by $\mathcal{P}$) is a triple $\langle X,D,C \rangle$, where:
\begin{itemize}
\item $X = \left\{x_1,\ldots,x_n\right\}$ is finite a set of variables,
\item $D = \left\{D_1,\ldots, D_n\right\}$ is the set of associated domains. Each domain $D_i$ specifies the set of possible values to the variable $x_i$. %is the set of associated domains to each variable in $X$,
\item $C = \left\{c_1,\ldots, c_m\right\}$ is a set of constraints. Each constraint is defined involving some variables from $X$, and specifies the possible combinations of values for these variables.
\end{itemize}
\end{definition}

In \csps, a solution is a configuration satisfying all constraints $c_i \in C$. %We say that a configuration $s$ satisfies the constraint $c_i \in C$ if and only if $c_i$

\begin{definition}{\bf (Solution of a \csp)}
\label{solCSP}
Given a \csp{} $\mathcal{P}=\langle X,D,C \rangle$ and a configuration $s \in D_1\times D_2\times\dots\times D_n$ we say that it is a solution if and only if:	
\begin{equation*}
c_i\left(s\right)\text{ is true }\forall c_i \in C
\end{equation*}
\end{definition}

Let $Var(c_i)$ be the set of involved variables $\left\{x_1, \dots, x_p\right\}$ in the constraint $c_i$, with $p\leq n$. Then, $c_i\left(s\right)$ denotes the evaluation using the values from the configuration $s$ to the variables $Var(c_i)$. The set of all solutions of $\mathcal{P}$ is denoted by $\mathnormal{Sol}(\mathcal{P})$.

A \csp{} can be considered as a special case of \cops, where the objective function is to reduce to the minimum the number of violated constraints in the model. A solution is then obtained when the number of violated constraints reach the value zero. 

\etal{Galinier} present in \cite{Galinier04} a general approach for solving \csps{}. In this work, authors present the concept of {\it penalty functions} that I pick up in order to write a \csp{} as an \textit{Unrestricted Optimization Problem} (UOP). This formulation was useful in this thesis for modeling the tackled benchmarks. In this formulation, the objective function of this new problem must be such that its set of optimal solutions is equal to the solution set of the original (associated) \csp.

\begin{definition}{\bf (Local penalty function)}
\label{def:local_cost}
Let a {\bf \csp} $\mathcal{P}\langle X,D,C \rangle$ and a configuration $s$ be. We define the operator {\bf local penalty function} as follow: 
\begin{equation*}
\begin{array}{l}
	\omega_i:D\left(X\right)\times 2^{D\left(X\right)}\rightarrow\mathbb{R}^+\text{ where: }\\
	\omega_i\left(s,c_i\right)=\left\{
	\begin{array}{lll}
	0 & \text{ if } & c_i(s)\text{ is true }\\
	k \in \mathbb{R}^+ \setminus {0} & \text{ otherwise } &
	\end{array}
	\right.
\end{array}
\end{equation*}
where $D\left(X\right) = D_1\times D_2 \times\dots\times D_n$
\end{definition}

This penalty function defines the cost of a configuration with respect to a given constraint, so if $\omega_i\left(s,c_i\right)=k$ we say that the configuration $s$ has a local cost $k$ with respect to the constraint $c_i$. In consequence, we define the \textit{global penalty function}, to define the cost of a configuration with respect to all constraint on a \csp:

\begin{definition}{\bf (Global penalty function)}
\label{def:global_cost}
Let a {\bf \csp} $\mathcal{P}\langle X,D,C \rangle$ and a configuration $s$. We define the operator {\bf global penalty function} as follows: 
\begin{equation*}
\begin{array}{l}
\Omega:D\left(X\right)\times 2^{D\left(X\right)}\rightarrow\mathbb{R}^+ \text{ where: }\\
\Omega\left(s,C\right)=\displaystyle\sum_{i=1}^{m}{\omega_i\left(s,c_i\right)}
\end{array}
\end{equation*}
\end{definition}

This global penalty function defines the cost of a configuration with respect to a given set of constraints, so if $\Omega\left(s,C\right)=k$ we say that the configuration $s$ has a cost $k$ with respect to $C$. We can now formulate a \CSP{} as an {\it unrestricted optimization problem}:

\begin{definition}{\bf (CSP's Associated Unrestricted Optimization Problem)}
\label{def:ass_CSP}
Given a {\bf \csp} $\mathcal{P}\langle X,D,C \rangle$ we define its associated Unrestricted Optimization Problem $\mathcal{P}_{opt}\langle X,D,C,f \rangle$ as follows: 
\begin{equation*}
\begin{array}{l}
\displaystyle\min_{X} f\left(X,C\right)\\
\text{Where:  } f\left(X,C\right) \equiv \Omega\left(X,C\right) \text{ is the objective function to be minimized over the variable } X
\end{array}
\end{equation*}
\end{definition}

It is important to note that a given $s$ is optimum if and only if $f\left(s,C\right) = 0$, which means that $s$ satisfies all the constrains in the original \csp{} $\mathcal{P}$. This work focuses in solving the \CSP{} using this formulation.

%An \textit{Optimization Problem} consists in finding the best solution among all possible ones, subject or not, to a set of constraints, depending on whether it is a restricted or an unrestricted problem. The suitable values for the involved variables belong to a set called {\it domain}. When this domain contains only discrete values, we are facing a \COP, and its goal is to find the best possible solution satisfying a global criterion, named {\it objective function}. {\it Resource Allocations} \cite{Akplogan2011}, \textit{Task Scheduling} \cite{Sibbesen2008}, \textit{Master-keying} \cite{Espelage2000}, \textit{Traveling Salesman}, \textit{Knapsack Problem}, among others, are well-known examples of \cops{} \cite{Smith2005}.

\section{Constraint programming}\label{sec:cp}
\csps{} find a lot of "real-world" applications in the industry. In practice, these problems are tackled through different techniques. One of the most popular is \textit{constraint programming}, a combination of three main ingredients: \begin{inparaenum}[i)] \item a declarative model of the problem, \item constraint reasoning techniques like \textit{filtering} and \textit{propagation}, and \item search techniques. \end{inparaenum} This field is a famous research topic developed by the field of artificial intelligence in the middle of the 70's, and a programming paradigm since the end of the 80's.

Modeling a constrained problem, to be solved using \cp{} techniques, means properly choosing variables and their domains, and a right and efficient representation of the constraints set, aiming to declare as explicitly as possible, the solutions space. On the right election of the problem's model depends not only finding the solution, but also doing it in a fast and efficient way.

For modeling \csps{}, two tools can be highlighted. {\sc MiniZinc} is a simple but expressive constraint programming modeling language which is suitable for modeling problems for a range of solvers. It is the most used language for codding \csps{} \cite{Nethercote}. {\sc XCSP} is a readable, concise and structured XML-like language for coding \csps. This format allows to represent constraints defined either extensionally or intensionally. %Is not more used than {\sc MiniZinc} but although 
It was mainly used as the standard in the {\it International Constraint Solver Competition} (ended in 2009), and the {\it ICSC} dataset is for sure the biggest dataset of \csps{} instances existing today. \new{{\sc XCSP3}\footnote{XCSP3 website: \href{www.xcsp.org}{www.xcsp.org}} is the last upgrade of this format and it is available on Internet. Compact and easy to parse, {\sc XCSP3} is able to capture the structure of the problem models, and identifying syntactic and semantic groups of constraints. It introduces a number of features that make it able to enclose practically all constraints that can be found in major constraint solvers \cite{Boussemart2016}}. 

Constraint reasoning techniques are filtering algorithms applied for each constraint to prune provably infeasible values from the domain of the involved variables. This process is called \textit{constraint propagation}, and they are methods used to modify a \CSP{} in order to reduce its variables domains, and turning the problem into one that is equivalent, but \new{with a smaller search space, hence} usually easier to solve \cite{ChristianBessiere2006}. The main goal is to choose one (or some) constraint(s) to enforce certain consistency levels in the constraints. Achieving global consistency is desirable, because only using brute force algorithms one can reach a solution, but this is computationally very costly and intractable. %extremely hard to obtain. 
For that reason, some other consistency levels, easier to achieve, have been defined, like for example, \textit{arc consistency} and \textit{bound consistency}, which means trying to find values in the variables domain which make constraint unsatisfiable, in order to remove them from the domain. The applied procedure to reduce the variable domains is called \textit{reduction function}, and it is applied until a new, "smaller" and easier to solve problem is obtained, and it can not be further reduced: a \textit{fixed point}. Local consistency restrictions on the filtering algorithms are necessary to ensure not loosing solution during the propagation process.

%Before presenting these two consistency levels mentioned before, is imperative to define some notations. 
We said that a variable $x \in c$, if it is involved into the constraint $c$. Let the set $Var(c) = \{x_1\dots x_k\}$ the set of variables  involved into a constraint $c$ be, denoted by $Var(c)$. Then, a constraint $c$ is called \textit{arc consistent} if for all $x_i \in Var(c)$ with $1\leq i\leq k$, and for all $v_j \in D_j$ with $1\leq j\leq \left\|D_j\right\|$:
\[
\exists (v_1, \dots, v_{i-1}, v_{i+1},\dots, v_k) \in D_1\times\dots\times D_{i-1}\times D_{i+1}\times\dots\times D_k
\]
such that $c(v_1, \dots, v_k)$ is fulfilled. In other words, $c$ is arc consistent if for each value of each variable, there exist values for the other variables fulfilling $c$. In that case, we said that each value in the domain of $x_i$ has a \textit{support} in the domain of the other variables.

We denote by $Bnd(D_i) = \left\{\min(D_i), \max(D_i)\right\}$ the bounds of the domain $D_i$. Then, a constraint is \textit{bound consistent} if for all $x_i \in Var(c)$ with $1\leq i\leq k$, and for all $v_j \in Bnd(D_j)$ with $1\leq j\leq 2$:
\[
\exists (v_1, \dots, v_{i-1}, v_{i+1},\dots, v_k) \in Bnd\left(D_1\right)\times\dots\times Bnd\left(D_{i-1}\right)\times Bnd\left(D_{i+1}\right)\times\dots\times Bnd\left(D_k\right)
\]
such that $c(v_1, \dots, v_k)$ is fulfilled. It means that each bound (min/max) in the domain of $x_i$ has a support in the bounds (min/max) of the other variables. As we can notice that arc consistency is a stronger property, but heavier to enforce.

Apt and Monfroy have proposed in \cite{Apt} and \cite{Monfroy}, respectively, a formalization of constraint propagation through \textit{chaotic iterations}, which is a technique that comes from numerical analysis to compute limits of iterations of finite sets of functions, and adapted for computer science needs for naturally explain constraint propagation \cite{Chazan1969, Cousot1977}. Another approach is presented by Monfroy in \cite{Monfroy2000}, a coordination-based chaotic iteration algorithm for constraint propagation, which is a scalable, flexible and generic framework for constraint propagation using coordination languages, not requiring special modeling of \csps. Zoeteweij provides an implementation of this algorithm in {\sc DICE} (Distributed Constraint Environment) \cite{Zoeteweij2003} using the {\sc Manifold} coordination language. Coordination services implement existing protocols for constraint propagation, termination detection and splitting of \csps. {\sc DICE} combines these protocols with support for parallel search and the grouping of closely related components into cooperating solvers.

Another implementation of constraint propagation is proposed by \etal{Granvilliers} in \cite{Granvilliers2001}, using composition of reductions. It is a general algorithmic approach to tackle strategies that can be dynamically tuned with respect to the current state of constraint propagation, using composition operators. A composition operator models a sub--sequence of an iteration, in which the ordering of application of reduction functions is described by means of combinators for sequential, parallel or fixed--point computation, integrating smoothly the strategies to the model. This general framework provides a good level of abstraction for designing an object-oriented architecture of constraint propagation. Composition can be handled by the {\it Composite Design Pattern} \cite{DP_Composite}, supporting inheritance between elementary and compound reduction functions. The propagation mechanism uses the {\it Observer (Listener) Design Pattern} \cite{DP_Observer}, that makes the connection between domain modifications and re--invocation of reduction functions (event-based relations between objects); and the generic algorithm has been implemented using the {\it Strategy Design Pattern} \cite{DP_Strategy}, that allows to parametrize parts of algorithms.

A propagation engine prototype with a \textit{Domain Specific Language} (DSL) was implemented by \etal{Prud'homme} in \cite{Prudhomme2013}. It is a solver--independent language able to configure constraint propagations at the modeling stage. The main contributions are a DSL to ease configure constraint propagation engines, and the exploitation of the basic properties of DSL in order to ensure both completeness and correctness of the produced propagation engine. %, like:	
%\begin{inparaenum}[i)]%\begin{itemize}
%	\item {\it Solver independent description}: The DSL does not rely on specific solver requirements (but assuming that solvers provide full access to variable and propagator properties), 
%	\item {\it Expressivity}: The DSL covers commonly used data structures and characteristics, 
%	\item {\it Extensibility}: New attributes can be introduced to make group definition more concise. New collections and iterators can provide new propagation schemes, 
%	\item {\it Unique propagation}: The top-bottom left-right evaluation of the DSL ensures that each arc is only represented once in the propagation engine.
%\end{inparaenum}%\end{itemize}
Some characteristics are required to fully benefit from the DSL. Due to their positive impact on efficiency, modern constraint solvers already implement these techniques:
\begin{inparaenum}[i)] %\begin{itemize}
	\item Propagators are discriminated thanks to their priority (deciding which propagator to run next): lighter propagators (in the complexity sense) are executed before heavier ones.
	\item A controller propagator is attached to each group of propagators.
	\item Open access to variable and propagator properties: for instance, variable cardinality, propagator arity or propagator priority.
\end{inparaenum}%\end{itemize}
To be more flexible and more accurate, they assume that all arcs from the current \textit{CSP}, are explicitly accessible. This is achieved by explicitly representing all of them and associating them with {\it watched literals} \cite{Gent2006} (controlling the behavior of variable--value pairs to trigger propagation) or {\it advisors} \cite{Lagerkvist2007} (a method for supporting incremental propagation in propagator--centered setting). %{\it Advisors} in \cite{Lagerkvist2007} are used to modify propagator state and to decide whether a propagator must be propagated or "scheduled". 

Most of the times, we can not solve \csps{} only applying constraint propagation techniques. It is necessary to combine them with search algorithms. The complete search process consists in testing all possible configurations in an ordered way. Each time a partial evaluation is executed (evaluating just a set of variables), new constraints are posted, meaning that the propagation process can be relaunched. The simplest approach is using a backtracking search. It can be seen as performing a depth--first traversal of a search tree. This search tree is generated as the search progresses and represents alternative choices that may have to be examined in order to find a solution. Constraints are used to check whether a node may possibly lead to a solution of the \csp{} and to prune subtrees containing no solutions. A node in the search
tree is a \textit{dead-end} if it does not lead to a solution. Differences between searches lies in the selection criteria of the order of variables to be evaluated, and the order of the values to be assigned to variables. A \textit{static} search strategy is based on selecting the variable with me minimum index, to be evaluated first with the minimum value of its domain. Using this search, the \textit{tree structure} of the search space does not change, but is good for testing propagators. A \textit{dynamic} search strategy is based on selecting the variable with me minimum domain element, to be evaluated first with the minimum value of its domain. In this search strategy, propagators affect the variable selection order. A classical search strategy is based on selecting the variable with me minimum domain size, to be evaluated first with any values of its domain. Based on the \textit{first fail} principle which tells "\textit{Focus first on the variable that is more likely to cause a fail}",  this strategy works pretty well in many cases because by branching early on variables with a few value, the search tree becomes smaller.

In the field of \cp{} we can find a lot of solvers, able to solve constrained problems using these techniques. As examples, we can cite {\sc Cplex}\footnote{CPLEX Optimizer, available at: \href{http://www.ilog.com/products/cplex/}{http://www.ilog.com/products/cplex/}}, \textit{OR-tools}\footnote{Google Optimization Tools, available at: \href{https://developers.google.com/optimization/}{https://developers.google.com/optimization/}}, {\sc Gecode} and \choco. {\sc Cplex} is an analytical decision support toolkit for rapid development and deployment of optimization models using mathematical and constraint programming, to solve very large, real-world optimization problems. {\sc Gecode} is an efficient open source environment for developing constraint-based system and applications, that provides a modular and extensible constraint solver \cite{Gecode}, written in C++ (winner of all gold medals in the \textit{MiniZinc Challenge} from 2008 to 2012). During the formation phase of this PhD, we had the opportunity to perform some pedagogical experiments using two other important and recognized solvers: \textit{OR-tools} and \choco. The \textit{OR-tools} is an open source, portable and documented software suite for combinatorial optimization. It contains an efficient  constraint programming solver, used internally at Google, where speed and memory consumption are critical.

\choco{} is a free and open-source tool written in java, to describe hard combinatorial problems in the form of \csps{} and solving them using \CP{} techniques. Mainly developed by people at \'Ecole des Mines de Nantes (France), is a solver with a nice history, wining some awards, including seven medals in four entries in the \textit{MiniZinc Challenge}. This solver uses multi-thread approach for the resolution, and provides a problem modeler able to manipulate a wide variety of variable types. This problem modeler accepts over 70 constraints, including all classical arithmetical constraints, the possibility of using boolean operations between constraints, table constraints, i.e. defining the sets of tuples that verify the intended relation for a set of variables and a large set of useful classical global constraints including the \textit{alldifferent} constraint, the global \textit{cardinality} constraint, the \textit{cumulative} constraint, among others. \choco{} also contains a {\sc MiniZinc} and \textit{XCSP} instance parser. 
\choco{} can either deal with satisfaction or optimization problems. The search can be parameterized using a set of predefined variable and value selection heuristics, and also the variable and/or value selectors can be parametrized \cite{Jussien2008, Prudhomme2016}.

Although \cp{} techniques have shown very good results solving constrained problems, the search space in practical instances becomes intractable for them. For that reason, these constrained problems are mostly tackled by {\it meta-heuristic methods} or hybrid approaches. %, like \textit{Monte Carlo Tree Search} methods, which combine precision (tree search) with randomness (meta-heuristic) showing good results in artificial intelligence for games \cite{Chaslot2008, Browne2012}.

\section{Meta-heuristic methods}\label{sec:meta}
{\it Meta-heuristic} methods are algorithms generally applied to solve problems without deprived of satisfactory problem-specific algorithms to solve them. They are general purpose techniques widely used to solve complex optimization problems in industry and services, in areas ranging from finance to production management and engineering, with relatively few modifications: \begin{inparaenum}[i)] \item they are nature-inspired (based on some principles from physics or biology), \item they involve random variables as an stochastic component, therefore approximate and usually non-deterministic, \item and they have several parameters that need to be fitted \cite{Dreo2006}.\end{inparaenum} 

A Meta-heuristic Method is formally defined as an iterative process which guides a subordinate heuristic by combining different concepts for \textit{exploration} (also called \textit{diversification}) i.e. guiding the search process through a much larger portion of the search space, and \textit{exploitation} (also called \textit{intensification}) i.e. guiding the search process into a limited, but promising, region of the search space \cite{Osman1996}.

In contrast with tree-search based methods, which are subject to combinatorial explosion (required time to find solutions of NP-hard problems increases exponentially w.r.t. the problem size), they do dot perform an ordered and complete search. For that reason they are not able to provide a proof that the optimal solution will be found in a finite (although often prohibitively large) amount of time. Meta-heuristics are therefore developed specifically to find a "\textit{acceptably good}" solution "\textit{acceptably}" fast. In the case of \csps, finding a feasible solution is enough, for that reason, these methods have been proven to be effective solving these kind of problems.

Sometimes meta-heuristics use domain-specific knowledge in the form of heuristics controlled by an upper level strategy. Nowadays more advanced meta-heuristics use search experience to guide the search \cite{Blum2003}.

Meta-heuristics are divided into two groups: % \cite{Boussaid2013}: 
\begin{enumerate}%\begin{inparaenum}[i)]
    \item {\it Single Solution Based:} more exploitation oriented, intensifying the search in some specific areas. This work focuses its attention on this first group.
    \item {\it Population Based:} more exploration oriented, identifying areas of the search space where there are (or where there could be) the best solutions. % \cite{Maturana2012, Reeves2010, Dorigo2010}.
\end{enumerate} %\end{inparaenum}

\subsection{Single Solution Based Meta-heuristic}

% you will focus on the first group => this is your thessi topic
Methods of the first group are also called {\it trajectory methods}. They usually start from a candidate configuration $s$ (usually random) inside the search space, and then iteratively make local moves consisting of applying some local modifications to $s$ to create a set of configuration called \textit{neighborhood} $\mathcal{V}\left(s\right)$, and selecting a new configuration $s'\in \mathcal{V}\left(s\right)$, following some criteria, to be the new candidate solution for the next iteration. This process is repeated until a solution for the problem is found. These methods can be seen as an extension of \textit{local search methods} \cite{Boussaid2013}. Local search methods are the most widely used approaches to solve \COPs{} because they often produces high--quality solutions in reasonable time \cite{Voss2012}.
 
{\it Simulated Annealing} (SA) \cite{Nikolaev2010} is one of the first algorithms with an explicit strategy to escape from local minima. It is a method inspired by the annealing technique used by metallurgists to obtain a "well ordered" solid state of minimal energy. Its main feature is to allow moves resulting in solutions of worse quality than the current solution under certain probability, in order to scape from local minima, which is decreased during the search process \cite{Blum2003}. As an example of an implementation of this algorithm obtaining good results, it can be cited a work presented by \etal{Anagnostopoulos} in \cite{Anagnostopoulos2006} which is an adaptation of a SA algorithm (TTSA) for the Traveling Tournament Problem (TPP) that explores both feasible and infeasible schedules that includes advanced techniques such as strategic oscillation to balance the time spent in the feasible and infeasible regions by varying the penalty for violations; and reheats (increasing the temperature again) to balance the exploration of the feasible and infeasible regions and to escape local minima.

{\it Tabu Search} (TS) \cite{Gendreau2010}, is a very classic meta-heuristic for \COPs. It explicitly maintains a history of the search, as a short term memory keeping track of the most recently visited solutions, to scape from local minima, to avoid cycles, and to deeply explore the search space. A TB meta-heuristic guides the search on the approach presented in \cite{IvanDotu2007} by Iván Dotú and Pascal Van Hentenryck to solve instances of the \textit{Social Golfers} problem, showing that local search is a very effective way to solve this problem. The used approach does not take symmetries into account, leading to an algorithm which is significant simpler than constraint programming solutions. 

{\it Guided Local Search} (GLS) \cite{Christos2010} consists of dynamically changing the objective function to change the search landscape, helping the search escape from local minima. The set of solutions and the neighborhood are fixed, while the objective function is dynamically changed with the aim of making the current local optimum less attractive \cite{Blum2003}. \etal{Mills} propose in \cite{Mills2000} an implementation of a GLS, which is used to solve the satisfiability (SAT) problem, a special case of a \csp{} where variables take booleans values and constraints are disjunctions of literals (i.e. variables or theirs negations).

The \textit{Variable Neighborhood Search} (VNS) is another meta-heuristic that systematically changes the neighborhood size during the search process. This neighborhood can be arbitrarily chosen, but often a sequence $\left|\mathcal{N}_1\right|<\left|\mathcal{N}_2\right|< \dots<\left|\mathcal{N}_{k_{max}}\right|$ of neighborhoods with increasing cardinality is defined. The choice of neighborhoods of increasing cardinality yields a progressive diversification of the search \cite{PierreNenad,Blum2003}. \etal{Bouhmala} introduce in \cite{Bouhmala2015} a \textit{generalized Variable Neighborhood Search} for \COPs, where the order in which the neighborhood structures are selected during the search process offers a more effective mechanism for diversification and intensification. %and in \cite{Burke2010} is presented a model combining integer programming and VNS for \textit{Constrained Nurse Rostering} problems.

{\it Greedy Randomized Adaptive Search Procedures} (GRASP) is an iterative randomized sampling technique in which each iteration provides a solution to the target problem at hand through two phases (constructive and search). The first one constructs an initial solution via an adaptive randomized greedy function. This function construct a solution performing partial evaluations using values of a restricted candidate list (RCL) formed by the best values, incorporating to the current partial solution values resulting in the smallest incremental costs (the greedy aspect of the algorithm). The value to be incorporated into the partial solution is randomly selected from those in the RCL (the random aspect of the algorithm). Then the candidate list is updated and the incremental costs are reevaluated (the adaptive aspect of the algorithm). The second phase applies a local search procedure to the constructed solution in to find an improvement \cite{Feo95}. GRASP does not make any smart use of the history of the search process. It only stores the problem instance and the best found solution. That is why GRASP is often outperformed by other meta-heuristics \cite{Blum2003}. However, \etal{Resende} introduce in \cite{Resende2009} some extensions like alternative solution construction mechanisms and techniques to speed up the search are presented.

Galinier et al. present in \cite{Galinier04} a general approach for solving constraint based problems by local search. In this work, authors present the concept of {\it penalty functions} that we pick up in order to write a \csp{} as an \textit{Unrestricted Optimization Problem} (UOP). This formulation was useful in this thesis for modeling the tackled benchmarks. In this formulation, the \textit{objective function} of this new problem must be such that its set of optimal solutions is equal to the solution set of the original (associated) \csp.

\begin{definition}{\bf (Local penalty function)}
\label{def:local_cost}
Let a {\bf \csp} $\mathcal{P}\langle X,D,C \rangle$ and a configuration $s$ be. We define the operator {\bf local penalty function} as follow: 
\begin{equation*}
\begin{array}{l}
	\omega_i:D\left(X\right)\times 2^{D\left(X\right)}\rightarrow\mathbb{R}^+\text{ where: }\\
	\omega_i\left(s,c_i\right)=\left\{
	\begin{array}{lll}
	0 & \text{ if } & c_i(s)\text{ is true }\\
	k \in \mathbb{R}^+ \setminus {0} & \text{ otherwise } &
	\end{array}
	\right.
\end{array}
\end{equation*}
\end{definition}

This penalty function defines the cost of a configuration with respect to a given constraint, so if $\omega_i\left(s,c_i\right)=k$ we say that the configuration $s$ has a local cost $k$ with respect to the constraint $c_i$. In consequence, we define the \textit{global penalty function}, to define the cost of a configuration with respect to all constraint on a \csp:

\begin{definition}{\bf (Global penalty function)}
\label{def:global_cost}
Let a {\bf \csp} $\mathcal{P}\langle X,D,C \rangle$ and a configuration $s$. We define the operator {\bf global penalty function} as follows: 
\begin{equation*}
\begin{array}{l}
\Omega:D\left(X\right)\times 2^{D\left(X\right)}\rightarrow\mathbb{R}^+ \text{ where: }\\
\Omega\left(s,C\right)=\displaystyle\sum_{i=1}^{m}{\omega_i\left(s,c_i\right)}
\end{array}
\end{equation*}
\end{definition}

This global penalty function defines the cost of a configuration with respect to a given set of constraints, so if $\Omega\left(s,C\right)=k$ we say that the configuration $s$ has a cost $k$ with respect to $C$. We can now formulate a \CSP{} as an {\it unrestricted optimization problem}:

\begin{definition}{\bf (CSP's Associated Unrestricted Optimization Problem)}
\label{def:ass_CSP}
Given a {\bf \csp} $\mathcal{P}\langle X,D,C \rangle$ we define its {\bf associated Unrestricted Optimization Problem} $\mathcal{P}_{opt}\langle X,D,C,f \rangle$ as follows: 
\begin{equation*}
\begin{array}{l}
\displaystyle\min_{X} f\left(X,C\right)\\
\text{Where:  } f\left(X,C\right) \equiv \Omega\left(X,C\right) \text{ is the objective function to be minimized over the variable } X
\end{array}
\end{equation*}
\end{definition}

It is important to note that a given $s$ is optimum if and only if $f\left(s,C\right) = 0$, which means that $s$ satisfies all the constrains in the original \csp{} $\mathcal{P}$.

Many other implementations of local search algorithms have been presented with good results. {\it Adaptive Search} is an algorithm based local search method, taking advantage of the structure of the problem in terms of constraints and variables. It uses also the concept of \textit{penalty function}, based on this information, seeking to reduce the \textit{error} (a projected cost of a variable, as a measure of how responsible is the variable in the cost of a configuration) on the worse variable so far. It computes the penalty function of each constraint, then combines for each variable the \textit{errors} of all constraints in which it appears. This allows to chose the variable with the maximal \textit{error} will be chosen as a "culprit" and thus its value will be modified for the next iteration with the best value, that is, the value for which the total error in the next configuration is minimal \cite{Diaz, Codognet2001, Caniou14}. In \cite{Munera2015} \etal{Munera} based their solution method in Adaptive Search to solve the \textit{Stable Marriage with Incomplete List and Ties} problem \cite{Iwama1999}, a natural variant of the \textit{Stable Marriage Problem} \cite{Gale1962}, using a cooperative parallel approach. Michel and Van Hentenryck propose in \cite{Michel2002} a constraint-based, object-oriented architecture to significantly reduce the development time of local search algorithms. This architecture consists of two main components: a declarative component which models the application in terms of constraints and functions, and a search component which specifies the meta-heuristic, illustrated using {\sc Comet}, an optimization platform that provides a Java-like programming language to work with constraint and objective functions \cite{Comet, Michel2005}, supporting the local search architecture. It also provides abstraction features to make a clean separation between the model an the search (promoting the reusing of the later) and novel control structures to implement nondeterminism.

\subsection{Population Based Meta-heuristic}

In the second group of meta-heuristic algorithms, we can find the methods based on populations. These methods do not work with a single configuration, but with a set of configurations named {\it population}. They were not part of the main investigation of this thesis, so I will nos get into details, but I thinks it is fair to mention some of the most important methods.

The most popular algorithms in this group are population-based methods. They are related to \begin{inparaenum}[i)] \item \textit{Evolutionary Computation} (EC), inspired by the "Darwin's principle", where only the best adapted individuals will survive, where a population of individuals is modified through recombination and mutation operators, and \item \textit{Swarm Intelligence} (SI), where the idea is to produce computational intelligence by exploiting behaviors of social interaction \cite{Boussaid2013}.\end{inparaenum} 

Algorithms based on evolutionary computation have a general structure. Every iteration of the algorithm corresponds to a \textit{generation}, where a population of candidate solutions (called individuals) to a given problem, is capable of reproducing and is subject to genetic variations and environmental pressure that causes natural selection. New solutions are created by applying recombination, by combining two or more selected individuals (parents) to produce one or more new individuals (the offspring). Mutation can be applied allowing the appearance of new traits in the offspring to promote diversity. The fitness (how good the solutions are) of the resulting solutions is evaluated and a suitable selection strategy is then applied to determine which solutions will be maintained into the next generation. As a termination condition, a predefined number of generations (or function evaluations) of simulated evolutionary process is usually used. The evolutionary algorithm's operators are another branch of study, because they have to be selected properly according to the specific problem, due to they will play an important roll in the algorithm behavior \cite{Maturana2012}.

Probably the most popular evolutionary algorithms are {\it Genetic Algorithms} (GA) \cite{Reeves2010}, where operators are based on the simulation of the genetic variation  process to achieve individuals (solutions in this case) more adapted. GAs are usually differently implemented according to the problem: representation of solution (chromosomes), selection strategy, type of crossover (the recombination operator) and mutation operators, etc. The most common representation of the chromosomes is a fixed-length binary string, because simple bit manipulation operations allow the easy implementation of crossover and mutation operations.

Swarm intelligence based methods are inspired by the collective behavior in society of groups different form of live. SI systems are typically made up of a population of simple element, capable of performing certain operations, interacting locally with one another and with their environment. These elements have very limited individual capability, but in cooperation with others can perform many complex tasks necessary for their survival. Ant colony optimization, Particle Swarm Optimization and Bee Colony Optimization are examples to this approach.

{\it Ant Colony optimization} algorithms are inspired by the behavior of real ants. Ants searching for food, initially explore the area surrounding the nest by performing a randomized walk. Along the path between food source and nest, ants deposit a pheromone trail on the ground in order to mark some promising path that should guide other ants to the food source. After some time, the shortest path between the nest and the food source has a higher concentration of pheromone, so it attracts more ants \cite{Dorigo2010}.

\textit{Particle Swarm optimization} uses the metaphor of the flocking behavior of birds to solve optimization problems. Each element of the swarm is a candidate solution to the problem, stochastically generated in the search space, and they are connected to some others elements called \textit{neighbors}. It is represented by a velocity, a location in the search space and has a memory which helps it to remember its previous best position. This values describe the \textit{influence} of each element over its neighbors \cite{Poli2007}.

\textit{Bee Colony optimization} consists of three groups of bees: employed bees, onlookers and scout bees. A food source is a possible solution to the problem. Employed bees are currently exploiting a food source. They exploit the food source, carry the information about food source back to the hive and share it with onlooker bees. Onlookers bees wait in the hive for the information to be shared with the employed bees to update their knowledge about discovered food sources. Scouts bees are always searching for new food sources near the hive. Employed bees share information about the nectar amount of a food source by dancing in the designated dance area inside the hive. This information represents the quality of the solution. The nature of dance is proportional to the nectar content of food source. Onlooker bees watch the dance and choose a food source according to the probability proportional to the quality of that food source. In that sense, good food sources attract more onlooker bees. Whenever a food source is fully exploited, all employed bees associated with it abandon the food source, and become scouts \cite{Gao2012}.

\section{Hyper-heuristic Methods}\label{sec:hyper}
\textit{Hyper-heuristics} are automated methodologies for selecting or generating meta-heuristics algorithms to solve hard computational problems \cite{Chakhlevitch2008}. This can be achieved with a learning mechanism that evaluates the quality of the algorithm solutions, in order to become general enough to solve new instances of a given problem. \textit{Hyper-heuristics} are related with the \textit{Algorithm Selection Problem}, so they establish a close relationship between a problem instance, the algorithm to solve it and its performance \cite{Ryser-welch}. Hyper-heuristic frameworks are also known as \textit{Algorithm-Portfolio}--based frameworks. Their goal is predicting the running time of algorithms using statistical regression. Then the fastest predicted algorithm is used to solved the problem until a suitable solution is found or a time-out is reached~\cite{Leyton-Brown2003}.

This approach have been followed for solving constrained problems. {\sc Hyperion}$^2$ \cite{Brownlee2014} is a Java framework for meta-- and hyper-- heuristics which allows the analysis of the trace taken by an algorithm and its constituent components through the search space. It promotes interoperability via component interfaces, allowing rapid prototyping of meta- and hyper-heuristics, with the potential of using the same source code in either case. It also provides generic templates for a variety of local search and evolutionary computation algorithms, making easier the construction of novel meta- and hyper-heuristics by hybridization (via interface interoperability) or extension (subtype polymorphism). {\sc Hyperion}$^2$ is faithful to "{\it only pay for what you use}", a design philosophy that attempts to ensure that generality doesn't necessarily imply inefficiency. \textit{hMod} is inspired by the previous frameworks, but using a new object-oriented architecture. It encodes the core of the hyper-heuristic in several modules, referred as algorithm containers. \textit{hMod} directs the programmer to define the heuristic using two separate XML files; one for the heuristic selection process and the other one for the acceptance criteria~\cite{Urra2013}.

\textit{Evolving evolutionary algorithms} are specialized hyper-heuristic methods which attempt to readjust an evolutionary algorithm to the problem needs. An evolutionary algorithm (EA) discover the rules and knowledge to find the best algorithm to solve a problem. In \cite{Diosan2009} \etal{Dio\c{s}an} use linear genetic programming and multi-expression genetic programming to optimize the EA solving unimodal mathematical functions and another EA to adjust the sequence of genetic and reproductive operators. A solution consists of a new evolutionary algorithm capable of outperforming genetic algorithms when solving a specific class of unimodal test functions. An different but interesting point of view is presented in \cite{Samulowitz2013}, where \etal{Samulowitz} present \textit{Snappy}, a \textit{Simple Neighborhood-based Algorithm Portfolio} written in \textit{Python}. It is a very resent framework that aims to provide a tool able to improve its own performances through on-line learning. Instead of using the traditional off-line training step, a neighborhood search predicts the performance of the algorithms. It incorporates available knowledge coming from portfolio's runs, by considering the following ways incrementally: \begin{inparaenum}[1-] \item Every time a test instance is considered, it is added to the current set of training instances. \item After selecting an algorithm for a given test instance, the actual runtime information for the selected algorithm on this instance is added to the data set. It means that the difference between neighborhoods of different algorithms represent how often algorithms will be selected. \end{inparaenum} Other interesting idea is proposed by \etal{Swan} in {\sc Templar}, a framework to generate algorithms changing predefined components using hyper-heuristics methods~\cite{Swan2015}.

\section{Hybridization}\label{sec:hybrid}
The \textit{Hybridization} approach is the one which combines different approaches into the same solution strategy, and recently, it leads to very good results in the constraint satisfaction field. We can find hybridization in combining algorithms in the same branch of investigation. This is the case of {\it ParadisEO}, a framework to design parallel and distributed hybrid meta-heuristics showing very good results, including a broad range of reusable features to easily design evolutionary algorithms and local search methods \cite{Cahon2004}. But we can find hybridization also in combining very different techniques, like the work of El-Ghazali Talbi presented in \cite{El-Ghazali2013}, which is a taxonomy of hybrid optimization algorithms is presented in an attempt to provide a mechanism to allow qualitative comparison of hybrid optimization algorithms, combining meta-heuristics with other optimization algorithms from mathematical programming, machine learning and constraint programming.

However, maybe one of the most common hybridization in this field, is the combination of meta-heuristic methods and constraint programming techniques. Constraint programming algorithms are based on backtracking mechanisms. These algorithms, also called {\it complete method} usually explore the search space systematically, and thus guarantee to find a solution if one exists. Meta-heuristic methods may find a solution to a problem, but they can fail even if the problem is satisfiable, because of its local nature. They perform a probabilistic exploration of the search space, so they are not able to guarantee finding a solution. For that reason they are also know as {\it incomplete methods}. However, they are more efficient (with respect to response time) than complete methods. The challenge is trying to get the best of both of these methods: exploration of a neighborhood from meta-heuristics, and the power of propagation from constraint programming \cite{Jussien2002,Pesant1996,Shaw1998}.

Hooker J.N. presents in \cite{Hooker2012} some ideas to illustrate the common structure present in exact and heuristic methods, to encourage the exchange of algorithmic techniques between them. The goal of this approach is to design solution methods able to smoothly transform their strategies from exhaustive to non-exhaustive search as the problem becomes more complex. Following this direction, \etal{Monfroy} present in \cite{Monfroya,Monfroyb} a general hybridization framework, proposed to combine complete constraints resolution techniques with meta-heuristic optimization methods in order to reduce the problem through domain reduction functions, ensuring not loosing solutions. 

A popular way of hybridization is the \textit{portfolio approach}, which is a methodology exploiting the significant variety in performances of different algorithms and combining them in order to create a globally better solver. In \cite{Amadini}, \etal{Amadini} propose \texttt{xcsp2mzn}, a tool for converting problem instances from the XCSP format %\cite{Committee} 
to {\sc MiniZinc}; %that is a simple but expressive constraint programming modeling language which is suitable for modeling problems for a range of solvers. It is the most used language for codding \csps{} \cite{Nethercote}. 
%The second contribution of this work is the development of
and \texttt{mzn2feat}, a tool to extract static and dynamic features from the {\sc MiniZinc} representation, with the help of the {\sc Gecode} interpreter, allowing a better and more accurate selection of the solvers to use according to the instances to solve. %Some results are showed proposing that the performances that can be obtained using these features are competitive with state of the art on \csp{} portfolio techniques. 
Based also on the portfolio approach, \etal{Amadini} propose in \cite{Amadini2014} a \textit{time splitting} technique to solve optimization problems. Given a problem $P$ and a schedule $Sch = \left[(\Sigma_1, t_1),\dots,(\Sigma_n, t_n)\right]$ of $n$ solvers, the corresponding time-split solver is defined as a particular solver such that:  
\begin{inparaenum} %\begin{enumerate}[label=\alph*)]
\item runs solver $\Sigma_1$ on $P$ for a period of time $t_1$, 
\item then, for $i = 1,\dots, n-1$, runs solver $\Sigma_{i+1}$ on $P$ for a period of time $t_{i+1}$ exploiting or not the best solution found by  the previous solver $\Sigma_i$ during $t_i$ units of time.
\end{inparaenum}%\end{enumerate}
{\it Autonomous search} is a technique based on supervised or controlled learning. This system are another portfolio point of view presented by \etal{Hamadi} in \cite{WhatIsAuto}, which improves its performance while it solves problems, either modifying its internal components to take advantage of the opportunities in the search space, or choosing adequately the solver to use.

An interesting hybridization point of view, is the integration of operations research into constraint programming. \etal{Fontaine} use in \cite{Fontaine2014} a generalization of the optimization paradigm \textit{Lagrangian relaxation}, to relax the hard constraints into the objective function, and applying them into constraint-programming and local search models. It combines concepts of constraint violation (typically used in constraint programming and local search) and constraint satisfiability (typically used in mathematical programming). Hooker J.N. presents in \cite{Hooker2006} a detailed description of how operation research models like mixed integer linear programming (MILP) models (which can themselves be relaxed), Lagrangian relaxations, and dynamic programming models can be applied to constraint programming. 

% COMENTAR 
%Nowadays there exists some tools to face this kind of problems. We can cite {\sc Choco}, an open source java constraint programming library \cite{Jussien2008}; ; ; {\sc Adaptive Search}, a constraint-based local search methods \cite{Diaz}; among others.

%COMENTAR   
%There exist also tools for modeling \textit{CSP} problems. Some of them intent to be a standards in terms of problem modeling.  Codding the problems using one of these tools (or both), it gives us the advantage of solving them using many solvers that support those languages. Furthermore, developing our own solver, it is also interesting to use them because we can test and compare our results using a wide range of available problems. 

%\nocite{Choco, Comet, CometPascal, Gecode, XCSP, Features, Minizinc, X10}

\section{Parallel computing}\label{sec:parallel}
Despite advances previously presented, hard instances of many problems are still complicated to solve through these techniques. Thanks to \textit{parallel computing}, we have been capable of going one step further in solving \csps. Parallel computing is a way to solve problems using several computation resources at the same time. It is a powerful alternative to solve problems which would require too much time by using sequential algorithms \cite{Grama2003}. %That is why this field is in constant development and it is the topic where we put most of our effort. 

Since the late 2000's all processors in modern machines are multi-core. Massively parallel architectures, previously expensive and so far reserved for super--computers, become now a trend available to a broad public through hardware like the Xeon Phi or GPU cards. The power delivered by massively parallel architectures allow us to treat faster constrained problems \cite{Borkar2007}. However this architectural evolution is a non-sense if algorithms do not evolve at the same time: the development and the implementation of algorithms should take this into account and tackling problems with very different methods, changing the sequential reasoning of researchers in Computer Science \cite{Hill2008, Sanders2014}. 

In the literature on parallel constraint solving \cite{Gent}, two main limiting factors on performance are addressed: \begin{inparaenum}[1-] \item inter-process communication overheads (explained in details in Section~\ref{sec:cooperation}), and \item the \textit{Amdahl}'s law. \end{inparaenum} The Amdahl's law of parallel computing states that the \textit{speed-up} of a parallel algorithm is limited by the fraction of the program that must be executed sequentially. It means that adding more processors may not make the program run faster. It assumes that some percentage of the program or code cannot be parallelized ($T_{sequential}$), and states that the ratio called speed-up of $T_{sequential}$ over $1 - T_{Sequerntial} = T_{parallel}$ is bounded by $1\setminus T_{sequential}$ when the number of processors $P \rightarrow \infty$:

\begin{equation}\label{amdahl}
Speed-Up = \frac{T_{sequential}}{T_{parallel}} \leq \frac{1}{T_{sequential}}
\end{equation}

%\item parallelizing the search process,  
%\item parallel and distributed arc-consistency, 
%\item multi-agent and cooperative search and
%\item combined parallel search and parallel consistency.

Another issue, usually underestimated, is the codification. Writing efficient code for parallel machines is less trivial, as it usually deals with low-level APIs such as OpenMP and message-passing interfaces (MPI), among others. However, years of experience have shown that using those frameworks is difficult and error-prone. Usually many undesired behaviors (like deadlocks) make parallel software development very slow compared to sequential approaches. In that sense, Falcou proposes in \cite{Falcou2009} a programming model: \textit{parallel algorithmic skeletons} (along with a C++ implementation called {\sc Quaff}) to make parallel application development easier. This model is a high-order pattern to hide all low-level, architecture or framework dependent code from the user, and provides a decent level of organization. {\sc Quaff} is a skeleton-based parallel programming library, which has demonstrated its efficiency and expressiveness solving some application from computer vision. It relies on C++ template meta-programming to reduce the overhead traditionally associated with object-oriented implementations of such libraries allowing some code generation at compilation time. \etal{Cahon} also propose {\it ParadisEO}, a framework to design parallel and distributed hybrid meta-heuristics showing very good results, including a broad range of reusable features to easily design evolutionary algorithms and local search methods \cite{Cahon2004}.

%The contribution in terms of hardware has been crucial, achieving powerful technologies to perform large--scale calculations. 

The development of techniques and algorithms to solve problems in parallel focuses principally on three fundamental aspects: 
\begin{enumerate}%\begin{inparaenum}[i)]
    \item {\it Problem subdivision},
    \item {\it Search parallelization} and %{\it Scalability} and
    \item {\it Inter-process communication}.
\end{enumerate}%\end{inparaenum}
They all pursue the same objective: achieving good levels of \textit{scalability}. Scalability is the ability of a system to handle the increasing growth of workload. 
%A system which has improved over time its performance after adding work resources, and it is capable of doing it proportionally is called {\it scalable}. 
%The increase has not been only in terms of calculus resources, but also in the amount of sub-problems coming from the sub-division of the original problem. The more we can divide a problem into smaller sub-problems, the faster we can solve it \cite{Hill}. 
\textit{Adaptive Search} is a good example of local search method that can scale up to a larger number of cores, e.g., a few hundreds or even thousands of cores \cite{Diaz}. For this algorithm, an implementation of a cooperative multi-walks strategy has been published by \etal{Munera} in \cite{Munera}. In this framework processes are grouped in teams to achieve search intensification, which cooperate with others teams through a head node (process) to achieve search diversification. Using an adaptation of this method, authors propose a parallel solution strategy able to solve hard instances of \textit{Stable Marriage with Incomplete List and Ties Problem} quickly. This technique has been combined in \cite{Munera2016} with an \textit{Extremal Optimization} procedure: a nature-inspired general-purpose meta-heuristic \cite{Boettcher2000}.
	
The issue of subdividing a given problem in some smaller sub-problems is sometimes not easy to address. Even when we can do it, the time needed by each process to solve its own part of the problem is rarely balanced. For that reason it is imperative to apply some complementary techniques to tackle this problem, taking into account that sometimes, the more a problem can be sub-divided, the more balanced will be the execution times of the process \cite{Hill}.%\cite{Rezgui2013, Hill}. 
In \cite{Arbab2000} Arbab and Monfory propose a mechanism to create sub-\csps{} (whose union contains all the solutions of the original \csp) by splitting the domain of the variables. The coordination is achieved though communication between processes. The contribution of this work is explained in details in Section~\ref{sec:cooperation}. 

In \cite{Yasuhara2015} \etal{Yasuhara} propose a new search method called \textit{Multi-Objective Embarrassingly Parallel Search} (MO--EPS) to solve multi-objective optimization problems, based on: 
\begin{inparaenum}[i)]
	\item Embarrassingly Parallel Search (EPS), where the initial problem is split into a number of independent sub-problems, by partitioning the domain of decision variables \cite{Regin2014}; and %\cite{Rezgui2013, Regin2014}; and
	\item Multi-Objective optimization adding cuts (MO--AC), an algorithm that transforms the multi-objective optimization problem into a feasibility one, searches a feasible solution and then the search is continued adding constraints to the problem until either the problem becomes infeasible or the search space gets entirely explored \cite{Kotecha2010}.
\end{inparaenum}
Multi-objective optimization problems involve more than one objective function to be optimized simultaneously. Usually these problems do not have an unique optimal solution because there exist a trade-off between one objective function and the others. For that reason, in a multi-objective optimization problem, the concept of \textit{pareto optimal} points is used. A pareto optimal point is a solution that improving one or some objective function values, implies the deterioration of at least one of the other objective function. %A collection of pareto optimal points defines a pareto front.

Related to parallelizing the search process, we can find two main approaches. First, the {\it single walk} approach, in which all the processes try to follow the same path towards the solution, solving their corresponding part of the problem, with or without cooperation (communication). The other is known as {\it multi walk}, consisting of the execution of various independent processes to find the solution. Each process applies its own strategies (portfolio approach) or simply explores different places inside the search space. Although this approach may seem too trivial and not so smart, it is fair to say that it is in fashion due to the good obtained results using it \cite{Diaz}.

\etal{Kishimoto} present in \cite{Kishimoto2013} a comparison between \textit{Transposition-table Driven Scheduling} (TDS) and a parallel implementation of a best-first search strategy (Hash Distributed A$^*$), that uses the standard approach of \textit{work stealing} for partitioning the search space. This technique is based on maintaining a local work queue, (provided by a root process through hash-based distribution that assign an unique processor to each work) accessible to other process that "steal" work from it if they become unoccupied. Authors use MPI, the paradigm of \textit{Message Passing Interface} that allows parallelization, not only in distributed memory based architectures, but also in shared memory based architectures and mixed environments (clusters of multi-core machines) \cite{Grama2003a}. The same approach is used by Jinnai and Fukunaga in \cite{Jinnai} to evaluate \textit{Zobrist Hashing}, an efficient hash function designed for table games like Chess and Go, to mitigate communication overheads.

In \cite{Arbelaez2012} \etal{Arbelaez} present a study of the impact of space-partitioning techniques on the performance of parallel local search algorithms to tackle the \textit{k-medoids} clustering problem. Using a parallel local search method, this work aims to improve the scalability of the sequential algorithm, which is measured in terms of the quality of the solution within a given timeout. Two main techniques are presented for domain partitioning: first, {\it space-filling curves}, used to reduce any N-dimensional representation into a one-dimension space (this technique is also widely used in the nearest-neighbor-finding problem \cite{Chen2005}); and second, {\it k-Means} algorithm, one of the most popular clustering algorithms \cite{Berkhin2002}.

An interesting work is presented by \etal{Truchet} in \cite{Truchet02}, which is an estimation of the speed-up (a performance prediction of a parallel algorithm) through statistical analysis of its sequential algorithm is presented. Using this approach it is possible to have a rough idea of the resources needed to solve a given problem in parallel. In this work, authors study the parallel performances of \textit{Las Vegas} algorithms \cite{Babai1979} (randomized algorithms whose runtime might vary from one execution to another, even with the same input) under independent multi-walk scheme, and predict the performances of the parallel execution from the runtime distribution of their sequential runs. These predictions are compared to actual speed--ups obtained for a parallel implementation of the same algorithm and show that the prediction can be quite accurate.

The other important aspect in parallel computing is the inter-process communication, also called \textit{solver cooperation} and it is treated in the next section.

%A lot of studies have been published around this topic. A parallel solver for numerical \csps{} is presented in \cite{Ishii2014} showing good results scaling on a number of cores.
%We can find in \cite{Diaz2012} a survey of the different parallel programming models and available tools, emphasizing on their suitability for high-performance computing.

%Some other efforts have been allocated in the exploitation of the power of calculus provided by the massively parallel architecture of the Graphic Processing Unit (GPU). In \cite{Arbelaez} are presented  implementations of efficient (and very fast) constraint-based local search solvers using GPU.
%\nocite{GPU}

\section{Solvers cooperation}\label{sec:cooperation}
The interaction between solvers exchanging information is called {\it solver cooperation}. %and it is very popular in this field due to their good results. 
Its main goal is to improve some kind of limitations or inefficiency imposed by the use of unique solver. In practice, each solver runs in a computation unit, i.e. thread or processor. The cooperation is performed through inter--process communication, by using different methods: \textit{signals}, asynchronous notifications between processes in order to notify an event occurrence; \textit{semaphore}, an abstract data type for controlling access, by multiple processes, to a common resource; \textit{shared memory}, a memory simultaneously accessible by multiple processes; \textit{message passing}, allowing multiple programs to communicate using messages; among others.

Many times a close collaboration between process is required, in order to achieve the solution. But the first inconvenient is the slowness of the communication process. Some work have achieved to identify what information is viable to share. One example is the work presented by \etal{Hamadi} in \cite{Hamadi2012}, where an idea to include low-level reasoning components in the SAT problems resolution is proposed, dynamically adjusting the size of shared clauses to reduce the possible blow up in communication. This approach allows to perform the clause-sharing, controlling the exchange between any pair of processes.

This is a very changeling field, that is way we can find a lot of interesting ideas in the literature to improve parallel solutions through solver cooperation techniques. 

\etal{Kishimoto} present in \cite{Kishimoto2009} a parallelization of an algorithm A$^*$ (Hash Distributed A$^*$) for \textit{optimal sequential planning} \cite{Schmegner2004}, exploiting distributed memory computers clusters, to extract significant speedups from the hardware. In classical planning solving, both the memory and the CPU requirements are main causes of performance bottlenecks, so parallel algorithms have the potential to provide required resources to solve changeling instances. 

In \cite{Pajot2003} Pajot and Monfroy present a paradigm that enables the user to properly separate strategies combining solver applications, from the way the search space is explored in solver cooperations. The cooperation must be supervised by the user, through {\it cooperation strategy language}, which defines the solver interactions during the search process.

{\sc Meta--S} is an implementation of a theoretical framework proposed in \cite{Frank2003} by \etal{Franc}, which allows to tackle constrained problems, through the cooperation of arbitrary domain--specific constraint solvers. Through its modular structure and its extensible strategy specification language, it also serves as a test--bed for generic and problem--specific \nobreak{(meta-)solving} strategies, which are employed to minimize the incurred cooperation overhead. Treating the employed solvers as black boxes, the meta--solver takes constraints from a global pool and propagates them to the individual solvers, which are in return requested to provide newly gained information (i.e., constraints) back to the meta--solver, through variable projections. The major advantage of this approach lies in the ability to integrate arbitrary, new or pre--existing constraint solvers, to form a system that is capable of solving complex mixed--domain constraint problems, at the price of increased cooperation overhead. This overhead can however be reduced through more intelligent and/or problem--specific cooperative solving strategies. 

%In \cite{Frank2003} have been presented an implementation of the meta-solver framework which coordinates the cooperative work of arbitrary pluggable constraint solvers. This approach intents to integrate arbitrary, new or pre--existing constraint solvers, to form a system capable of solving complex mixed--domain constraint problems. The existing increased cooperation overhead is reduced through problem-specific cooperative solving strategies.

%{\sc Hyperion} \cite{Brownlee2014} is an already mentioned framework for meta-- and hyper--heuristics built with the principle of interoperability, generality by providing generic templates for a variety of local search and evolutionary computation algorithms; and efficiency, allowing rapid prototyping with the possibility of reusing source code.

Arbab and Monfory propose in \cite{Arbab2000} a technique to guide the search by splitting the domain of variables. A \textit{master} process builds the network of variables and domain reduction functions, and sends this information to the \textit{worker} processes. Workers concentrate their efforts on only one sub-\csp{} and the master collects solutions. The main advantage is that by changing only the search agent, different kinds of search can be performed. The coordination process is managing using the {\sc Manifold} coordination language \cite{Arbab1995}.

A component-based constraint solver in parallel is proposed in \cite{Zoeteweij} by Zoeteweij and Arbab. In this work, a parallel solver coordinates autonomous instances of a sequential constraint solver, which is used as a software component. The component solvers achieve load balancing of tree search through a time-out mechanism. It is implemented a specific mode of solver cooperation that aims at reducing the turn-around time of constraint solving through parallelization of tree search. The main idea is to try to solve a \csp{} before a time-out. If it cannot find a solution, the algorithm defines a set of disjoint sub-problems to be distributed among a set of solvers running in parallel. The goal of the time-out mechanism is to provide an implicit load balancing: when a solver is idle, and there are no subproblems available, another solver produces new sub-problems when its time-out elapses.

%{\sc Manifold} is a strongly-typed, block-structured, event-driven language for managing events, dynamically changing interconnections among sets of independent, concurrent and cooperative processes. A {\sc Manifold} application consists of a number of processes running on a heterogeneous network. Processes in the same application may be written in different programming languages. {\sc Manifold} has been successfully used in a broad range of applications \cite{Arbab1995}.

\etal{Munera} present in \cite{Munera} a new paradigm that includes cooperation between processes, in order to improve the independent multi-walk approach. In that case, cooperative search methods add a communication mechanism to the independent walk strategy, to share or exchange information between solver instances during the search process. This proposed framework is oriented towards distributed architectures based on clusters of nodes, with the notion of {\it teams} running on nodes and controlling several search engines ({\it explorers}) running on cores. All teams are distributed and thus have limited inter--node communication. This tool provides diversification through communication between teams, extending the search to different regions of the search space. Intensification is ensured through communication between explorers, and it is achieved swarming to the most promising neighborhood found by explorers. %This framework is oriented towards distributed architectures based on clusters of nodes, where teams are mapped to nodes and explorers run on cores. 
This framework was developed using the {\it X10 programming language}, which is a novel language for parallel processing developed by IBM Research, giving more flexibility than traditional approaches, e.g. MPI communication package.

A similar approach is presented by \etal{Guo} in \cite{Guo2010}, exploring principles of diversification and intensification in portfolio--based parallel SAT solving. To study their trade--off, they define two roles for the computational units. Some of them classified as {\it masters} perform an original search strategy, ensuring diversification. The remaining units, classified as {\it slaves} are there to intensify their master's strategy. 
%There are some important questions to be answered:
%\begin{inparaenum}[i)]
%	\item what information should be given to a slave in order to intensify a given search effort?, 
%	\item how often, a subordinated unit has to receive such information? and 
%	\item the question of finding the number of subordinated units and their connections with the search efforts? 
%\end{inparaenum}
Results lead to an original intensification strategy which outperforms the best parallel SAT solver {\it ManySAT}, and solves some open SAT instances.

\etal{Hamadi} propose in \cite{Hamadi2011} the first {\it Deterministic Parallel DPLL} (a complete, backtracking-based search algorithm for deciding the satisfiability of propositional logic formulas in conjunctive normal form) engine. The experimental results show that their approach preserves the performance of the parallel portfolio approach while ensuring full reproducibility of the results. Parallel exploration of the search space, defines a controlled environment based on a total ordering of solvers interactions through synchronization barriers. The frequency of exchanges (conflict-clauses) influences considerably the performance of the solver. The paper explores the trade off between frequent synchronizing which allows the fast integration of foreign conflict--clauses at the cost of more synchronizing steps, and infrequent synchronizing at the cost of delayed foreign conflict-clauses integration.

Considering the problem of parallelizing restarted backtrack search (the problem of finding the right time to to restart the search after some fails), \etal{Cire} have developed in \cite{Cire2011} a simple technique for parallelizing restarted search deterministically. They demonstrate experimentally that they can achieve near--linear speed--ups in practice, when the number of processors is constant and the number of restarts grows to infinity. The proposed technique is the following: each parallel search process has its own local copy of a scheduling class which assigns restarts and their respective fail--limits to processors. This scheduling class computes the next {\it Luby} restart fail--limit and adds it to the processor that has the lowest number of accumulated fails so far, following an {\it earliest--start--time--first strategy}. Like this, the schedule is filled and each process can infer which is the next fail--limit that it needs to run based on the processor it is running on -- without communication. Overhead is negligible in practice since the scheduling itself runs extremely fast compared to CP search, and communication is limited by informing other processes when a solution has been found.

\section{Parameter setting techniques}\label{sec:tunning}
Most of previously exposed methods to tackle combinatorial problems, involve a number of parameters that govern their behavior, and they need to be well adjusted. Most of the times they depend on the nature of the specific problem, so they require a previous analysis to study their behavior \cite{Birattari2005}. That is why another branch of the investigation arises: {\it parameter tuning}. It is also known as a meta optimization problem, because the main goal is to find the best solution (parameter configuration) for a program, which will try to find the best solution for some problem as well. In order to measure the quality of some found parameter setting for a program (solver), one of these criteria are taken into consideration: the speed of the run or the quality of the found solution for the problem that it solves. The selection of proper parameters for a particular algorithm is a quite complicate subject. This is the reason why many researchers are motivated to develop techniques to find good parameter settings automatically.

There exist tow classes to classify these methods: 
\begin{enumerate}
\item \textit{Off-line tuning}: Also known just as parameter tuning, were parameters are computed before the run, and
\item \textit{On-line tuning}: Also known as parameter control, were parameters are adjusted during the run.
\end{enumerate}

\subsection{Off-line tuning}

The technique of parameter tuning or off-line tuning, is used to compute the best parameter configuration for an algorithm before the run (solving a given instance of a problem), in order to obtain the best performance. The found parameters configuration does not change once the run is started. 

There exist some techniques to tune algorithms. The most common is the so called {\it racing procedure} which is based on a simple idea: sequentially evaluating candidates (parameter setups) on a series of benchmark instances and eliminate parameter configurations as soon as they are found too far behind the candidate with the overall best performance at a given stage of the race (\textit{incumbent}). A popular variant of this technique is the \textit{F-Race} method. It is also based on eliminating parameter configurations in some given steps. However, rather than just performing pairwise comparisons with the incumbent, it first uses the \textit{rank-based Friedman test}: for some independent randomly chosen variables, it assesses whether configurations in the race show no significant performance differences on some given instances. If there exist some configurations show better results than others, a series of pairwise tests between the incumbent and all other configurations is performed. All configurations found to have substantially worse results than the incumbent are removed from the race. The procedure is terminated either when only one configuration remains, or after a given timeout \cite{Hoos2012}. \etal{Eiben} present in \cite{A.E.Eiben2012} a study of this methods and their applications tuning Evolutionary Algorithms (EA).

{\sc Revac} is a method presented in \cite{Nannen2007} by \etal{Nannen}, based on information theory to measure parameter relevance, to calibrate parameters of EAs in a robust way. Instead of estimating the performance of an EA for different parameter values, the method estimates the expected performance when parameter values are chosen following a probability density distribution $C$. The method iteratively refines the probability distribution $C$ over possible parameter sets, and starting with a uniform distribution $C_0$ over the initial parameter space $\mathcal{X}$, the method gives a higher probability to regions of $\mathcal{X}$ that increase the expected performance of the target EA. In \cite{Smit2010} \etal{Smit} present a case study demonstrating that using {\sc Revac} the "world champion" EA (the winner of the CEC-2005 competition) can be improved with few effort.

In \cite{Riff2013}, \etal{Riff} present \textit{EVOCA}, a tool which allows meta-heuristics designers to obtain good parameter configuration with few effort. \textit{EVOCA} use an EA to find a good parameter setting for a meta-heuristic, and it is used as a step of the iterative design process. This tool allows designer to find parameter settings both for quantitative parameters (numerical values) and for qualitative parameters (e.g. crossover operators). Another tool for this end, but this time using local search methods to find the parameter setting is proposed by \etal{Hutter} in \cite{Hutter2009}. It has been applied with success in many combinatorial problems in order to find the best parameter configuration. {\sc ParamILS} It is an open source program written in {\it Ruby}, and the public source include some examples and a detailed and complete user guide with a compact explanation about how to use it with a specific solver \cite{Hutter2008}.

A different but interesting technique was successfully used to tune parameters for EAs through a model based on a {\it case-based reasoning} system. The work presented in \cite{Yeguas2014} by \etal{Yeguas} attempts to imitate the human behavior in solving problems: look in the memory how we have solved a similar problem.

\subsection{On-line tunning}

Although parameter tunning shows to be an effective way to adjust algorithms parameters, in some problems the optimal parameter settings may be different for various phases of the search process. This is the main motivation to use on-line tuning techniques to find the best parameter setting, also called \textit{parameter control techniques}. Parameter control techniques are divided into 
\begin{inparaenum}[i)]
\item \textit{deterministic parameter control}, where the value of a parameter is altered by some deterministic rule, ignoring any feedback; 
\item \textit{adaptive parameter control}, which continually update their parameters using feedback from the population or the search, and this feedback is used to determine the direction or magnitude of the parameter changes; and 
\item \textit{self-adaptive parameter control}, which assigns different parameters to each individual. Here, parameters to be adapted are coded into the chromosomes that undergo mutation and recombination
\end{inparaenum}\cite{Eiben1999}.

\etal{Drozdik} present in \cite{Drozdik} a study of various approaches to find out if one can find an inherently better one in terms of performance and whether the parameter control mechanisms can find favorable parameters in problems which can be successfully optimized only with a limited set of parameters. They focused in the most important parameters: 
\begin{inparaenum}[i)]
\item the \textit{scaling factor}, which controls the structure of new invidious; and
\item the \textit{crossover probability}.
\end{inparaenum}

Differential Evolution (DE) algorithm has been demonstrated to be an efficient and robust optimization method. However, its performance is very sensitive to the parameters setting, and this sensibility changes from a problem to another. \etal{Liu} propose in \cite{Liu2005} an adaptive approach which uses fuzzy logic controllers to guide the search parameters, with the novelty of changing the mutation control parameter and the crossover operator during the optimization process. \textit{SaDE} is a self-adaptive DE algorithm proposed by \etal{Qin} in \cite{Qin2009}, where both trial vector generation strategies and their associated control parameter values are gradually adjusted by learning from the way they have generated their previous promising solutions, eliminating this way the time-consuming exhaustive search for the most suitable parameter setting. This algorithm has been generalized by \etal{Huang} to multi-objective realm, with objective-wise learning strategies (\textit{OW-MOSaDE}) \cite{Huang2009}.

{\sc Meta-GAs} is a genetic self-adapting algorithm proposed by \etal{Clune}, adjusting operators of genetic algorithms. In this paper the authors propose an approach of moving towards a Genetic Algorithm that does not require a fixed and predefined parameter setting, because it evolves during the run \cite{Clune2005}.


\section{Summarizing}

In this chapter we have presented an overview of the different techniques to solve \CSPs{}. %Special attention was given to the \textit{local-search meta-heuristics}, as well as \textit{parallel computing}, which are directly related to this investigation.
The classics are the so called tree-search methods, which perform an exhaustive search of the solution inside the search space. The main disadvantage of this methods is that in practice, real-world instances of the problems are very complicated to solve, due to the search space's huge size, making them in most of the cases intractable. Some techniques can be applied to reduce the search space. This is the case of constraint propagation, which are methods used to modify a \csp{} in order to reduce its variables domains, and turning the problem into another that is equivalent, but usually easier to solve. However, sometimes it is not enough, and we must resort %turn 
to more powerful methods, like meta-heuristics. 

%In contrast with tree-based methods (complete methods), 
Meta-heuristic methods have shown good results solving large and complex \csps. They are algorithms applying different techniques to guide the search as directly as possible through the solution. Meta-heuristic methods are divided into two categories. In the first category we can find population-based methods (\eg evolutionary algorithms). These methods work with a set of configurations (population). Every iteration the population is capable of reproducing itself and is subject to variations. New configurations are created by applying some modification's operators. The fitness of the resulting population is evaluated and a selection strategy is applied to determine configurations to the next generation. In the second category we can find single-solution-based methods (\eg tabu search). Also called trajectory methods, they start from a candidate configuration inside the search space. To this configuration some local modifications are applied to create a set of configuration called neighborhood. Then a selection criteria is applied to choose a new configuration for the next iteration. In this thesis, special attention was given to techniques in this category.

With the goal of improving even more the obtained results, some other techniques have been implemented. This is the case of hyper-heuristics, which are automated techniques for selecting or generating meta-heuristics algorithms to solve hard computational problems. They are also known as portfolio-based algorithm, and their goal is predicting the performance of algorithms using different techniques in order to select the predicted algorithm to solved the problem. An other popular approach is the hybridization, which combines different approaches into the same solution strategy. We can find hybridization in combining algorithms in the same branch of investigation, but not only there. Interesting results have been published when combining constraint programming, meta-heuristics and operations research techniques.

The era of multi/many-core computers, and the development of parallel algorithms have opened new ways to solve constrained problems. A lot of results have been presented suggesting that this approach can substantially improve the efficiency in solving these problems, with or without cooperation. For that reason the present work attempts to propose new algorithms to solve \CSPs{} in parallel in Chapter~\ref{chap:posl}, and to show the importance and the success of this approach by providing a deep study of some parallel \comstrs{} in Chapter~\ref{chap:expe}.

%The main contribution of this thesis is presented in Chapter~\ref{chap:posl}, where is proposed a framework to build local-search meta-heuristics combining small functions (\oms) through an operator-based language. \textit{Hybridization} is also an important point in this investigation due to their good results in solving \csps. With the proposed framework, many different solvers can be created using solvers templates (\ass), that can be instantiated with different \oms.