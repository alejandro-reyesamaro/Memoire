\chapter{Conclusion and future works}
\label{chap:conclusion}
\textit{In this chapter, the conclusions of the work is presented, emphasizing on our contribution and obtained results. Future branches to follow are also discussed.}
\vfill
\minitoc
\newpage

\section{Conclusions}
\label{sec:conclusion_conclusion}

The era of parallel computing has opened new and more efficient ways to solve constraint problems. This development is leading us to the multi/many--core technology and massive parallel architectures, which are nowadays more accessible for a broad public through hardware like the Xeon Phi or GPU cards. For that reason, this new architecture implies new ways for designing and implementing algorithms to exploit its full potential.

In this thesis I have presented as a main contribution a Parallel-Oriented Solver Language (\posl) focused in the solution of \CSPs, which are very complicated. These problems have huge search spaces, making them intractable through tree-search techniques. \posl{} propose a language to build meta-heuristic-based solvers, tacking into account the success of these methods solving \csps{}. This meta-heuristics are built using the \posl's language following rigorous but well detailed steps, based on the re-usability and coupling small pieces of computation and communication (\oms{} and \opchs), designed to the resolution of a broad range of \csps. 

Meta-heuristic methods have some times a lot of parameters to be adjusted. Prior to the \posl{}'s design, Chapter~\ref{chap:prior} (Section~\ref{sec:paramils}) contains a study in which the tool {\sc ParamILS} was used to tune {\it Adaptive Search} to solve \carr{} and {\it All-Interval Series} problems. The main goals of that work were studying the performance of the tool, and finding a new and more efficient parameters setting that allow a faster resolution of the mentioned benchmark problems. However, the conclusion, after a comparison between obtained results using default parameters found through manually experiments, and obtained results using {\sc ParamILS}, were that, for this implementation of {\it Adaptive Search} the tool is not able to find parameter settings improving obtained results using default parameters. This corroborates the practical intuition that, when the parameters set is not so large, the experience of the scientist is crucial and more accurate that using this kind of tools.

The most important characteristic of \posl{} is allowing the construction of many solvers to work in parallel using the \textit{multi-walk} approach, which has shown very good results solving constrained problems. Into another work prior to \posl's design, I have presented a study of some techniques to improve the performance of algorithms proposed in \cite{Arbelaez2012} were a study of the impact of space-partitioning techniques on the performance of parallel local search algorithms is proposed to tackle the \textit{K-Medoids Clustering Problem}. \modified{The basic idea of their specific problem is to how allocate communication metronodes in order to maximize the client covering.} Their solution is based on domain partitioning techniques like {\it space-filling curves}, and {\it k-Means} algorithm, but they do not take into account the number of clients associated to each new sub-domain. For that reason, in Chapter~\ref{chap:prior} (Section~\ref{sec:split}) are proposed a set of ideas/hypothesis to improve the performance, based on geometrical balancing of the search space. This work was not validated, because it was performed in parallel with the first ideas of \posl, which finally was the main direction of this thesis.

In Chapter~\ref{chap:posl} was dedicated to \posl, the main contribution of this thesis, a Parallel-Oriented Solver Language to build
interconnected meta-heuristic-based solvers working in parallel. The language was formally presented by defining each provided operator, as well as the benchmark codification method, and the process of creation/usage of the \bothmodules. 

The most important advantage of \posl{} is allowing the codding, easily and fast, of many different solvers through a mechanism of module re-usability, and communication strategies through communication operators, which are also formally defined. Hence, as other contribution of this thesis, is presented in Chapter~\ref{chap:expe} a detailed study of various communication strategies to analyze the behavior and relevance of the information sharing solving constraint problems.

Solving \sgp{}, it was successfully applied an exploitation-oriented communication strategy, in which the current configuration is communicated to focus various solvers in a more promising area. The same idea was applied to solve the \nqp{}, showing no better results than obtained without communication. However, a deep study of the \posl's behavior during the search process allows to design a communication strategy able to improve the results obtained using non-communicating strategies. It was based on crating \textit{partial} solvers (solvers only searching into a portion of the search space) to accelerate other's solvers search, by communicating the current configuration at the beginning of the search process. The \carrp{} is a very complicated constrained problem, and very sensitive to the methods to solve it. Thanks to some studies of different communication strategies, based on the communication of the current configuration at different times (places) in the algorithm, it was possible to find a communication strategy to improve the performance, in comparison with those obtained without communication. Finally, the \grp{} was chosen to study a different and innovative communication strategy in which the communicated information is a potential local minimum to be avoided. This new communication strategy showed to be effective to solve these kind of problems.

Thanks to the operator-based language provided by \posl{} it was possible to test many different strategies (communicating and non-communicating). The process of building solvers implementing different solution strategies is complex and tedious, but \posl{} gives the possibility to make communicating and non-communicating solver prototypes and to study them with few efforts. It was possible to show that a good selection and management of inter-solvers communication can play an important role during the search process, working with constrained problems, most of them very complicated.

\section{Future works} 

\posl{} already has an important library of ready--to--use computation and connection modules, based on a deep study of classical meta-heuristics algorithms for solving combinatorial problems. In the near future we plan to make it grow, in order to increase possibilities of \posl{}. In such a way, building new algorithms by using \posl{} will be easier.
At the same time we plan to enrich the language by proposing new operators. It is necessary, for example, to improve the solver definition language, allowing to build sets of many new solvers faster and easier. Furthermore, we are aiming to expand the communication definition language, in order to create versatile and more complex and dynamic communication strategies, to allow a communication strategy to change during runtime.

The operators described above only give the possibility to define static communication strategies. However, we aim to improve \posl{} with more expressive operators in terms of communication between solvers to allow dynamic modifications of communication strategies, that is, having such strategies adapting themselves during runtime. This way, many different communication strategies would be defined in the same \soset. Then, after some time of calculation, an evaluation would be performed in order to make all solvers able to adopt the best strategy until the end of the search process.

\tet{In the performed experiments, the shared information was in every case the best found configuration. So far, there are no results showing what "a good information to communicate" is. Actually, \cite{Caniou14} shows that in fact, the current configuration is not always a relevant information to share among solvers. That is why this subject deserves a deep study. We plan in the near future to investigate other informations to be communicated, such as really costly configurations, in order to avoid similar ones; search directions, to be avoided or taken into account; among others.}