\chapter{Conclusion}
\label{chap:conclusion}
\textit{We resume our work, emphasizing on our contribution and obtained results, and we expose the conclusions of the work. We also discus future branches to follow that can be derived from our work. Finally we give our conclusions.}
\vfill
\minitoc
\newpage

In this paper we have analyzed some results using \posl{}, a Parallel--Oriented Solver Language to solve constraint-based problems \cite{Reyes-amaro, ale2015mic}, to solve instances of the Social Golfers and Costas Array problems. It was possible to implement different communicating and non communicating strategies using the operator-based language provided. \posl{} gives the possibility to define different solver connections, to tune the percentage of communicating solvers, the moment when this communication takes place inside each iteration and the type of information to communicate. Results show \posl{} ability to solve these problems, showing at the same time that communication can play a decisive role in the search process, and provide more stable results. We show also that for some problems, the communication of the current configuration, as a mechanism of search intensification, has a positive effect in the search process. Although, a deep study to the nature of information to communicate would been very beneficial to make efficient parallel solvers. \modified{We also show that the communication of \textit{tabu} configurations can accelerate the convergence to a solution.}

\posl{} already has an important library of ready--to--use computation and connection modules, based on a deep study of classical meta-heuristics algorithms for solving combinatorial problems. In the near future we plan to make it grow, in order to increase possibilities of \posl{} %. In such a way, building new algorithms by using \posl{} will be easier.
%At the same time we plan to enrich the language 
by proposing new operators. It is necessary, for example, to improve the solver definition language, allowing to build sets of many new solvers faster and easier. Furthermore, we are aiming to expand the communication definition language, in order to create versatile and more complex and dynamic communication strategies, to allow a communication strategy to change during runtime.

%The operators described above only give the possibility to define static communication strategies. However, we aim to improve \af{} with more expressive operators in terms of communication between solvers to allow dynamic modifications of communication strategies, that is, having such strategies adapting themselves during runtime. This way, many different communication strategies would be defined in the same {\bf \af{}-Solver}. Then, after some time of calculation, an evaluation would be performed in order to make all solvers able to adopt the best strategy until the end of the search process.

%\tet{In the performed experiments, the shared information was in every case the best found configuration. So far, there are no results showing what "a good information to communicate" is. Actually, \cite{Caniou14} shows that in fact, the current configuration is not always a relevant information to share among solvers. That is why this subject deserves a deep study. We plan in the near future to investigate other informations to be communicated, such as really costly configurations, in order to avoid similar ones; search directions, to be avoided or taken into account; among others.}