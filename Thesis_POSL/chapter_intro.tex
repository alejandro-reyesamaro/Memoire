\chapter{Introduction}
\label{chap:Intro}
\textit{Designing and implementing programs solving \CSPs{} (\csp) is complex and tedious. The main contribution of this thesis is to propose a Parallel-Oriented Solver Language (\posl), as well as the first implementation of this framework. \posl{} provides simple mechanisms to build different solvers working in parallel and to establish \commstrs{} with few effort.1}
%\textit{In this chapter \CSPs{} (\csp) are introduced as the main target problems. \posl{}, the main contribution of this thesis, is proposed: a Parallel-Oriented Solver Language to build many different solvers working in parallel. \posl{} also provides simple mechanisms to construct communication strategies with few effort. Finally, the structure of this document is presented.}
\vfill
\minitoc
\newpage

Combinatorial optimization is the act of computing the best result for a problem from a finite set of possibilities. These possibilities are the different combinations of values that the variables can take (configurations), taking into account their domains (this thesis takes into account only discrete domains with integer values), and sometimes a given finite set of restrictions (constraints). These problems have several applications in many fields. In airlines, the crew scheduling is an example of \COP. It consists of covering flights of the company scheduled in a given time window, with minimum cost, making an efficient and realistic use of the available personal. The problem of task assignment for parallel programs is one of the most important in parallelism because it represents the core of a good program performance running on several machines \cite{Paschos2013}. %The design of telecommunication networks is an other example of these kind of problems, and it becomes more complicated in the same way as the development of the technology. 
\COPs{} are also present in electrical engineering, when an efficient circuit layout design is needed \cite{Barahona1988}; as well as in the constructions of master-keys (known as the \textit{Master-keying} problem), which consists in designing a key capable to open the maximum number of doors in a building\cite{Espelage2000}.

In some cases, all feasible solutions (\ie configurations fulfilling all constraints) have the same importance, meaning no solution is better than another one. Hence, the main goal is only to find  one of these solutions. These problems are called {\it \CSPs{}} (\csp). Here, a solution is an assignment of variables satisfying the constraint set. There exist many different techniques to solve such problems, mainly classified into two categories. \new{The first category are constraint programming methods. These techniques have two main stages:
\begin{inparaenum}[1)] \item \textit{Constraint propagation}, a constraint reasoning technique also appearing under the name of filtering algorithm, and consists in explicitly forbidding values or combinations of values for some variables of a given problem because a given subset of its constraints cannot be satisfied otherwise. Its main goal is to reduce the problem variables domains, and turning it into one that is equivalent, but with a smaller search space, hence usually easier to solve. 
%This way, an equivalent but smaller and easier to solve problem is obtained. 
\item \textit{Backtracking search algorithms}, one of the main algorithmic techniques for solving \csps{}. They performs a depth-first traversal of the search space, \ie variables are assigned successively and respecting some order, until arriving to a dead end (an assignation into the search space no leading to a solution) or finding a solution. They are also called complete methods, because guarantee that a solution will be found if one exists, and can be used to prove that a \csp{} does not have a solution.
%or tree-search. It incrementally constructs candidate solutions, by assigning values to variables. Each time an assignment cannot possibly be completed to a valid solution, the previous assignation is discarded. 
\end{inparaenum} }

\CSPs{} are NP-complete. In practice large problem instances are intractable for tree-based search algorithms. These instances are more likely to be solved by \textit{meta-heuristics}, the second category of solvers.
%In practice, \CSPs{} are intractable. Their search space is huge enough to make tree-search based algorithms useless to solve them. In contrast, in the second category, \textit{meta-heuristics} methods are algorithms that have been shown to be effective in the resolution of these kind of problems. They are an iterative generation process which guides algorithms by combining intelligently different mechanisms for exploring and exploiting the search space in order to find efficiently near-optimal solutions \cite{Osman1996}. 
In this category, two groups of methods can be found: %\cite{Boussaid2013}: 
\begin{inparaenum}[a)] \item Single-solution based methods (also known as \textit{local search}). They start with an initial solution and move trying to improve it inside the search space. \item Population-based methods. in these methods, a set of solutions is modified through operators (recombination, mutation, etc.). \end{inparaenum} Both are nature-inspired methods.

The development of computer architecture is leading us toward massively multi/many--core computers. These architectures unlock new algorithmic possibilities to tackle problems sequential algorithms cannot handle by reducing the search times.  However, this architecture mutation must go hand by hand with the mutation of parallel algorithms. %At the time of writing a solution algorithm in parallel, some important decisions have to be taken. 
The way of organizing the search (either by dividing the search space, or by dividing the problem into smaller and easier to solve sub-problems) can provide an important speed-up. However, other parallel scheme, like the multi-walk parallel scheme, also called embarrassingly parallel scheme, where solvers try to solve the whole problem but searching in different zones of the search space, have shown also very good results \cite{Arbelaez2014a, Caniou2011, Diaz2012a}. %Another important decision is whether to use communication between process as a cooperation mechanism. 

Cooperation mechanism can be achieved via inter-process communications. Theoretically, sharing information between solvers helps the search process, but in practice, an equilibrium between the contribution of the communication and its inherent overheads is needed.

%Many results can be showed in parallel computing. 
Adaptive Search~\cite{Diaz} is an efficient method reaching linear speed-ups on hundreds and even thousands of cores (depending of the problem), using an independent multi-walk local search parallel scheme. Munera et al~\cite{Munera} present another implementation of this algorithm using communication between search engines, showing the efficiency of cooperative multi-walks strategies. All these results use a multi-walk parallel approach and show the robustness and efficiency of this parallel scheme. %Although, they all concluded there is room for improvements. 

With all these elements, the idea of making a tool to quickly prototype solution and communication strategies emerged. In that sense, the \textbf{main goal} of this work is to provide a framework to build/use easily and rapidly:
\begin{enumerate}
\item \textit{Computation modules}: simple functions easily reusable, which can be used to built meta-heuristic-based algorithm by joining them.
\item \textit{Abstract solvers}: algorithms templates, which can be used to build many different solvers, instantiating different \oms.
\item Different \comstrs.
\end{enumerate} 

\section{Goals and contributions}

The \underline{main~contribution} of this thesis is the design and \hfill \\\nobreak{implementation~of}: 
\begin{center}
\posl{} (pronounced "puzzle")\\
\textbf{Parallel-Oriented Solver Language}
\end{center}
\posl{} is a framework based on the creation or reuse of \textit{modules} interconnected by operators, to build local-search solvers. These solvers work in parallel using the multi-walk parallel scheme, and are connected each other through communication operators, allowing information sharing. It is well-known that software programming is a very time-consuming activity. This is even more the case while developing a parallel software, where debugging is an extremely difficult task. That is why \posl{} is based on re-usability to propose to \csp{} solver designers/programmers a parallel framework to quickly build parallel prototypes, speeding-up the design process.

Before all, \posl{} is a language designed to combine \ms{} available in the framework, or to create new ones. There exist two types of \ms{}: \oms{}, simple functions receiving an input, then executing an internal algorithm and returning an output, and \opchs{}, responsible for the information sharing.
The created or ready-to-use \ms{} are joined by operators defined by \posl's language creating independent solvers. Then, solvers can be connected each other using communication operators. Such solvers compose what we call a \soset. %\posl{} also provides a framework specification to implement the benchmark (problems to solve). %, respecting some requirements.

%In this thesis we present \posl, a framework for easily building many and different cooperating solvers based on coupling four fundamental and independent components: \oms, \opchs, the \ass{} and \comstrs. Recently, the hybridization approach leads to very good results in constraint satisfaction. For that reason, since the solver's component can be combined, our framework is designed to execute in parallel sets of different solvers, with and without communication.

This framework was inspired by a similar idea proposed by Alex S. Fukunaga in \cite{Fukunaga2008} without communication, introducing an evolutionary approach that uses a simple composition operator to automatically discover new local search heuristics for SAT and to  visualize them as combinations of a set of building blocks. \etal{Landtsheer} present in \cite{Landtsheer2015} a framework to facilitate the development of search procedures by using \textit{combinators} to design features commonly found in search procedures as standard bricks and joining them. This approach can speed-up the development and experimentation of search procedures when developing a specific solver based on local search. In \cite{Martin2016}, \etal{Martin} propose an approach of using cooperating local search processes using an asynchronous message passing protocol. The cooperation is based on the general strategies of pattern matching and reinforcement learning. \posl{} uses the combination of both ideas, by combining features of the search process through provided operators, and it also provides an operator-based mechanism to connect solvers, creating \comstrs.

%In the last phase of the coding process with \posl{}, solvers can be connected each others, depending on the structure of their \opchs, and this way, they can share not only information, but also their behavior, by sharing their \oms. This approach makes the solvers able to evolve during the execution.

The \underline{second contribution} of the present work is a detailed study of \commstrs{} %the solution process of some \CSPs{} chosen as benchmarks, using \posl{}. In this study some different strategies are proposed, 
in order to show how the communication can play an important role to speed-up the search process.

%A first study was made using the \sgp{}, in which a \textit{standard} communication strategy is used: the communication of the current configuration. This strategy shows to be effective because it helps to preserve the equilibrium between exploration and exploitation, necessary in the efficient resolution of this problem.

%With the \carrp{}, a similar study was performed, with the slightly difference that the configuration was transmitted in different places of the algorithm.

%Another study was performed using the \nqp{}, in which it was observed that a standard communication strategy is not enough to improve the results without communication. However, with this benchmark a strategy of search-space partitioning was implemented, \new{which was able to accelerate the search process. Hence results in terms of runtime and iterations show that it is effective only for small instances.}

%Finally, the \grp{} was used to study a different communication strategy, in which the current configuration is communicated. In contrast to the previews strategies, this configuration is not used to be improved, but to be avoided. The principle is to communicate potential local minima to some solvers, and they will avoid them every time they perform a restart. 

In every case, it was possible to show the positive effect of the inter-processes communication. Despite intrinsic overheads, the solver cooperation scheme can help significantly in the search process, if it is studied and chosen correctly. For that reason this study has allowed the validation of the effectiveness of \posl{} to this purpose.

\section{Structure of the document}

Chapter~\ref{chap:art} presents an overview to the state-of-the-art of \COPs{}. Its definition and the link with \CSPs{} (\csp) are presented, as well as the principal methods to solve them. This chapter is a travel among \begin{inparaenum}[1)] \item the basic techniques of constraint programming and meta--heuristic methods; \item the advanced techniques such as hyper--heuristic methods, hybridization, parallel computing, and solvers cooperation; and \item parameter setting techniques. \end{inparaenum}

%In Chapter~\ref{chap:prior} prior works leading to \posl{} are presented. The problem subdivision approach was adopted to divide the domain of a given problem in parallel, in particular, to solve the \textit{K-Medoids Problem}. It contains also a performed study applying the {\sc ParamILS} tool to find the optimum parameter configuration to \textit{Adaptive Search} solver. {\sc ParamILS} (version 2.3) is a tool for parameter optimization for parametrized algorithms.

In Chapter~\ref{chap:posl}, we formally present \posl, the Parallel-Oriented Solver Language that is the heart of this thesis and its main contribution to the community. Its characteristics, main advantages, and a general procedure to be followed in order to use it to solve \CSPs{} are
exposed.

Results for each study to build \posl{} \comstrs{} to solve the proposed benchmarks are presented in Chapter~\ref{chap:expe}. In each section, a benchmark problem is defined, the used \soset{} for each communication strategy is presented, and results are analyzed. %(details of experiments can be found in Appendices). 

Main results of this thesis are summarized in Chapter~\ref{chap:conclusion}, where possible new lines of investigation are also discussed.

This document includes also Appendixes~\ref{app:sgp}, \ref{app:cap}, \ref{app:nqp}, and~\ref{app:grp} in order to bring more detailed information about results of the experimentation process.
%This document includes some appendixes in order to bring more detailed information: %A theoretical study of the problem subdivision approach applied to divide the variable domains of the \textit{K-Medoids Problem} is presented in Appendix~\ref{chap:prior_split} as a prior works leading to \posl{}. Another study was performed: applying the {\sc ParamILS} tool to find the optimum parameter configuration to \textit{Adaptive Search} solver (presented in Appendix~\ref{chap:prior_paramils}). 
%Appendix~\ref{app:sgp}, \ref{app:cap}, \ref{app:nqp}, and~\ref{app:grp} show detailed results of the experimentation process.