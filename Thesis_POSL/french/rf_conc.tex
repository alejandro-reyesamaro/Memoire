Dans cette thèse, nous avons présenté \posl{}, un système pour construire des solveurs parallèles coopératifs. Il propose une manière modulable  pour créer des solveurs capables d'échanger n'importe quel type d'informations, comme par exemple leur comportement même, en partageant
leurs \oms. Avec \posl{}, de nombreux solveurs différents pourront être créés et lancés en parallèle, en utilisant une unique stratégie générique mais en instanciant différents \oms{} et \opchs{} pour chaque solveur. 

Il est possible d'implémenter différentes stratégies de communication, puisque \posl{} fournit une couche pour définir les canaux de communication connectant les solveurs entre eux.

Nous avons présenté aussi des résultats en utilisant \posl{} pour résoudre des instances des problèmes classiques CSP. Il a été possible d'implémenter différentes stratégies communicatives et non communicatives, grâce au langage basé sur des opérateurs fournis, pour combiner différents \oms{}. \posl{} donne la possibilité de relier dynamiquement des solveurs, étant capable de définir des stratégies différentes en terme de pourcentage de solveurs communicatifs. Les résultats montrent la capacité de \posl{} à résoudre ces problèmes, en montrant en même temps que la communication peut jouer un rôle décisif dans le processus de recherche.

\posl{} a déjà une importante bibliothèque de \oms{} et de \opchs{} prête à utiliser, sur la base d'une étude approfondie sur les algorithmes méta-heuristiques classiques pour la résolution de problèmes combinatoires. Dans un avenir proche, nous prévoyons de la faire grandir, afin d'augmenter les capacités de \posl.

En même temps, nous prévoyons d'enrichir le langage en proposant de nouveaux opérateurs. Il est nécessaire, par exemple, d'améliorer le langage de {\it définition du solveur}, pour permettre la construction plus rapide et plus facile des ensembles de nombreux nouveaux solveurs. En plus, nous aimerions élargir le langage des opérateurs de communication, afin de créer des stratégies de communication polyvalentes et plus complexes, utiles pour étudier le comportement des solveurs.