L'optimisation combinatoire a plusieurs applications dans diff\'erents domaines tels que l'apprentissage de la machine, l'intelligence artificielle, et le g\'enie du logiciel. Dans certains cas, le but principal est seulement de trouver une solution, comme pour les Probl\`emes de Satisfaction de Contraintes (CSP). Une solution sera une affectation de variables r\'epondant aux contraintes fix\'ees, en d'autres termes: une solution faisable.

Plus formellement, un CSP (d\'enot\'e par $\mathcal{P}$) est d\'efini par le trio $\langle X,D,C \rangle$  o\`u $X = \{x_1, x_2,\dots,x_n\}$ est un ensemble fini de variables; $D = \{D_1, D_2,\dots, D_n\}$, est l'ensemble des domaines associ\'es \`a chaque variable dans $X$; et $C = \{c_1, c_2,\dots,c_m\}$, est un ensemble de contraintes. Chaque contrainte est d\'efinie en impliquant un ensemble de variables, et sp\'ecifie les combinaisons possibles de valeurs de ces variables. Une configuration $s\in D_1\times D_2\times\dots\times D_n$, est une combinaison de valeurs des variables dans $X$. Nous disons que $s$ est une solution de $\mathcal{P}$ si et seulement si $s$ satisfait toutes les contraintes $c_i \in C$.

Les {\it CSP}s sont connus pour \^etre des probl\`emes extr\^emement difficiles. Parfois les m\'ethodes compl\`etes ne sont pas capables de passer \`a l'\'echelle de probl\`emes de taille industrielle. C'est la raison  pour laquelle les techniques m\'eta--heuristiques sont de plus en plus utilis\'ees pour la r\'esolution de ces derniers. Par contre, dans la plupart des cas industriels, l'espace de recherche est assez important et devient donc intraitable, m\^eme pour les m\'ethodes m\'eta-heuristiques. Cependant, les r\'ecents progr\`es dans l'architecture de l'ordinateur nous conduisent vers les ordinateurs {\it multi/many--c\oe ur}, en proposant une nouvelle fa\c{c}on de trouver des solutions \`a ces probl\`emes d'une mani\`ere plus r\'ealiste, ce qui r\'eduit le temps de recherche.

Gr\^ace \`a ces d\'eveloppements, les algorithmes parall\`eles ont ouvert de nouvelles fa\c{c}ons de r\'esoudre les probl\`emes de contraintes: Adaptive Search \cite{Diaz} est un algorithme efficace, montrant de tr\`es bonnes performances et passant \`a l'\'echelle de plusieurs centaines ou m\^eme milliers de c\oe urs, en utilisant la recherche locale {\it multi-walk} en parall\`ele. Munera et al. \cite{Munera} ont pr\'esent\'e une autre impl\'ementation d'Adaptive Search en utilisant la coop\'eration entre des strat\'egies de recherche. {\it Meta--S} est une impl\'ementation d'un cadre th\'eorique pr\'esent\'e dans \cite{Frank2003}, qui permet d'attaquer les probl\`emes par la coop\'eration de solveurs de contraintes de domaine sp\'ecifique. 
Ces travaux ont montr\'e l'efficacit\'e du sch\'ema parall\`ele multi-walk.  

De plus, le temps de d\'eveloppement n\'ecessaire pour coder des solveurs en parall\`ele est souvent sous-estim\'e, et dessiner des algorithmes efficaces pour r\'esoudre certains probl\`emes consomme trop de temps. Dans cette thèse nous présentons \posl{}, un langage orient\'e parall\`ele pour construire des solveurs de contraintes bas\'es sur des m\'eta-heuristiques, qui r\'esolvent des {\it CSP}s. Il fournit un m\'ecanisme pour dessiner des stratégies de communication. L'autre but de cette thèse est de présenter une analyse d\'etaill\'ee des r\'esultats obtenus en r\'esolvant plusieurs instances des probl\`emes CSP. Sachant que cr\'eer des solveurs utilisant diff\'erentes strat\'egies de solution peut être compliqu\'e et p\'enible, \posl{} donne la possibilit\'e de faire des prototypes de solveurs communicants facilement.