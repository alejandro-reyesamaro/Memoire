Beaucoup de chercheurs se concentrent sur la programmation par contraintes, particuli\`erement dans le d\'eveloppement de solution \`a haut-niveau, qui facilite la construction de strat\'egies de recherche. Cela permet de citer plusieurs contributions. 

{\sc Hyperion} \cite{Brownlee2014} est un syst\`eme cod\'e en Java pour m\'eta et hyper-heuristiques bas\'e sur le principe d'interop\'erabilit\'e, fournissant des patrons g\'en\'eriques pour une vari\'et\'e d'algorithmes de recherche locale et \'evolutionnaire, et permettant des prototypages rapides avec la possibilit\'e de r\'eutiliser le code source. \posl{} offre ces avantages, mais il fournit \'egalement un m\'ecanisme permettant de d\'efinir des protocoles de communication entre solveurs. Il fournit aussi, \`a travers d'un simple langage bas\'e sur des op\'erateurs, un moyen de construire des \ass, en combinant des \ms{} d\'ej\`a d\'efinis (\oms{} et \opchs). Une id\'ee similaire a \'et\'e propos\'ee dans \cite{Fukunaga2008} sans communication, qui introduit une approche \'evolutive en utilisant une simple composition d'op\'erateurs pour d\'ecouvrir automatiquement les nouvelles heuristiques de recherche locale pour SAT et les visualiser comme des combinaisons d'un ensemble de blocs.

R\'ecemment, \cite{El-Ghazali2013} a montr\'e l'efficacit\'e de combiner diff\'erentes techniques pour r\'esoudre un probl\`eme donn\'e (hybridation). Pour cette raison, lorsque les composants du solveur peuvent \^etre combin\'es, \posl{} est dessin\'e pour ex\'ecuter en parall\`ele des ensembles de solveurs diff\'erents, avec ou sans communication. Une autre id\'ee int\'eressante est propos\'ee dans {\sc Templar}. Il s'agit d'un syst\`eme qui g\'en\`ere des algorithmes en changeant des composants pr\'ed\'efinis, et en utilisant des m\'ethodes hyper-heuristiques \cite{Swan2015}. Dans la derni\`ere phase du processus de codage avec \posl{}, les solveurs peuvent \^etre connect\'es les uns aux autres, en fonction de la structure de leurs \opchs, et de cette fa\c{c}on, ils peuvent partager non seulement des informations, mais aussi leur comportement, en partageant leurs \oms. Cette approche donne aux solveurs la capacit\'e d'\'evoluer au cours de l'ex\'ecution.

Renaud De Landtsheer et al. pr\'esentent dans \cite{Landtsheer2015} un cadre facilitant le d\'eveloppement des syst\`emes de recherches en utilisant des \textit{combinators} pour dessiner les caract\'eristiques trouv\'ees tr\`es souvent dans les proc\'edures de recherches comme des briques et les assembler. Dans \cite{Martin2016} est propos\'ee une approche qui utilise des syst\`emes coop\'eratifs de recherche locale bas\'ee sur des m\'eta-heuristiques. Celle-ci se sert de protocoles de transfert de messages. \posl{} combine ces deux id\'ees pour assembler des composants de recherche locale \`a travers des op\'erateurs fournis (ou en cr\'eant des nouveaux), mais il fournit aussi un m\'ecanisme bas\'e sur op\'erateurs pour connecter et combiner des solveurs, en cr\'eant des strat\'egies de communication.

Dans cette th\`ese, nous pr\'esentons quelques nouveaux op\'erateurs de communication afin de concevoir des strat\'egies de communication. Avant de clore ce résumé par une br\`eve conclusion, nous pr\'esentons quelques r\'esultats obtenus en utilisant \posl{} pour r\'esoudre certaines instances des probl\`emes {\it Social Golfers}, {\it Costas Array}, \textit{N-Queens} et \textit{Golomb Ruler}.