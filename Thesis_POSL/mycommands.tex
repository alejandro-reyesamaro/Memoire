\newcommand{\Aref}[1]{Appendix  \ref{#1}} %\thechapter

%----------------- Colors --------------------------
\definecolor{naranja}{RGB}{241,70,34}
\definecolor{verde}{RGB}{104,198,135}
\definecolor{dred}{RGB}{120,7,7}
\definecolor{darkgreen}{rgb}{0.0, 0.42, 0.24}
\definecolor{orange}{RGB}{241,70,34}
\definecolor{gray}{RGB}{135,131,131}
\definecolor{intenso}{RGB}{126,13,13}
\definecolor{shadecolor}{RGB}{215,215,215}

%----------------- Commands --------------------------
\newcommand{\tet}[1]{\textcolor{naranja}{#1}}
\newcommand{\new}[1]{\textcolor{blue}{#1}}
\newcommand{\modified}[1]{\textcolor{verde}{#1}}
\newcommand{\highlight}[1]{\textcolor{red}{#1}}
\newcommand{\good}[1]{\textcolor{dred}{\bf #1}}
\newcommand{\commentary}[1]{\textcolor{verde}{#1}}
\newcommand{\hsep}[1]{\hspace{7pt} #1}

\newcommand{\gracias}[1]{\textcolor{red}{\bf #1}}
\newcommand{\receiver}[1]{\textcolor{dred}{\bf #1}}
\newcommand{\sender}[1]{\textcolor{blue}{\bf #1}}
\newcommand{\nocomm}[1]{\textcolor{black}{\bf #1}}

\newcommand{\ale}{Alejandro {\sc Reyes Amaro}}
\newcommand{\titre}{{\sc POSL}: A Parallel-Oriented Solver Language}
\newcommand{\posl}{{\sc POSL}}

\newcommand{\om}{\omprefix{} module}
\newcommand{\oms}{\om s}
\newcommand{\INTROom}{{\it \omprefix{} module}}
\newcommand{\opch}{communication module}
\newcommand{\opchs}{\opch s}
\newcommand{\INTROopch}{{\it \opch}}
\newcommand{\dopch}{data \opch}
\newcommand{\INTROdopch}{{\it \dopch}}
\newcommand{\oopch}{object \opch}
\newcommand{\INTROoopch}{{\it \oopch}}
\newcommand{\comstr}{communication strategy}
\newcommand{\comstrs}{communication strategies}
\newcommand{\INTROcomstr}{{\it \comstr}}
\newcommand{\omprefix}{computation}
\newcommand{\bothmodules}{\omprefix{} and \opchs}
\newcommand{\m}{module}
\newcommand{\ms}{\m s}
\newcommand{\INTROm}{{\it \m}}
\newcommand{\as}{abstract solver}
\newcommand{\ass}{\as s}
\newcommand{\Ass}{Abstract solvers}
\newcommand{\INTROas}{{\it \as}}
\newcommand{\cm}{compound module}
\newcommand{\cms}{\cm  s}
\newcommand{\Cms}{Compound modules}
\newcommand{\INTROcm}{{\it compound module}}
\newcommand{\soset}{solver set}
\newcommand{\INTROsoset}{{\it \soset}}
\newcommand{\jack}{communication jack}
\newcommand{\jacks}{\jack s}
\newcommand{\INTROjack}{{\it \jack}}
\newcommand{\outlet}{communication outlet}
\newcommand{\outlets}{\outlet s}
\newcommand{\INTROoutlet}{{\it \outlet}}
\newcommand{\operation}{operation}
\newcommand{\INTROoperation}{{\it \operation}}
\newcommand{\INTROcommoper}{\it communication operator}
\newcommand{\INTROcommopers}{\it \INTROcommoper s}
\newcommand{\commoper}{communication operator}
\newcommand{\commopera}{\commoper s}

\newcommand{\poslop}[1]{\textbf{  }\circled{$#1$}\textbf{  }}
\newcommand{\poslopcond}[1]{\textbf{  }\circled{?}_{#1}\textbf{  }}

\usepackage{stmaryrd}
\newcommand{\lbk}{\left\llbracket}
\newcommand{\rbk}{\right\rrbracket}
\newcommand{\parallelexec}[1]{\lbk#1\rbk_p}
\newcommand{\lsendpar}{\llparenthesis}
\newcommand{\rsenddatapar}{\rrparenthesis^d}
\newcommand{\rsendmodulepar}{\rrparenthesis^m}
\newcommand{\senddataop}[1]{\lsendpar#1\rsenddatapar}
\newcommand{\sendmoduleop}[1]{\lsendpar#1\rsendmodulepar}

\newcommand{\produce}{\hspace{5pt}\longmapsto\hspace{5pt}}
\newcommand{\OR}{\hspace{5pt}|\hspace{5pt}}

\newcommand{\TM}{$^{\scalebox{0.6}{\mbox{TM}}}$}
\newcommand{\R}{$^{\scalebox{0.7}{\textregistered}}$}
\newcommand{\curiosiphyfull}{Intel\R{} Xeon\TM{} E5-2680 v2 (10$\times$4 cores, 2.80GHz)}
\newcommand{\curiosiphy}{Intel\R{} Xeon\TM{}}
\newcommand{\mylaptopProc}{Intel Core i7-4702HQ 2.2 GHz}
\newcommand{\mylaptopMemo}{16384 MB, Dual-channel DDR3L 1600 MHz}
\newcommand{\mylaptopName}{Dell XPS 15}

\newcommand{\pclass}[1]{\mbox{\textcolor{blue}{\textbf{#1}}}}
\newcommand{\pmethod}[2]{\mbox{\textcolor{green}{\textbf{\it #1}}({\it #2})}}

\def\chapterautorefname{Chapter}
\usepackage{xspace}

\usepackage{pgfplots}
\usetikzlibrary{matrix}
\usepgfplotslibrary{groupplots}
\pgfplotsset{compat=newest}

% <Verbatim>
\usepackage{fancyvrb}
\fvset{frame=single,framesep=1mm,fontfamily=courier,fontsize=\scriptsize,framerule=.3mm,numbersep=1mm,commandchars=\\\{\}} %numbers=left
% </Verbatim>

% <ConsoleLikeListing>
\usepackage{tcolorbox,listings}
\lstdefinestyle{mystyle}{
     basicstyle=\ttfamily\bf\tiny\color{white}, %
     moredelim=**[is][\color{red}]{@}{@}
     %numbers=left, 
     %numberstyle=\tiny, 
     %numbersep=5pt     
 }
\tcbuselibrary{listings,skins,breakable}
\newtcblisting{BGVerbatim}{
      arc=0mm,
      top=0mm,
      bottom=0mm,
      left=1mm,
      right=0mm,
      width=\textwidth,
      boxrule=0.5pt,
      colback=black,
      colupper=white,
      spartan,
      listing only,
      %listing options={style=mystyle},
      breakable
}
% </ConsoleLikeListing>

\newcommand{\textover}[2]{\begin{tabular}{c}#1\\#2\end{tabular}}
\newcommand{\tablePILSresults}[1]{
\begin{tabular}{|p{0.5cm}|p{0.5cm}|p{0.5cm}|p{0.5cm}|p{0.5cm}||p{0.5cm}|p{0.5cm}|p{0.5cm}|p{0.5cm}|p{0.5cm}||p{2cm}|p{1.7cm}|p{1.7cm}|}
	\hline\hline
	\multicolumn{5}{|c||}{\bf Initial configuration} & \multicolumn{5}{c||}{\bf Final best configuration} & \multirow{2}{*}{\footnotesize{\centering {\bf \textover{Training}{quality}}}} & \multirow{2}{*}{\footnotesize{\centering {\bf \textover{Number}{of runs}}}} & \multirow{2}{*}{\footnotesize{\centering {\bf \textover{Test}{quality}}}} \\ %[0.5ex]
	\cline{1-10}
	F & P & f & l & p & F & P & f & l & p &  &  &  \\ %[1ex]
	\hline\hline
	#1
	\hline
\end{tabular}}

\newcommand{\COP}{\textit{Combinatorial Optimization Problem}}
\newcommand{\COPs}{\COP\textit{s}}
\newcommand{\CSP}{\textit{Constraint Satisfaction Problem}}
\newcommand{\CSPs}{\CSP\textit{s}}
\newcommand{\csp}{\textit{CSP}}
\newcommand{\csps}{\csp\textit{s}}
\newcommand{\CP}{\textit{Constraint Programing}}

\newcommand{\sg}{{\it Social Golfers}}
\newcommand{\sgp}{\sg{} {\it Problem}}
\newcommand{\SGP}{{\it SGP}}
\newcommand{\nq}{{\it N-Queens}}
\newcommand{\nqp}{\nq{} {\it Problem}}
\newcommand{\NQP}{{\it NQP}}
\newcommand{\carr}{{\it Costas Array}}
\newcommand{\carrp}{\carr{} {\it Problem}}
\newcommand{\CARRP}{{\it CAP}}
\newcommand{\gr}{{\it Golomb Ruler}}
\newcommand{\grp}{\gr{} {\it Problem}}
\newcommand{\GRP}{{\it GRP}}

\newtheorem{definition}{Definition}

\definecolor{NatGreen}{RGB}{50,93,61}
\newcolumntype{x}[1]{%
>{\raggedleft\hspace{0pt}}m{#1}}%



\hypersetup
{pdftitle={Spectroscopic Tools for Quantitative Studies of DNA Structure and Dynamics.},
pdfauthor={Søren Preus},
pdfsubject={PhD thesis. ``Spectroscopic Tools for Quantitative Studies of DNA Structure and Dynamics.'' }, %subject of the document
%pdftoolbar=false, % toolbar hidden
pdfmenubar=true, %menubar shown
pdfhighlight=/O, %effect of clicking on a link
colorlinks=true, %couleurs sur les liens hypertextes
pdfpagemode=UseOutlines,%UseNone, %aucun mode de page
pdfpagelayout=TwoPageRight,%SinglePage, %ouverture en simple page
pdffitwindow=true, %pages ouvertes entierement dans toute la fenetre
linkcolor=black, %couleur des liens hypertextes internes
citecolor=black, %couleur des liens pour les citations
urlcolor=black, %couleur des liens pour les url
bookmarksopenlevel=2
}

\usepackage[framemethod=tikz]{mdframed}
\usetikzlibrary{shadows}
\newmdenv[shadow=true,shadowcolor=black,font=\sffamily,rightmargin=8pt]{shadedbox}

\newcommand{\defname}[2]{ \begin{definition}\textbf{(#1)}
#2
\end{definition}}

\newcommand{\myboxlist}[2]
{
\begin{list}{\boxed{#1 \arabic{qcounter}:~}}{\usecounter{qcounter}} \itemsep0em
#2
\end{list}
}

\newcommand*\circled[1]{\tikz[baseline=(char.base)]{
		\node[shape=circle,draw,inner sep=2pt] (char) {#1};}}


%-----------------ALGORITHM2e--------------------------
\usepackage{algorithm2e}
\SetAlFnt{\small\sf}
\newcommand\mycommfont[1]{\footnotesize\ttfamily\textcolor{verde}{#1}}
\SetCommentSty{mycommfont}
%\newsavebox{\mycode}
\SetKwFor{While}{$[\circlearrowleft $}{}{$]$} %{\tet{\bf begin}}{\tet{\bf end}$]$}
\SetKwIF{Strategy}{}{}{\tet{\bf abstract solver}: }{}{}{}{}
\SetKwIF{oModule}{}{}{\tet{\bf computation} : }{}{}{}{}
\SetKwIF{oChannel}{}{}{\tet {\bf connection} : }{}{}{}{}

\SetKwBlock{Begin}{\tet{\bf begin}}{\tet{\bf end}}
%\SetKwIF{If}{ElseIf}{Else}{if}{then}{else if}{else}{endif}
\SetKwData{Iter}{{\sc Itr}}
\SetKwData{Sci}{{\sc Sci}}
\SetKwData{sec}{\text{  }\circled{$\mapsto$}\text{  }}
\SetKwData{kdom}{{\bf oModule}}
\SetKwData{kdoch}{{\bf oChannel}}
\SetKwFunction{str}{\bf strategy}
\SetKwFunction{solver}{\bf solver}
\newcommand{\whileinline}[2] {$[\circlearrowleft $ #1 #2 $]$}%{$[\circlearrowleft $ #1 \tet{\bf begin} #2 \tet{\bf end}$]$}


\newcounter{qcounter}

\usepackage{array}
\newcolumntype{L}[1]{>{\raggedright\let\newline\\\arraybackslash\hspace{0pt}}m{#1}}
\newcolumntype{C}[1]{>{\centering\let\newline\\\arraybackslash\hspace{0pt}}m{#1}}
\newcolumntype{R}[1]{>{\raggedleft\let\newline\\\arraybackslash\hspace{0pt}}m{#1}}

%\usepackage{xcolor}
%\usepackage{newverbs}
%\newverbcommand{\bverb}{\color{blue}}{}

\newenvironment{example}{\fontsize{10pt}{12pt}\fontfamily{phv}\selectfont}{\par}

\definecolor{mycolor}{rgb}{0.122, 0.435, 0.698}% Rule colour
\makeatletter
\newcommand{\mybox}[1]{%
  \setbox0=\hbox{#1}%
  \setlength{\@tempdima}{\dimexpr\wd0+13pt}%
  \begin{tcolorbox}[colframe=mycolor,boxrule=0.5pt,arc=4pt,
      left=6pt,right=6pt,top=6pt,bottom=6pt,boxsep=0pt,width=\@tempdima]
    #1
  \end{tcolorbox}
}
\makeatother



\usepackage{mdframed}
\newmdenv[
  topline=false,
  bottomline=false,
  skipabove=\topsep,
  skipbelow=\topsep
]{siderules}

\newcommand{\poslexample}[1]{
\begin{siderules}
{\fontsize{10pt}{12pt}\fontfamily{phv}\selectfont #1}
\end{siderules}
}

\usepackage{enumerate}
\usepackage{bbding}
\newcommand{\separation}{
\begin{center}
{\small \Asterisk{} \Asterisk{} \Asterisk{}}
\end{center}
\vspace{10pt}
}