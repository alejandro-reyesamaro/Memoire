%----------------------------------------------------------------------------------------------
%------ POSL
%----------------------------------------------------------------------------------------------
\chapter[\posl{}: Parallel Oriented Solver~Language]{A Parallel-Oriented Language for Modeling Meta-Heuristic-Based Solvers and communication strategies}
\label{chap:posl}
\textit{In this chapter \posl{} is introduced as the main contribution of this thesis, and a new way to solve \csps{} (Section \ref{sec:posl_intro}). Its characteristics and advantages are summarized, and a general methodology for building parallel solvers using \posl{} is described. Then a detailed description of each step is presented: modeling the target benchmark in Section~\ref{sec:model}, creating \posl's modules in Section~\ref{sec:1ststage}, assembling \posl's modules in Section~\ref{sec:2ndstage}, creating \posl{} solvers in Section~\ref{sec:3rdstage}, connecting solvers in Section~\ref{sec:4thstage} and then a brief summary in Section~\ref{sec:posl_zum}).}
\vfill
\minitoc
\newpage

\section{Introduction}
\label{sec:posl_intro}

Meta-heuristic methods, despite showing very good results for solving \CSPs, are frequently not enough for solving instances with extremely large search spaces. Most of these methods are sensible to their large number of parameters. For that reason, a first direction for this thesis was to tackle one of the weakest points of meta-heuristic methods: theirs parameters. In Appendix~\ref{chap:prior_paramils} a performed study applying {\sc ParamILS} to {\it Adaptive Search} in order to find a general parameter settings was presented. This experiment did not produce encouraging results. That is why it was decided to abandon the idea as the main direction of the thesis, but not as future work.

From the beginning of the current investigation, the target problems were big and complex instances of \csps. For that reason, even if the current version of the framework does not provide auto tuning mechanisms, this thesis focuses on the implementation of a mechanism to easily build solvers based on local search meta-heuristic, providing an easy way of reusing algorithm's components commons to different methods.

With the development of parallelism, opening new ways to tackle constrained problems, the accessibility to this technology to a broad public has also increased. It is available through multi-core personal computers, Xeon Phi cards and GPU video cards. For that reason it was decided to focus this thesis completely on the parallel approach. In Appendix~\ref{chap:prior_split} it was presented a study in which the problem-subdivision approach was applied to the resolution of {\it K-Medoids Problem}. The main goal of this work was generalizing the proposed ideas to similar problems. It was only a theoretical study, performed in parallel with what would latter be the main scientific contribution of this thesis.

Many results from the literature indicate that the combination of meta-heuristic methods with parallelism provides very good results for large scale \csps. This investigation focuses in the implementation of the multi-walk parallel approach. Most of the methods found in the state of the art of this field are based on applying clever techniques to accelerate the solution process of specific problems. The present work does not apply partitioning techniques neither for the search space nor for the target problem. This make the proposed framework applicable in a general and more easy way for a broad range of problems.

Another weak point of the development process that is frequently undervalued is the codding time, which is always long when coding parallel programs. This was the main motivation to start searching techniques for implementing parallel solution strategies with or without communication in a fast and easy way. The main goal was creating a tool providing 
\begin{inparaenum}[1-]
%\item a simple way to create \textit{flexible} solvers, i.e. solvers ables to be modified with a few effort; 
\item fast and simple mechanisms to connect solvers, ables to exchange information; 
\item and a way to create numerous and different parallel strategies, where different communicating and not communicating solvers can be combined, exploiting to the maximum computation resources.
\end{inparaenum}

\subsection{Precedents}

During the development process, some inspired ideas were taken into account. {\sc Hyperion}$^2$ \cite{Brownlee2014} is a java framework for building meta-- and hyper--heuristics providing generic templates for a variety of local search and evolutionary computation algorithms, allowing quick prototyping with the possibility of reusing source code. This tool illustrate a bit one of the main goals of this thesis. Rapid and fast prototyping of algorithms through high-level languages is more and more imperative due to the increasing demand of algorithms to solve very complex algorithms coming from the development of the technology. Nevertheless, this solution does not take into account parallelism.

Alex~S.~Fukunaga propose in~\cite{Fukunaga2008} an evolutionary approach that uses a simple composition operator to automatically discover new local search heuristics for SAT and to visualize them as combinations of blocks. A similar idea is presented in~\cite{Landtsheer2015} by \etal{Landtsheer}, a framework to facilitate the development of local search methods by using \textit{combinators} to design features commonly found in these methods as standard bricks and joining them. Authors define four types of bricks: \begin{inparaenum}[1-] \item neighborhoods functions, \item strategies to scape from local minima, \item solution managers, allowing to store the best found solution during the search, and \item stop criteria.
\end{inparaenum} This approach can speed up the development and experimentation of search procedures when developing a specific solver based on local search. The goal of this thesis is to create a tool offering the same advantages, but providing also a mechanism to define communication protocols between solvers working in parallel. It must also provide a way to create an abstract solver by combining simple components or functions that called \ms.

{\it ParadisEO} is a framework presented by \etal{Cahon} in~\cite{Cahon2004} to design parallel and distributed hybrid meta-heuristics showing very good results, including a broad range of reusable features to easily design evolutionary algorithms and local search methods. \etal{Martin} propose in~\cite{Martin2016} an approach of using cooperating meta--heuristic based local search processes, using an asynchronous message passing protocol. The cooperation is based on the general strategies of pattern matching and reinforcement learning. 

%The main challenge faced in this thesis is finding the right communication strategy  

We can cite a lot of works to prove that high-level framework for the development of efficient algorithms, as well as the parallel approach with or without communication have been widely used reporting good results. However, a solution combining a modular way to construct algorithms with tools to manage the parallelism is still missing. It is well know that interprocess communication can help during the search process of constrained solvers, but it is also well know that this communication is very complicated to perform in practice, due to certain overheads. In that sense this thesis propose a framework to create parallel communication strategies, by providing tools to easily manipulate:
\begin{enumerate}
\item \textbf{where to perform the communication} $\rightarrow$ using operators to combine information reception modules with \cms{} inside the algorithm, 
\item \textbf{what to communicate} $\rightarrow$ configurations, neighborhoods, configuration costs, algorithms, etc., and
\item \textbf{how to perform the communication} $\rightarrow$ providing instructions to create and connect in different ways sets of solvers.
\end{enumerate}

%The tool developed for this thesis, uses this idea where search process features can be combined and reused, and it is also possible to design communication strategies between solvers.

\subsection{POSL}

In this chapter is presented \posl{}, the main contribution of this thesis, as well as the different steps to build communicating parallel solvers with. It is proposed as a new way to implement \textit{solution algorithms} to solve \CSPs, through local-search meta-heuristics using the multi-walk parallel approach. It is based on improving step by step an initial configuration, driven by a \textit{cost function} provided by the user through the model. The implementation must follow the following stages.

\begin{enumerate}
\item The conceived \textit{solution algorithm} to solve the target problem is decomposed into small modules of computation, which are implemented as separated {\it functions}. We name them \oms{} (see Figure~\ref{subfig:modules}, blue shapes). At this point it is crucial to find a good decomposition of its \textit{solution algorithm}, because it will have a significant impact in its future re-use. In this stage is also decided which information is interesting to \textit{receive} from other solvers. This information is encapsulated into another kind of component called \opch, allowing data transmission between solvers (see Figure~\ref{subfig:modules}, red shapes).
\item Solvers templates called \INTROass{} are created through \posl{}'s inner language only by using module signatures. 
\item \Ass{} are instantiated with concrete modules (\oms{} and \opchs). It allows good levels of code re-use: many different solvers can be created sharing the same template. %(the interested reader is referred to  Appendix~\tet{[...]}) 
\item The parallel-oriented language based on operators provided by \posl{} (see Figure~\ref{subfig:as}, green shapes) allows the information exchange, and executing modules in parallel. In this stage the information that is interesting to be shared with other solvers is sent using operators. After that we can connect them using {\it communication operators}. This final entity is called a \INTROsoset{} (see Figure~\ref{subfig:conn}).
\end{enumerate}

\begin{figure}[h]
	\centering
	\subfloat[][Creating \posl's modules]{
		\label{subfig:modules}
		\includegraphics[width=0.4\linewidth]{modules_1.png}
	}\\
	\subfloat[][Assembling modules using \posl's operators]{%
		\label{subfig:as}
		\includegraphics[width=0.6\linewidth]{example_1.png}
	}\\
	\subfloat[][Connecting \posl{} solvers to create \comstrs]{%
		\label{subfig:conn}
		\includegraphics[width=0.6\linewidth]{conn.png}
	}
	\caption[]{Solver construction process using \posl}
	\label{fig:posl}
\end{figure}

%Once the solvers set is ready, the last step is to model the problem to solve. To do so, the user must follow the framework specification to implement the benchmark, respecting some requirements. The most important one is to implement a {\it cost function} computing the cost for a given configuration, i.e., an integer indicating how much the configuration violates the set of constraints. This integer equals zero if the configuration is a solution.

In the following sections all these steps are explained in details, but first, I explain how to model the target benchmark using \posl.

\section{Modeling the target benchmark}\label{sec:model}
%In this stage we explain formally our modeling process of a benchmark to be solved (or study) through \posl{}. We explain how to make use of the already existing models or to create new benchmarks using the basic layer of the framework in C++ making a proper usage of the object-oriented design.

Target problems are modeled using the low-level framework provided by \posl{} (written in C++ programming language) %, respecting some rules of the object-oriented design. 
They have to be coded respecting an object-oriented hierarchy designed for the optimum performance of the language. The most important functionalities that the proposed model have to provide are the following:

\underline{\textbf{Cost function}}. This function must compute the \textit{cost} of a given configuration. It must return an integer value taking into account the problem constraints. Given a configuration $s$, the \textit{cost function}, as a mandatory rule, must return 0 if and only if $s$ is a solution of the problem, i.e., $s$ fulfills all the problem constraints. Otherwise, it must return an integer describing "how long" is the given configuration from a solution. An example of \textit{cost function} is the one returning the number of violated constraints. However, the more expressive the cost function is, the better the performance of \posl{} is, leading to the solution.

\poslexample{
Let us take the example of the {\it 4-Queens Problem}. This problem is about placing 4 queens on a $4\times 4$ chess board so that none of them can hit any other in one move. A configuration for this benchmark is a vector of 4 integer indicating the row where a queens is placed on each column. So, the configuration $s_a = (1,3,1,2)$ corresponds to the example in Figure~\ref{subqueen:1}. 

Now, let us suppose two different \textit{cost functions}:

\begin{enumerate}
\item $f_1(s) = c$ if and only if $c$ is the maximum number of queens hitting another.
\item $f_2(s) = c$ if and only if $c$ is the sum of the number of queens that each queen hits.
\end{enumerate}

Tacking these two functions into account, it is easy to see that $f_1(s_a) = 3$ and $f_2(s_a) = 4$. If we take the example in Figure~\ref{subqueen:2}, the corresponding configuration is $s_b = (0,1,0,2)$ with $f_1(s_b) = 3$ and $f_2(s_b) = 6$. In this case, according to the \textit{cost function} $f_1$ both configurations have the same opportunity of being selected, because they have the same cost. However, applying the \textit{cost function} $f_2$, the best configuration is $s_b$ in which a solution can be obtained just moving the queen \textit{b3} to \textit{a3}.

In that sense, $f_2$ is \textit{more expressive} than $f_1$.
}

\begin{figure}[h]
	\centering
	\subfloat[][]{
		\label{subqueen:1}
		\includegraphics[width=0.25\linewidth]{queen1_1.png}
	}
	\hspace{0.05\textwidth}%
	\subfloat[][]{%
		\label{subqueen:2}
		\includegraphics[width=0.25\linewidth]{queen1_2.png}
	}
	\hspace{0.05\textwidth}%
	\subfloat[][]{%
		\label{subqueen:3}
		\includegraphics[width=0.25\linewidth]{queen1_3.png}
	}
	\caption[]{4-Queens examples}
	\label{fig:exqueens}
\end{figure}

\underline{\textbf{Relative cost function}}. This function must compute the \textit{cost} of a given configuration with respect to another, with the help of some stored information.

\poslexample{
Coming back to the previews example, let us suppose that the current configuration is $s_a = (1,3,1,2)$ corresponding to the Figure~\ref{subqueen:1}. Taking the \textit{cost function} $f_2$, the cost of this configurations is $f_2(s_a) = 4$. If we want to compute the cost of $s_c = (1, 3, 0, 2)$ (Figure~\ref{subqueen:3}), knowing that the only change with respect to the current configuration is the queen in the column $3$, we can use the following \textit{relative cost function}:
\begin{align*} 
rf(s_c) &=  c - 2\cdot q + a\\ 
&= 4 - 2\cdot 2 + 0 \\
&= 0
\end{align*}
where $c$ is the current cost, $q$ is the number of queens that the queen in column $3$ hits (an information that can be stored), and $a$ the number of queens that the queen in the column $3$ hits in the new position (\textit{a3}).
}

\underline{\textbf{Showing result function}}. This function represents the way a benchmark shows a configuration, in order to provide more information about the structure. 

%\poslexample{
For example, a configuration of the instance 3--3--2 of the \sgp{} (see bellow for more details about this benchmark) can be written as follows:


%\begin{BGVerbatim}
\begin{Verbatim}
[1, 2, 3, 4, 5, 6, 7, 8, 9, 3, 4, 5, 6, 7, 8, 9, 1, 2]
\end{Verbatim}
%\end{BGVerbatim}

This text is, nevertheless, very difficult to be read if the instance is larger. Therefore, it is recommended that the user implements this class in order to give more details and to make it easier to interpret the configuration. For example, for the same instance of the problem, a solution could be presented as follows:

%\begin{BGVerbatim}
\begin{Verbatim}
Golfers: players-3, groups-3, weeks-2
6	8	7	
1	3	5	
4	9	2	
--
7	2	3	
4	8	1	
5	6	9	
--
\end{Verbatim}
%\end{BGVerbatim}

Once we have modeled the target benchmark, it can be solved using \posl{}. In the following sections we describe how to use this parallel-oriented language to solve \CSPs.

\section{First stage: creating \posl's modules}\label{sec:1ststage}
There exist two types of basic modules in \posl: \INTROom{} and \INTROopch{}. A \om{} is basically a function and a \opch{} is also a function, but in contrast, it can receive information from two different sources: through input parameters or from outside, \ie by communicating with a module from another solver.

\subsection{Computation module}

%In this sub-section we expose the definition and the characteristics and the details of the \om, and give some examples. We explain how to create new \oms{} using the basic layer of the framework.

A \om{} is the most basic and abstract way to define a piece of computation. It is a function which receives an instance of a \posl{} data type as input, then executes an internal algorithm, and returns an instance of a \posl{} data type as output. The input and output types will characterize the computation module signature. It can be dy\-na\-mi\-cally replaced by (or combined with) other computation modules, since they can be transmitted to other solvers working in parallel. They are joined through operators defined in Section~\ref{sec:2ndstage}.

\defname{Computation Module}{\label{def:om}
A \om{} $Cm$ is a mapping defined by: 
\begin{equation}
\label{eq:om}
Cm:\mathcal{I} \rightarrow \mathcal{O}
\end{equation}
}

where $I$ and $O$, for instance, can be independently a set of configurations, a set of sets of configurations, a set of values of some data type, etc.

Consider a local search meta-heuristic solver. One of its \oms{} can be the function returning the set of configurations composing the neighborhood of a given configuration:

\begin{equation*}
Cm_{neighborhood}:I_1\times I_2\times\dots\times I_n \rightarrow 2^{I_1\times I_2\times\dots\times I_n}
\end{equation*}

\noindent where $I_i$ represents the definition domains of each variable of the input confi\-gura\-tion.

Figure~\ref{fig:om} shows an example of \om{} which receives a configuration $S$ and then computes the set $\mathcal{V}$ of its neighbor configurations $\left\{S^1, S^2, \dots, S^m\right\}$.

%\vspace{0.5cm}
\begin{figure}
	\centering	
	\includegraphics[width=0.7\linewidth]{OM.png}
	\caption{An example of a computation module computing a neighborhood}\label{fig:om}
\end{figure}

\subsubsection{Creating new \oms}
\label{subsubsec:creatingoms}

%To create new \oms{} we use C++. 
Each \om{} is written in C++. \posl{} provides a hierarchy of data types to work with %(see Appendix~\ref{app:diag}) 
and some abstract classes to inherit from, depending on the type of \om{} the user wants to create. These abstract classes represent {\it abstract} \oms{} and define a type of action to be executed. In the following we present the most important ones:

\begin{itemize}
\item \textbf{First Configuration Generation} $\rightarrow$ Represents \oms{} returning a configuration $s$, usually used for generating the starting configuration on local search meta-heuristics. 
\item \textbf{Neighborhood Function} $\rightarrow$ Represents \oms{} receiving a configuration $s$ as input and returning its neighborhood $\mathcal{V}(s)$ as output. These output configurations are efficiently stored in term of space.
\item \textbf{Selection Function} $\rightarrow$ Represents \oms{} receiving a neighborhood as input and selecting a configuration $s'$ from it as output. This function returns the pair $(s, s')$ containing both current and selected configuration.
\item \textbf{Decision Function} $\rightarrow$ Represents \oms{} receiving a couple af configurations encapsulated into a pair $(s, s')$, and returning the configuration to be the current one for the next iteration. 
\item \textbf{Processing Configuration Function} $\rightarrow$ Represents \oms{} receiving a configuration and returning another configuration as result of some arrangement, like for example, a reset. 
\end{itemize}

\subsection{Communication modules}

%In this sub-section we expose the definition and the characteristics and the details of the \opch, and give some examples. We explain how to create new \opchs{} using the basic layer of the framework.

A \opch{} is the component managing the information reception in the communication between solvers (we talk about information transmission in Section~\ref{sec:2ndstage}). They can interact with \oms{} through operators (see Figure~\ref{fig:och}).

A \opch{} can receive two types of information from an external solver: data or \oms{}. It is important to notice that by sending/receiving \oms, we mean sending/receiving the required information to identify and being able to instantiate the \om. For instance, an integer identifier.

In order to distinguish from the two types of \opchs, we will call \INTROdopch{} the \opch{} responsible for the data reception (Figure~\ref{subfig:doch}), and \INTROoopch{} is responsible for the reception and instantiation of \oms{} (Figure~\ref{subfig:ooch}).

\defname{Data Communication Module}{
A \emph{Data Communication Module} $Ch$ is a module that produces a mapping defined as follows: 
\begin{equation}
\label{def:dopench}
Ch:I\times \left\{D\cup \left\{NULL\right\}\right\} \rightarrow D \cup \left\{NULL\right\}
\end{equation}
No matter what the input $I$ is, it returns the information $D$ coming from an external solver.
}

\defname{Object Communication Module}{
	If we denote by $\mathbb{M}$ the space of all the \oms{} defined by Definition~\ref{def:om}, then an \emph{\oopch} $Ch$ is a module that produces and executes a \om{} coming from an external solver as follows:
	\begin{equation}
	\label{def:oopench}
	Ch:I\times\left\{\mathbb{M}\cup\left\{NULL\right\}\right\} \rightarrow O \cup \left\{NULL\right\}
	\end{equation}
It returns the output $O$ of the execution of the \om{} coming from an external solver, using $I$ as the input.
}%

\begin{figure}
	\centering
	\subfloat[][Data \opch]{
		\label{subfig:doch}
		\includegraphics[width=0.4\linewidth]{D_OCh_v2.png}
	}
	\hspace{0.05\textwidth}%
	\subfloat[][Object \opch]{%
		\label{subfig:ooch}
		\includegraphics[width=0.4\linewidth]{O_OCh_v2.png}
	}
	\caption[]{Communication module}
	\label{fig:och}
\end{figure}

Users can implement new computation and connection modules but \posl{} already contains many useful modules for solving a broad range of problems.

Due to the fact that \opchs{} receive information coming from outside without having control on them, it is necessary to define the {\it NULL} information, in order to denote the absence of information. If a Data Communication Module receives information, it is returned automatically. If an Object Communication Module receives a \om{}, it is instantiated and executed with the \opch's input and its result is returned. In both cases, if no available information exists (no communications performed), the \opch{} returns the {\it NULL} object.

\section{Second stage: assembling \posl's modules}\label{sec:2ndstage}
Modules mentioned above are grouped \modified{according its signature. An \textit{abstract module} is a module that represents all modules with the same signature.} For example, the module showed in Figure~\ref{fig:om} is a \om{} based on an abstract module that receives a configuration and returns a neighborhood. %In that sense, an example of a concrete \om{} (or just \om{}) can be a function receiving a configuration, and returning a neighborhood constituted by $N$ configurations which only differ from the input configuration in one entry.

In this stage an \INTROas{} is coded using \posl{}. It takes abstract modules as {\it parameters} and combines them through operators. Through the \as, we can also decide which information to send to other solvers. % by using some operators to send the result of a computation module (see below). In the following we present a formal and more detailed specification of \posl{}'s operators. 

The \as{} is the solver's backbone. It joins the \oms{} and the \opchs{} coherently. It is independent from the \oms{} and \opchs{} used in the solver. It means that modules can be changed or modified during the execution, respecting the algorithm structure. Each time we combine some of them using \posl's operators, we are creating a \INTROcm. Here we formally define the concept of \textit{module} and \INTROcm.

\begin{definition}
\label{def:module}
Denoted by the letter $\mathcal{M}$, a {\bf module} is:
\begin{enumerate}\renewcommand{\labelitemi}{\scriptsize$\blacksquare$}
\item a \om{}; or
\item a \opch{}; or
\item $\left[\text{OP } \mathcal{M}\right]$, which is the composition of a module $\mathcal{M}$ to be executed sequentially, returning an output depending on the nature of the unary operator \emph{OP}; or\label{subdef:seq_uni}
\item $\left[\mathcal{M}_1 \text{ OP } \mathcal{M}_2\right]$, which is the composition of two modules $\mathcal{M}_1$ and $\mathcal{M}_2$ to be executed sequentially, returning an output depending on the nature of the binary operator \emph{OP}; or\label{subdef:seq}
\item $\lbk\mathcal{M}_1 \text{ OP } \mathcal{M}_2\rbk_p$, which is the composition of two modules $\mathcal{M}_1$ and $\mathcal{M}_2$ to be executed, returning an output depending on the nature of the binary operator \emph{OP}. These two modules will be executed in parallel if and only if \emph{OP} supports parallelism, %(i.e. some modules will be executed sequentially although they were grouped this way); 
or it throws an exception otherwise.\label{subdef:par}
\end{enumerate}
I denote by $\mathbb{M}$ the space of the modules, and I call \cms{} to the composition of modules described in \ref{subdef:seq} and/or \ref{subdef:par}.
\end{definition}

For a better understanding of Definition~\ref{def:module}, Figure~\ref{fig:cm} shows graphically  the structure of a \cm.

\begin{figure}[h]
	\centering
	\includegraphics[width=0.5\linewidth]{cm.png}
	\caption[]{A \cm{} made of two modules $M_1$ and $M_2$}
	\label{fig:cm}
\end{figure}

As mentioned before, the \as{} is independent from the \oms{} and \opchs{} used in the solver. It means that one \as{} can be used to construct many different solvers, by implementing it using different modules. %(see below the related concept of \as{} instantiation). 
This is the reason why the \as{} is defined only using {abstract} modules. Formally, we define an \as{} as follows:

\defname{Abstract Solver}{
An \emph{Abstract Solver} $AS$ is a triple $\left(\mathbf{M},\mathcal{L}^m, \mathcal{L}^c\right)$, where: $\mathbf{M}$ is a \cm{} (also called \emph{root \cm{}}), $\mathcal{L}^m$ a list of abstract \oms{} appearing in $\mathcal{M}$, and $\mathcal{L}^c$ a list of \opchs{} appearing in $\mathcal{M}$.}

\Cms{}, and in particular the \textit{root} \cm{}, can be defined also as a context-free grammar as follows:

\begin{definition}\label{def:grammar} A {\bf \cm{}'s grammar} is the set $G_{POSL} = \left(\mathbf{V},\Sigma, \mathbf{S}, \mathbf{R}\right)$, where:
\begin{enumerate}\renewcommand{\labelitemi}{\scriptsize$\blacksquare$}
	\item $\mathbf{V} = \left\{CM, OP\right\}$ is the set of {\it variables},
	\item $\Sigma = \left\{\alpha, \beta, be, [, ], \lbk, \rbk_p, \llparenthesis, \rrparenthesis^m, \rrparenthesis^o, \circled{$\mapsto$},\circled{?},\circlearrowleft, \circled{$\rho$}, \circled{$\vee$}, \circled{$\wedge$}, \circled{M}, \circled{m}, \circled{$\shortdownarrow$}, \circled{$\cup$}, \circled{$\cap$}\right\}$ is the set of {\it terminals},
	\item $\mathbf{S} = \left\{CM\right\}$ is the set of {\it start variables},
	\item and $\mathbf{R} = $
		\begin{align*} 
		CM \produce & \alpha \OR \beta \OR \llparenthesis CM \rrparenthesis^o \OR \llparenthesis CM \rrparenthesis^m \OR \left[OP\right] \OR \lbk OP\rbk_p\\
		OP \produce & CM \circled{$\mapsto$} CM \OR CM \circled{?} CM \OR CM \circled{$\rho$} CM \OR CM \circled{$\vee$} CM \OR CM \circled{$\wedge$} CM  \OR\\
		& CM \circled{M} CM \OR CM \circled{m} CM \OR CM \circled{$\shortdownarrow$} CM \OR CM \circled{$\cup$} CM \OR CM \circled{$\cap$} CM  \OR\\
		& CM \circlearrowleft \text{ be } CM%\left\{CM\right\}		
		\end{align*} is a set of {\it rules}
\end{enumerate} 
%For simplicity, we will use, from now on, the name of \emph{Module} to refer ether to an \module, to an \opch, or to a \cm.
\end{definition}

In the following I explain some of the concepts in Definition~\ref{def:grammar}: 
\begin{itemize}
	\item The variables $CM$ and $OP$ correspond to a \cm{} and an {\it operator}, respectively.
	\item The terminals $\alpha$ and $\beta$ represent a \om{} and a \opch{}, respectively.
	\item The terminal $be$ is a boolean expression.
	\item The terminals $[\text{ } ], \lbk\text{ } \rbk_p$ are symbols for grouping and defining the way the involved \cms{} are executed. Depending on the nature of the operator, this can be either sequentially or in parallel:
	\begin{enumerate}\renewcommand{\labelitemi}{\scriptsize$\blacksquare$}
		\item $\left[\text{OP}\right]$: The involved operator will always executed sequentially.
		\item $\lbk\text{OP}\rbk_p$: The involved operator will be executed in parallel if and only if \emph{OP} supports parallelism. Otherwise, an exception is thrown.
	\end{enumerate}
	\item The terminals $\llparenthesis. \rrparenthesis^m, \llparenthesis.\rrparenthesis^o,$ are operators to send information to other solvers (explained bellow).
	\item All other terminals are \posl{} operators that are detailed later.
\end{itemize}

In the following we define \posl{} operators. In order to group modules, like in Definition~\ref{def:module}(\ref{subdef:seq}) and \ref{def:module}(\ref{subdef:par}), we will use $\left|OP\right|$ as generic grouper. \modified{In order to help the reader to easily understand how to use operators, I use an example of a solver that I build step by step, while presenting the definitions.}

%\begin{example}
%\mybox{Example}
\poslexample{
\posl{} creates solvers based on local search meta-heuristics algorithms. These algorithms have a common structure: \begin{inparaenum}[1.] \item They start by initializing some data structures (e.g., a \emph{tabu list} for \emph{Tabu Search}, a \emph{temperature} for \emph{Simulated Annealing}, etc.). \item An initial configuration $s$ is generated. \item A new configuration $s'$ is selected from the neighborhood $\mathcal{V}\left(s\right)$. \item If $s'$ is a solution for the problem $P$, then the process stops, and $s'$ is returned. If not, the data structures are updated, and $s'$ is accepted or not for the next iteration, depending on a certain criterion. \end{inparaenum}
%An example of such data structure is the penalizing features of local optima in \emph{Guided Local Search} algorithm.

Abstract \oms{} composing local search meta--heuristics are:

\begin{list}{\boxed{Abstract\hspace{4pt}computation\hspace{4pt}module- \arabic{qcounter}~}}{\usecounter{qcounter}} \itemsep0em 
	\item $I$: Generating a configuration $s$
	\item $V$: Defining the neighborhood $\mathcal{V}\left(s\right)$
	\item $S$: Selecting $s' \in \mathcal{V}\left(s\right)$
	\item $A$: Evaluating an acceptance criterion for $s'$
\end{list}
} %\end{example}

The list of modules to be used in the examples have been presented. Now I present \posl{} operators.

\separation

\begin{definition}\label{op:seqexec}
$\circled{$\mapsto$}$ {\bf Sequential Execution Operator.} Let
\begin{enumerate}%\begin{inparaenum}[i)]
	\item $\mathcal{M}_1 : \mathcal{D}_1 \rightarrow \mathcal{I}_1$ and 
	\item $\mathcal{M}_2 : \mathcal{D}_2 \rightarrow \mathcal{I}_2$, 
\end{enumerate}%\end{inparaenum} 
be modules, where $\mathcal{I}_1 \subseteq \mathcal{D}_2$. Then the operation $\left|\mathcal{M}_1\circled{$\mapsto$} \mathcal{M}_2\right|$ defines the \cm{} $\mathcal{M}_{seq}$ as the result of executing $\mathcal{M}_1$ followed by executing $\mathcal{M}_2$:

\[
\mathcal{M}_{seq}:\mathcal{D}_1 \rightarrow \mathcal{I}_2
\]
\end{definition}

This is an example of an operator that does not support the execution of its involved \cms{} in parallel, because the input of the second \cm{} is the output of the first one.

\poslexample{
Coming back to the example, I can use defined abstract \oms{} to create a \cm{} that perform only one iteration of a local search, using the {\bf Sequential Execution} operator. I create a \cm{} to execute sequentially $I$ and $V$ (see Figure~\ref{subfig:ex_seq1}), then I create an other \cm{} to execute sequentially the \cm{} already created and $S$ (see Figure~\ref{subfig:ex_seq2}), and finally this \cm{} and the \om{} $A$ are executed sequentially (see Figure~\ref{subfig:ex_seq3}). The \cm{} presented in Figure~\ref{subfig:ex_seq3} can be coded as follows:
$$\left[\left[\left[I\poslop{\mapsto}V\right]\poslop{\mapsto}S\right]\poslop{\mapsto}A\right]$$
In the figure, each rectangle is a \cm.
}

\begin{figure}[h]
	\centering
	\subfloat[][]{
		\label{subfig:ex_seq1}
		\includegraphics[width=0.2\linewidth]{seq_1.png}
	}\hspace{0.05\linewidth}
	\subfloat[][]{%
		\label{subfig:ex_seq2}
		\includegraphics[width=0.3\linewidth]{seq_2.png}
	}\\
	\subfloat[][]{%
		\label{subfig:ex_seq3}
		\includegraphics[width=0.4\linewidth]{seq_3.png}
	}
	\caption[]{Using {\bf sequential execution} operator}
	\label{fig:seq_example}
\end{figure}

\separation

The following operator is very useful to execute modules sequentially creating bifurcations, subject to some boolean condition:

\begin{definition}\label{op:conditional}
$\circled{?}$ {\bf Conditional Execution Operator} Let
\begin{enumerate}%\begin{inparaenum}[i)]
	\item $\mathcal{M}_1 : \mathcal{D}_1 \rightarrow \mathcal{I}_1$ and  
	\item $\mathcal{M}_2 : \mathcal{D}_2 \rightarrow \mathcal{I}_2$,
\end{enumerate}%\end{inparaenum} 
be modules, where $\mathcal{D}_1 \cap \mathcal{D}_2 \neq \emptyset$. %and $\mathcal{I}_1 \subset \mathcal{I}_2$. 
Then the operation $\left|\mathcal{M}_1\circled{?}_{<cond>}\mathcal{M}_2\right|$ defines the \cm{} $\mathcal{M}_{cond}$ as result of the sequential execution of $\mathcal{M}_1$ if $<cond>$ is {\bf true} or $\mathcal{M}_2$, otherwise:

\[
\mathcal{M}_{cond}:\mathcal{D}_1\cap\mathcal{D}_2 \rightarrow \mathcal{I}_1 \cup \mathcal{I}_2 
\]
\end{definition}

\poslexample{
This operator can be used in the example if I want to execute two different {\it selection} \oms{} ($S_1$ and $S_2$) depending on certain criterion (see Figure~\ref{fig:cond_example}):
$$\left[\left[\left[I\poslop{\mapsto}V\right]\poslop{\mapsto}\left[S_1\poslop{?}S_2\right]\right]\poslop{\mapsto}A\right]$$
In examples I remove the clause $<cond>$ for simplification.
}

\begin{figure}[h]
	\centering	
	\includegraphics[width=0.5\linewidth]{cond.png}
	\caption{Using {\bf conditional execution} operator}\label{fig:cond_example}
\end{figure}

\separation

We can execute modules sequentially creating also cycles.

\begin{definition}\label{op:cyclic}
$\circlearrowleft$ {\bf Operator Cyclic Execution Operator} Let $\mathcal{M} : \mathcal{D} \rightarrow \mathcal{I}$ be a module, where $\mathcal{I} \subseteq \mathcal{D}$. Then, the operation $\left|\circlearrowleft_{<cond>}\mathcal{M}\right|$ defines the \cm{} $\mathcal{M}_{cyc}$ repeating sequentially the execution of $\mathcal{M}$ while $<cond>$ remains {\bf true}:

\[
\mathcal{M}_{cyc}:\mathcal{D} \rightarrow \mathcal{I} 
\]
\end{definition}

\poslexample{
Using this operator I can model a local search algorithm, by executing the {\it abstract} \om{} $I$ and then the other \oms{} ($V$, $S$ and $A$) cyclically, until finding a solution (i.e, a configuration with cost equals to zero) (see Figure~\ref{fig:cyc_example}):
$$\left[I\poslop{\mapsto}\left[\circlearrowleft\left[\left[V\poslop{\mapsto}S\right]\poslop{\mapsto}A\right]\right]\right]$$
In the examples, I remove the clause $<cond>$ for simplification. 
}

\begin{figure}[h]
	\centering	
	\includegraphics[width=0.5\linewidth]{cyc.png}
	\caption{Using {\bf cyclic execution} operator}\label{fig:cyc_example}
\end{figure}

\separation

\begin{definition}\label{op:rho}
{\bf (Operator Random Choice)} Let
\begin{enumerate}%\begin{inparaenum}[i)]
	\item $\mathcal{M}_1 : \mathcal{D}_1 \rightarrow \mathcal{I}_1$ and  
	\item $\mathcal{M}_2 : \mathcal{D}_2 \rightarrow \mathcal{I}_2$,
\end{enumerate}%\end{inparaenum} 
be modules, where $\mathcal{D}_1 \subset \mathcal{D}_2$, % and $\mathcal{I}_1 \subset \mathcal{I}_2$, 
and a real value $\rho \in (0,1)$. Then the operation $\left|M_1\circled{$\rho$}\mathcal{M}_2\right|$ defines the \cm{} $\mathcal{M}_{rho}$ executing $\mathcal{M}_1$ with probability $\rho$, or executing $\mathcal{M}_2$ with probability $(1-\rho)$:

\[
\mathcal{M}_{rho}:\mathcal{D}_1\cap\mathcal{D}_2 \rightarrow \mathcal{I}_1 \cup \mathcal{I}_2 
\]
\end{definition}

\poslexample{
In the example I can create a \cm{} to execute two {\it abstract} \oms{} $A_1$ and $A_2$ following certain probability $\rho$ using the {\bf random choice} operator as follows (see Figure~\ref{fig:rho_example}):
$$\left[I\poslop{\mapsto}\left[\circlearrowleft\left[\left[V\poslop{\mapsto}S\right]\poslop{\mapsto}\left[A_1\poslop{\rho}A_2\right]\right]\right]\right]$$
}

\begin{figure}[h]
	\centering	
	\includegraphics[width=0.6\linewidth]{rho.png}
	\caption{Using {\bf random choice} operator}\label{fig:rho_example}
\end{figure}

\separation

The following operator is very useful if the user needs to use a \opch{} inside an \as{}. As explained before, if a \opch{} does not receive any information from another solver, it returns {\it NULL}. This may cause the undesired termination of the solver if this case is not considered correctly. Next, I introduce the operator \textbf{Operator Not {\it NULL} Execution} and illustrate how to use it in practice with an example.

\begin{definition}\label{op:or}
{\bf (Operator Not {\it NULL} Execution)} Let
\begin{enumerate}%\begin{inparaenum}[i)]
	\item $\mathcal{M}_1 : \mathcal{D}_1 \rightarrow \mathcal{I}_1$ and  
	\item $\mathcal{M}_2 : \mathcal{D}_2 \rightarrow \mathcal{I}_2$,
\end{enumerate}%\end{inparaenum} 
be modules, where $\mathcal{D}_1 \subseteq \mathcal{D}_2$. % and $\mathcal{I}_1 \subset \mathcal{I}_2$. 
Then, the operation $\left|\mathcal{M}_1\circled{$\vee$}\mathcal{M}_2\right|$ defines the \cm{} $\mathcal{M}_{non}$ that executes $\mathcal{M}_1$ and returns its output if it is not {\it NULL}, or executes $\mathcal{M}_2$ and returns its output otherwise:

\[
\mathcal{M}_{non}:\mathcal{D}_1\cap\mathcal{D}_2 \rightarrow \mathcal{I}_1 \cup \mathcal{I}_2 
\]
\end{definition}

\poslexample{Let us make consider a slightly more complex example: When applying the acceptance criterion, suppose that we want to receive a configuration from other solver to combine the \om{} $A$ with a \opch:

\begin{list}{\boxed{Communication\hspace{4pt}module- \arabic{qcounter}:~}}{\usecounter{qcounter}} \itemsep0em
	\item $C.M.$: Receiving a configuration.\label{struct:opch}
\end{list}

Figure~\ref{fig:2difBeh} shows how to combine a \opch{} with the \om{} $A$ through the operator $\circled{$\vee$}$. Here, the \om{} $A$ will be executed as long as the \opch{} remains \textit{NULL}, i.e., there is no information coming from outside. This behavior is represented in Figure~\ref{subfig:beh1} by the orange lines. If some data has been received through the \opch, the later is executed instead of the module $A$, represented in Figure~\ref{subfig:beh2} by blue lines. 
The code can be written as follows:
$$\left[I\poslop{\mapsto}\left[\circlearrowleft\left[\left[V\poslop{\mapsto}S\right]\poslop{\mapsto}\left[A\poslop{\vee}C.M.\right]\right]\right]\right]$$
}

\begin{figure}[h]
\centering
\subfloat[][The solver executes the computation module {\bf A} if no information is received through the connection module]{
	\label{subfig:beh1}
	\includegraphics[width=0.6\linewidth]{muta1_v4.png}
}\\
%\hspace{0.05\textwidth}%
\subfloat[][The solver uses the information coming from an external solver]{%
	\label{subfig:beh2}
	\includegraphics[width=0.6\linewidth]{muta2_v4.png}
}
\caption[]{Two different behaviors within the same solver}
\label{fig:2difBeh}
\end{figure}

This is {\it short-circuit} operator. It means that if the first argument (module) does not return {\it NULL}, the second will not be executed. \posl{} provides another operator with the same functionality but not {\it short-circuit}:

\begin{definition}\label{op:and}
{\bf (Operator {\it BOTH} Execution)} Let 
\begin{enumerate}%\begin{inparaenum}[i)]
	\item $\mathcal{M}_1 : \mathcal{D}_1 \rightarrow \mathcal{I}_1$ and  
	\item $\mathcal{M}_2 : \mathcal{D}_2 \rightarrow \mathcal{I}_2$,
\end{enumerate}%\end{inparaenum} 
be modules, where $\mathcal{D}_1 \subseteq \mathcal{D}_2$. % and $\mathcal{I}_1 \subset \mathcal{I}_2$. 
Then the operation $\left|\mathcal{M}_1\circled{$\wedge$}\mathcal{M}_2\right|$ defines the \cm{} $\mathcal{M}_{both}$ that executes both $\mathcal{M}_1$ and $\mathcal{M}_2$, then returns the output of $\mathcal{M}_1$ if it is not {\it NULL}, or the output of $\mathcal{M}_2$ otherwise:

\[
\mathcal{M}_{both}:\mathcal{D}_1\cap\mathcal{D}_2 \rightarrow \mathcal{I}_1 \cup \mathcal{I}_2 
\]
\end{definition}

In the following definitions, the concepts of {\it cooperative parallelism} and {\it competitive parallelism} are implicitly included. We say that cooperative parallelism exists when two or more processes are running separately, they are independent, and the general result will be some combination of the results of all the involved processes (e.g. Definitions~\ref{op:min} and~\ref{op:max}). \modified{On the other hand, competitive parallelism arise when the general result is the result of the process ending first (e.g. Definition~\ref{op:race}).}

\begin{definition}\label{op:min}
{\bf (Operator Minimum)} Let
\begin{enumerate}%\begin{inparaenum}[i)]
	\item $\mathcal{M}_1 : \mathcal{D}_1 \rightarrow \mathcal{I}_1$ and  
	\item $\mathcal{M}_2 : \mathcal{D}_2 \rightarrow \mathcal{I}_2$,
\end{enumerate}%\end{inparaenum} 
be modules, where $\mathcal{D}_1 \subseteq \mathcal{D}_2$. %and $\mathcal{I}_1 \subset \mathcal{I}_2$. 
Let also $o_1$ and $o_2$ be the outputs of $\mathcal{M}_1$ and $\mathcal{M}_2$, respectively. Assume that there exists some order criteria between them. Then the operation $\left|\mathcal{M}_1\circled{m}\mathcal{M}_2\right|$ defines the \cm{} $\mathcal{M}_{min}$ that executes $\mathcal{M}_1$ and returns $\min\left\{o_1,o_2\right\}$:

\[
\mathcal{M}_{min}:\mathcal{D}_1\cap\mathcal{D}_2 \rightarrow \mathcal{I}_1 \cup \mathcal{I}_2 
\]
\end{definition}

Similarly we define the operator \textbf{Maximum}:

\begin{definition}\label{op:max}
{\bf (Operator Maximum)} Let
\begin{enumerate}%\begin{inparaenum}[i)]
	\item $\mathcal{M}_1 : \mathcal{D}_1 \rightarrow \mathcal{I}_1$ and  
	\item $\mathcal{M}_2 : \mathcal{D}_2 \rightarrow \mathcal{I}_2$,
\end{enumerate}%\end{inparaenum} 
be modules, where $\mathcal{D}_1 \subseteq \mathcal{D}_2$. %and $\mathcal{I}_1 \subset \mathcal{I}_2$. 
Let also $o_1$ and $o_2$ be the outputs of $\mathcal{M}_1$ and $\mathcal{M}_2$, respectively. Assume that there exists some order criteria between them. Then the operation $\left|\mathcal{M}_1\circled{M}\mathcal{M}_2\right|$ defines the \cm{} $\mathcal{M}_{max}$ that executes $\mathcal{M}_1$ and returns $\max\left\{o_1,o_2\right\}$:

\[
\mathcal{M}_{max}:\mathcal{D}_1\cap\mathcal{D}_2 \rightarrow \mathcal{I}_1 \cup \mathcal{I}_2 
\]
\end{definition}

\poslexample{Comming back to the previews example, the {\bf minimum operator} can be applied to obtain a more interesting behavior in the solver: When applying the acceptance criteria, suppose that we want to receive a configuration from other solver, to compare it with ours and select the one with the lowest cost. We can do that by applying the operator~$\circled{m}$ to combine the \om{} $A$ with a \opch{} $C.M.$ (see Figure\ref{fig:min_example}):
$$\left[I\poslop{\mapsto}\left[\circlearrowleft\left[\left[V\poslop{\mapsto}S\right]\poslop{\mapsto}\lbk A\poslop{m}C.M.\rbk_p\right]\right]\right]$$
Notice that in this example, I can use the grouper $\lbk .\rbk_p$ since the {\bf minimum operator} supports parallelism.}

\begin{figure}[h]
	\centering	
	\includegraphics[width=0.6\linewidth]{min.png}
	\caption{Using {\bf minimum} operator}\label{fig:min_example}
\end{figure}

\begin{definition}\label{op:race}
{\bf (Operator Race)} Let 
\begin{enumerate}%\begin{inparaenum}[i)]
	\item $\mathcal{M}_1 : \mathcal{D}_1 \rightarrow \mathcal{I}_1$ and  
	\item $\mathcal{M}_2 : \mathcal{D}_2 \rightarrow \mathcal{I}_2$,
\end{enumerate}%\end{inparaenum} 
be modules, where $\mathcal{D}_1 \subseteq \mathcal{D}_2$ and $\mathcal{I}_1 \subset \mathcal{I}_2$. Then the operation $\left|\mathcal{M}_1\circled{$\shortdownarrow$}\mathcal{M}_2\right|$ defines the \cm{} $\mathcal{M}_{race}$ that executes both modules $\mathcal{M}_1$ and $\mathcal{M}_2$, and returns the output of the module ending first:

\[
\mathcal{M}_{race}:\mathcal{D}_1\cap\mathcal{D}_2 \rightarrow \mathcal{I}_1 \cup \mathcal{I}_2 
\]
\end{definition}

\poslexample{Sometimes nighborhood functions are slow depending on the configuration. In that case two neighborhood \oms{} can be executed and take into account the output of the module ending first (see Figure\ref{fig:race_example}):
$$\left[I\poslop{\mapsto}\left[\circlearrowleft\left[\left[\lbk V_1\poslop{\shortdownarrow}V_2\rbk_p\poslop{\mapsto}S\right]\poslop{\mapsto}\lbk A\poslop{m}C.M.\rbk_p\right]\right]\right]$$}

\begin{figure}[h]
	\centering	
	\includegraphics[width=0.7\linewidth]{race.png}
	\caption{Using {\bf race} operator}\label{fig:race_example}
\end{figure}

Some others operators can be useful when dealing with {\it sets}.

\begin{definition}\label{op:union}
{\bf (Operator Union)} Let 
\begin{enumerate}%\begin{inparaenum}[i)]
	\item $\mathcal{M}_1 : \mathcal{D}_1 \rightarrow \mathcal{I}_1$ and  
	\item $\mathcal{M}_2 : \mathcal{D}_2 \rightarrow \mathcal{I}_2$,
\end{enumerate}%\end{inparaenum} 
be modules, where $\mathcal{D}_1 \subseteq \mathcal{D}_2$. %and $\mathcal{I}_1 \subset \mathcal{I}_2$. 
Let also $V_1$ and $V_2$ be the outputs of $\mathcal{M}_1$ and $\mathcal{M}_2$, respectively. Then the operation $\left|\mathcal{M}_1\circled{$\cup$}\mathcal{M}_2\right|$ defines the \cm{} $\mathcal{M}_{\cup}$ that executes both modules $\mathcal{M}_1$ and $\mathcal{M}_2$, and returns $V_1\cup V_2$:

\[
\mathcal{M}_{\cup}:\mathcal{D}_1\cap\mathcal{D}_2 \rightarrow \mathcal{I}_1 \cup \mathcal{I}_2
\]
\end{definition}

Similarly we define the operators \textbf{Intersection} and \textbf{Subtraction}:

\begin{definition}\label{op:intersec}
{\bf (Operator Intersection)} Let 
\begin{enumerate}%\begin{inparaenum}[i)]
	\item $\mathcal{M}_1 : \mathcal{D}_1 \rightarrow \mathcal{I}_1$ and  
	\item $\mathcal{M}_2 : \mathcal{D}_2 \rightarrow \mathcal{I}_2$,
\end{enumerate}%\end{inparaenum} 
be modules, where $\mathcal{D}_1 \subseteq \mathcal{D}_2$. % and $\mathcal{I}_1 \subset \mathcal{I}_2$. 
Let also $V_1$ and $V_2$ be the outputs of $\mathcal{M}_1$ and $\mathcal{M}_2$, respectively. Then the operation $\left|\mathcal{M}_1\circled{$\cap$}\mathcal{M}_2\right|$ defines the \cm{} $\mathcal{M}_{\cap}$ that executes both modules $\mathcal{M}_1$ and $\mathcal{M}_2$, and returns $V_1\cap V_2$:

\[
\mathcal{M}_{\cap}:\mathcal{D}_1\cap\mathcal{D}_2 \rightarrow \mathcal{I}_1 \cup \mathcal{I}_2
\]
\end{definition}

\begin{definition}\label{op:subst}
{\bf (Operator Subtraction)} Let 
\begin{enumerate}%\begin{inparaenum}[i)]
	\item $\mathcal{M}_1 : \mathcal{D}_1 \rightarrow \mathcal{I}_1$ and  
	\item $\mathcal{M}_2 : \mathcal{D}_2 \rightarrow \mathcal{I}_2$,
\end{enumerate}%\end{inparaenum} 
be modules, where $\mathcal{D}_1 \subseteq \mathcal{D}_2$. % and $\mathcal{I}_1 \subset \mathcal{I}_2$. 
Let also $V_1$ and $V_2$ be the outputs of $\mathcal{M}_1$ and $\mathcal{M}_2$, respectively. Then the operation $\left|\mathcal{M}_1\circled{-}\mathcal{M}_2\right|$ defines the \cm{} $\mathcal{M}_{-}$ that executes both modules $\mathcal{M}_1$ and $\mathcal{M}_2$, and returns $V_1 - V_2$:

\[
\mathcal{M}_{-}:\mathcal{D}_1\cap\mathcal{D}_2 \rightarrow \mathcal{I}_1 \cup \mathcal{I}_2
\]
\end{definition}

Now, I define the operators which allows to send information to other solvers. Two types of information can be sent: 
\begin{inparaenum}[i)]
	\item the output of the \om{} and send its output, or 
	\item the \om{} itself.
\end{inparaenum}. This utility is very useful in terms of sharing behaviors between solvers.

\begin{definition}\label{op:osend}
{\bf (Sending Data Operator)} Let $\mathcal{M} : \mathcal{D} \rightarrow \mathcal{I}$ be a module. Then the operation $\left|\llparenthesis \mathcal{M}\rrparenthesis^{o}\right|$ defines the \cm{} $\mathcal{M}_{sendD}$ that executes the module $\mathcal{M}$ and sends its output outside:

\[
\mathcal{M}_{sendD}:\mathcal{D} \rightarrow \mathcal{I}
\]
\end{definition}

Similarly we define the operator \textbf{Send Module}:

\begin{definition}\label{op:msend}
{\bf (Sending Module Operator)} Let $\mathcal{M} : \mathcal{D} \rightarrow \mathcal{I}$ be a module. Then the operation $\left|\llparenthesis \mathcal{M}\rrparenthesis^{m}\right|$ defines the \cm{} $\mathcal{M}_{sendM}$ that executes the module $\mathcal{M}$, then returns its output and sends the module itself outside:

\[
\mathcal{M}_{sendM}:\mathcal{D} \rightarrow \mathcal{I}
\]
\end{definition}

\poslexample{In the following example, I use one of the \cms{} already presented in the previews examples, using a \opch{} to receive a configuration (see Figure~\ref{subfig:receiver_example}):  
$$\left[I\poslop{\mapsto}\left[\circlearrowleft\left[\left[V\poslop{\mapsto}S\right]\poslop{\mapsto}\lbk A\poslop{m}C.M.\rbk_p\right]\right]\right]$$

I also build another, as its complement: sending the accepted configuration to outside, using the {\bf sending data operator} (see Figure\ref{subfig:sender_example}):
$$\left[I\poslop{\mapsto}\left[\circlearrowleft\left[\left[V\poslop{\mapsto}S\right]\poslop{\mapsto}\llparenthesis A\rrparenthesis^{o}\right]\right]\right]$$

In the Section~\ref{sec:4thstage} I explain how to connect solvers to each other.
}

\begin{figure}[h]
\centering
\subfloat[][]{
	\label{subfig:receiver_example}
	\includegraphics[width=0.6\linewidth]{min.png}
}\\
\subfloat[][]{%
	\label{subfig:sender_example}
	\includegraphics[width=0.6\linewidth]{send.png}
}
\caption[]{Sender and receiver behaviors}
\label{fig:send_recv}
\end{figure}

Once all desired abstract modules are linked together with operators, we obtain the {\it root} \cm{}, an important part of an \as. To implement a concrete solver from an \as, one must instantiate each abstract module with a concrete one respecting the required signature. From the same \as, one can implement many different concrete solvers simply by instantiating abstract modules with different concrete modules.

An \as{} is defined as follows: after declaring the \mbox{\tet{\bf abstract solver}}'s name, the first line defines the list of abstract \oms, the second one the list of abstract \opchs, then the algorithm of the solver is defined as the solver's body (the root \cm), between \mbox{\tet{\bf begin}} and \mbox{\tet{\bf end}}.

An \as{} can be declared through the  simple regular expression:

\begin{center}
\tet{\bf abstract solver} {\it name} \tet{\bf computation}: $L^m$ (\tet{\bf communication}: $L^c$)? \tet{\bf begin} $\mathcal{M}$ \tet{\bf end}
\end{center}

where:
\begin{itemize}
\item {\it name} is the identifier of the \as{}, 
\item $L^m$ is the list of abstract \oms{},
\item $L^c$ is the list of abstract \opchs{}, and
\item $\mathcal{M}$ is the root \cm.
\end{itemize}

For instance, Algorithm~\ref{algo:as_example} illustrates the abstract solver corresponding to Figure~\ref{subfig:as}.

\begin{algorithm}[H]
\dontprintsemicolon
\SetNoline
\SetKwProg{myproc}{}{}{}
\myproc{\tet{\bf abstract solver} as\_01\;
\tet{\bf computation} : $I, V, S, A$ \; 
\tet{\bf connection}: $C.M.$}{
	\Begin{
		%\While{$($\Iter $< K_1)$}
		%{
			$I$ \sec \; %\circled{$\mapsto$}$\;
			\While{$($\Iter \% $K_1)$}{
				$\left[V\poslop{\mapsto}S\poslop{\mapsto}\left[C.M.\poslop{m} \llparenthesis A\rrparenthesis^o\right]\right]$\;
			}
		%}
	}
}
\caption{\posl{} pseudo-code for the \as{} presented in Figure~\ref{subfig:as}}\label{algo:as_example}
\end{algorithm}	

\section{Third stage: creating \posl{} solvers}\label{sec:3rdstage}
%With operation modules and open channels already assembled through the \as, we can create solvers by instantiating modules. \posl{} provides an environment to this end and we present the procedure to use it.

%With \module s, \opch s and \cstr{} defined, we can create solvers by instantiating the declared components. \af{} provides an environment to this end, presented in Algorithm~\ref{algo:solver_def}, where $m_i$ and $ch_i$ represent the instances of the \module s and the instances of the \opch s to be passed by parameters to the \cstr{} $St$.

With \bothmodules{} composing an \as, one can create solvers by instantiating \ms. This is simply done by specifying that a given \mbox{\tet{\bf solver}} must \mbox{\tet{\bf implements}} a given \as, followed by the list of \omprefix{} then \opchs{}. These modules must match signatures required by the \as. Algorithm~\ref{algo:solver_def} implements Algorithm~\ref{algo:as_example} by instantiating modules shown in Figure~\ref{fig:2difBeh}.

\begin{algorithm}[H]
\dontprintsemicolon
\SetNoline
\SetKwProg{myproc}{}{}{}
%\myproc{
\tet{\bf solver} solver\_01 \tet{\bf implements} as\_01\;
\tet{\bf computation} : $I_{rand}, V_{std}, S_{best}, A_{alw}$ \; 
\tet{\bf connection}: $CM_{last}$\; %}{
%	\Begin{
%	}
%}
\caption{An instantiation of the \as{} presented in Algorithm~\ref{algo:as_example}}\label{algo:solver_def}
\end{algorithm}

Algorithm~\ref{algo:solver_def} is just an example of a solver instantiation, using some \oms{} provided by \posl{}, that are used and explained in details in the Chapter~\ref{chap:expe} of this document:
\begin{itemize}
\item $I_{rand}$ creates a random configuration.
\item $V_{std}$ creates a neighborhood of a given configuration, changing one element at a time.
\item $S_{best}$ selects the configuration of a neighborhood with the lowest cost.
\item $A_{alw}$ always accepts the incoming configuration.
\item $CM_{last}$ returns the last configuration arrived, if at the time of its execution, there is more than one configuration waiting to be received. 
\end{itemize}

\section{Forth stage: connecting solvers}\label{sec:4thstage}
We call \soset{} the pool of (concrete) solvers we plan to use in parallel to solve a problem. Once we have our solvers set, the last stage is to connect the solvers each others. Up to here, solvers are disconnected, but they have everything to establish the communication. \posl{} provides to the user a platform to easily define cooperative strategies that solvers must follow.

Following, we present two important concepts before we can formalize the {\it communication operators}.

\begin{definition}\label{def:comm_jack}
{\bf (Communication Jack)} Let a solver $\mathcal{S}$ be. Then, the operation $\mathcal{S}\cdot\mathcal{M}$ opens an outgoing connection from the solver $\mathcal{S}$ sending to the outside either 
\begin{inparaenum}[a)]
	\item the output of $\mathcal{M}$ if it is affected by a {\it sending data operator} presented in Definition~\ref{op:osend}, or
	\item $\mathcal{M}$ itself, if it is affected by a {\it sending module operator} presented in Definition~\ref{op:msend}.
\end{inparaenum}
\end{definition} 

\begin{definition}\label{def:comm_outlet}
{\bf (Communication Outlet)} Let a solver $\mathcal{S}$ be. Then, the operation $\mathcal{S}\cdot\mathcal{CM}$ opens an ingoing connection to the solver $\mathcal{S}$ receiving from the outside either 
\begin{inparaenum}[a)]
	\item the output of some \om{} if $\mathcal{CM}$ is a {\it data} \opch{}, or
	\item a \om{} if $\mathcal{CM}$ is an {\it object} \opch.
\end{inparaenum}
\end{definition} 


The communication is established by following the next rules guideline:
\begin{enumerate}%\begin{inparaenum}
	\item Each time a solver sends any kind of information by using a {\it sending} operator, it creates a \jack.
	\item Each time a solver defines a \opch, it creates a \outlet. 
	\item Solvers can be connected each others by linking \jacks{} to \outlets.
\end{enumerate} %\end{inparaenum}

%With the operator $(\cdot)$ we can have access to \oms{} sending information and to the \opch's names in a solver. 
%For example: $Solver_0\cdot \mathcal{M}$ provides access to the \om{} $\mathcal{M}$ in $Solver_0$ if and only if it is affected by a {\it sending} operator, and $Solver_1\cdot CM$ provides access to the \opch{} $CM$ in $Solver_1$.

Following, we define the \textit{connection operators} that \posl{} provides.

\begin{definition}\label{op_conn:1to1}
{\bf (Connection Operator One-to-One)} Let 
\begin{enumerate}
\item the list $\mathcal{J} = \left[\mathcal{S}_0\cdot \mathcal{M}_0, \mathcal{S}_1\cdot \mathcal{M}_1, ..., \mathcal{S}_{N-1}\cdot \mathcal{M}_{N-1}\right]$ of \jacks, and
\item the list $\mathcal{O} = \left[\mathcal{Z}_0\cdot \mathcal{CM}_0, \mathcal{Z}_1\cdot \mathcal{CM}_1, ..., \mathcal{Z}_{N-1}\cdot \mathcal{CM}_{N-1}\right]$ of \outlets{}
\end{enumerate} be. Then, the operation 
\[
\mathcal{J} \poslop{\rightarrow} \mathcal{O}
\]
Connects each \jack{} $\mathcal{S}_i\cdot \mathcal{M}_i \in \mathcal{J}$ with the corresponding \outlet{} $\mathcal{Z}_i\cdot \mathcal{CM}_i \in \mathcal{O}$, $\forall 0 \leq i \leq N-1$.
\end{definition}

\begin{definition}\label{op_conn:1ton}
{\bf (Connection Operator One-to-N)} Let 
\begin{enumerate} 
\item the list $\mathcal{J} = \left[\mathcal{S}_0\cdot \mathcal{M}_0, \mathcal{S}_1\cdot \mathcal{M}_1, ..., \mathcal{S}_{N-1}\cdot \mathcal{M}_{N-1}\right]$ of \jacks, and 
\item the list $\mathcal{O} = \left[\mathcal{Z}_0\cdot \mathcal{CM}_0, \mathcal{Z}_1\cdot \mathcal{CM}_1, ..., \mathcal{Z}_{M-1}\cdot \mathcal{CM}_{M-1}\right]$ of \outlets{} 
\end{enumerate} be. Then, the operation 
\[
\mathcal{J} \poslop{\rightsquigarrow} \mathcal{O}
\]
Connects each \jack{} $\mathcal{S}_i\cdot \mathcal{M}_i \in \mathcal{J}$ with every \outlet{} $\mathcal{Z}_j\cdot \mathcal{CM}_j \in \mathcal{O}$, $\forall 0 \leq i \leq N-1$ and $0 \leq j \leq M-1$.
\end{definition}

\begin{definition}\label{op_conn:ring}
{\bf (Connection Operator Ring)} Let 
\begin{enumerate} 
\item the list $\mathcal{J} = \left[\mathcal{S}_0\cdot \mathcal{M}_0, \mathcal{S}_1\cdot \mathcal{M}_1, ..., \mathcal{S}_{N-1}\cdot \mathcal{M}_{N-1}\right]$ of \jacks, and 
\item the list $\mathcal{O} = \left[\mathcal{S}_0\cdot \mathcal{CM}_0, \mathcal{S}_1\cdot \mathcal{CM}_1, ..., \mathcal{S}_{N-1}\cdot \mathcal{CM}_{N-1}\right]$ of \outlets{} 
\end{enumerate} be. Then, the operation 
\[
\mathcal{J} \poslop{\leftrightarrow} \mathcal{O}
\]
Connects each \jack{} $\mathcal{S}_i\cdot \mathcal{M}_i \in \mathcal{J}$ with the corresponding \outlet{} $\mathcal{Z}_{(i+1)\%N}\cdot \mathcal{CM}_{(i+1)\%N} \in \mathcal{O}$, $\forall 0 \leq i \leq N-1$.
\end{definition}

\begin{figure}[h]
\centering
\subfloat[][Communication 1 to 1]{
	\label{subfig:comm_simple}
	\includegraphics[width=0.25\textwidth]{comm_11.png}
}
\hspace{0.05\textwidth}%
\subfloat[][Communication 1 to N]{%
	\label{subfig:comm_diff}
	\includegraphics[width=0.25\textwidth]{comm_1n.png}
}
\hspace{0.05\textwidth}%
\subfloat[][Cyclic communication]{%
	\label{subfig:comm_ring}
	\includegraphics[width=0.25\textwidth]{comm_ring2.png}
}
\caption[]{Graphic representation of communication operators}
\label{fig:comm}
\end{figure}

%These connections \begin{inparaenum}[1.]
%	\item assign an available unit of computation (typically a thread) to each solver, and
%	\item connect solvers each other.
%\end{inparaenum}
 

%We can also schedule non-communicating solvers:
%\[
%\left[\Sigma_0\cdot M_0, \Sigma_1\cdot M_1, ..., \Sigma_{N-1}\cdot M_{N-1}\right]
%\]

\section{Summarize}
\label{sec:posl_zum}

In this chapter \posl{} have been formally presented, as a Parallel--Oriented Solver Language to build meta-heuristic-based solver to solve \CSPs{}. This language provides a set of \oms{} useful to solve a wide range of constrained problems. It is also possible to create new ones, through the low-level framework in C++ programming language. \posl{} also provides a set of \opchs{}, essential features to share information between solvers.

One of the \posl's advantages is the possibility of creating, using an operator-based language, \ass{} remaining independent from concrete \bothmodules{}. It is then possible to create many different solvers builded upon the same \as{} by only instantiating different modules. It is also possible to create different \comstrs{} upon the same \soset{} by using \commopers{} that \posl{} provides.

In the next chapter, a detailed study of various communicating and non-communicating strategies is presented using some \CSPs{} as benchmarks. %In this study, is showed the efficacy of \posl{} to analyze quickly and easily these strategies.