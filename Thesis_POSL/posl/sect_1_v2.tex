There exist two types of basic modules in \posl: \INTROom{} and \INTROopch{}. A \om{} is basically a function and a \opch{} is also a function, but in contrast, it can receive information from two different sources: through input parameters or from outside, \ie by communicating with a module from another solver.

\subsection{Computation module}

%In this sub-section we expose the definition and the characteristics and the details of the \om, and give some examples. We explain how to create new \oms{} using the basic layer of the framework.

A \om{} is the most basic and abstract way to define a piece of computation. It is a function which receives an instance of a \posl{} data type as input, then executes an internal algorithm, and returns an instance of a \posl{} data type as output. The input and output types will characterize the computation module signature. It can be dy\-na\-mi\-cally replaced by (or combined with) other computation modules, since they can be transmitted to other solvers working in parallel. They are joined through operators defined in Section~\ref{sec:2ndstage}.

\defname{Computation Module}{\label{def:om}
A \om{} $Cm$ is a mapping defined by: 
\begin{equation}
\label{eq:om}
Cm:\mathcal{I} \rightarrow \mathcal{O}
\end{equation}
}

where $I$ and $O$, for instance, can be independently a set of configurations, a set of sets of configurations, a set of values of some data type, etc.

Consider a local search meta-heuristic solver. One of its \oms{} can be the function returning the set of configurations composing the neighborhood of a given configuration:

\begin{equation*}
Cm_{neighborhood}:I_1\times I_2\times\dots\times I_n \rightarrow 2^{I_1\times I_2\times\dots\times I_n}
\end{equation*}

\noindent where $I_i$ represents the definition domains of each variable of the input confi\-gura\-tion.

Figure~\ref{fig:om} shows an example of \om{} which receives a configuration $S$ and then computes the set $\mathcal{V}$ of its neighbor configurations $\left\{S^1, S^2, \dots, S^m\right\}$.

%\vspace{0.5cm}
\begin{figure}
	\centering	
	\includegraphics[width=0.7\linewidth]{OM.png}
	\caption{An example of a computation module computing a neighborhood}\label{fig:om}
\end{figure}

\subsubsection{Creating new \oms}
\label{subsubsec:creatingoms}

%To create new \oms{} we use C++. 
Each \om{} is written in C++. \posl{} provides a hierarchy of data types to work with %(see Appendix~\ref{app:diag}) 
and some abstract classes to inherit from, depending on the type of \om{} the user wants to create. These abstract classes represent {\it abstract} \oms{} and define a type of action to be executed. In the following we present the most important ones:

\begin{itemize}
\item \textbf{First Configuration Generation} $\rightarrow$ Represents \oms{} returning a configuration $s$, usually used for generating the starting configuration on local search meta-heuristics. 
\item \textbf{Neighborhood Function} $\rightarrow$ Represents \oms{} receiving a configuration $s$ as input and returning its neighborhood $\mathcal{V}(s)$ as output. These output configurations are efficiently stored in term of space.
\item \textbf{Selection Function} $\rightarrow$ Represents \oms{} receiving a neighborhood as input and selecting a configuration $s'$ from it as output. This function returns the pair $(s, s')$ containing both current and selected configuration.
\item \textbf{Decision Function} $\rightarrow$ Represents \oms{} receiving a couple af configurations encapsulated into a pair $(s, s')$, and returning the configuration to be the current one for the next iteration. 
\item \textbf{Processing Configuration Function} $\rightarrow$ Represents \oms{} receiving a configuration and returning another configuration as result of some arrangement, like for example, a reset. 
\end{itemize}

\subsection{Communication modules}

%In this sub-section we expose the definition and the characteristics and the details of the \opch, and give some examples. We explain how to create new \opchs{} using the basic layer of the framework.

A \opch{} is the component managing the information reception in the communication between solvers (we talk about information transmission in Section~\ref{sec:2ndstage}). They can interact with \oms{} through operators (see Figure~\ref{fig:och}).

A \opch{} can receive two types of information from an external solver: data or \oms{}. It is important to notice that by sending/receiving \oms, we mean sending/receiving the required information to identify and being able to instantiate the \om. For instance, an integer identifier.

In order to distinguish from the two types of \opchs, we will call \INTROdopch{} the \opch{} responsible for the data reception (Figure~\ref{subfig:doch}), and \INTROoopch{} is responsible for the reception and instantiation of \oms{} (Figure~\ref{subfig:ooch}).

\defname{Data Communication Module}{
A \emph{Data Communication Module} $Ch$ is a module that produces a mapping defined as follows: 
\begin{equation}
\label{def:dopench}
Ch:I\times \left\{D\cup \left\{NULL\right\}\right\} \rightarrow D \cup \left\{NULL\right\}
\end{equation}
No matter what the input $I$ is, it returns the information $D$ coming from an external solver.
}

\defname{Object Communication Module}{
	If we denote by $\mathbb{M}$ the space of all the \oms{} defined by Definition~\ref{def:om}, then an \emph{\oopch} $Ch$ is a module that produces and executes a \om{} coming from an external solver as follows:
	\begin{equation}
	\label{def:oopench}
	Ch:I\times\left\{\mathbb{M}\cup\left\{NULL\right\}\right\} \rightarrow O \cup \left\{NULL\right\}
	\end{equation}
It returns the output $O$ of the execution of the \om{} coming from an external solver, using $I$ as the input.
}%

\begin{figure}
	\centering
	\subfloat[][Data \opch]{
		\label{subfig:doch}
		\includegraphics[width=0.4\linewidth]{D_OCh_v2.png}
	}
	\hspace{0.05\textwidth}%
	\subfloat[][Object \opch]{%
		\label{subfig:ooch}
		\includegraphics[width=0.4\linewidth]{O_OCh_v2.png}
	}
	\caption[]{Communication module}
	\label{fig:och}
\end{figure}

Users can implement new computation and connection modules but \posl{} already contains many useful modules for solving a broad range of problems.

Due to the fact that \opchs{} receive information coming from outside without having control on them, it is necessary to define the {\it NULL} information, in order to denote the absence of information. If a Data Communication Module receives information, it is returned automatically. If an Object Communication Module receives a \om{}, it is instantiated and executed with the \opch's input and its result is returned. In both cases, if no available information exists (no communications performed), the \opch{} returns the {\it NULL} object.